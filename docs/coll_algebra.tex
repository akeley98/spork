% xelatex </dev/null coll_algebra.tex

\documentclass[11pt]{article}

\usepackage[letterpaper, portrait, margin=2.54cm]{geometry}
\usepackage{enumitem}
\usepackage{amsmath}
\usepackage{amssymb}
\usepackage{amsfonts}
\usepackage{placeins}
\usepackage{graphicx}
\usepackage{listings}
\usepackage{caption}
\usepackage{colortbl}
\usepackage[parfill]{parskip}
\usepackage[mathscr]{euscript}
\usepackage[usenames,dvipsnames,svgnames,table,hyperref]{xcolor}
\usepackage[hidelinks]{hyperref}
\usepackage{fontspec}
\usepackage{mdframed}
\usepackage{tikz}
\usetikzlibrary{shapes.geometric, arrows, positioning, calc}

\setsansfont{FreeSans}
\setmonofont{Ubuntu Mono}

\hyphenation{WebGL}

%\definecolor{webColor}{RGB}{0, 124, 204}
\definecolor{webColor}{RGB}{0, 108, 174}
\newcommand{\web}[1]{{\color{webColor} \small \url{#1}}}
\newcommand{\webText}[2]{{\color{webColor} \href{#2}{#1}}}
\newcommand{\email}[2]{{\small \color{webColor} \textsf{\href{mailto:#1@#2}{#1[at]#2}}}}
\definecolor{titleColor}{RGB}{0, 0, 0}
\ifdefined\RESUME
\newcommand{\myTitle}[1]{{\large \color{titleColor} \hspace{-12mm} \textbf{\textsf{#1}}}}
\else
\newcommand{\myTitle}[1]{{ \vspace{2mm} \large \color{titleColor} \hspace{-12mm} \textbf{\textsf{#1}} \vspace{2mm}}}
\fi
\definecolor{subColor}{RGB}{170, 149, 0}
\newcommand{\mySub}[1]{{\color{subColor}\hspace{-9mm} \textsf{#1}}}
\definecolor{keyColor}{RGB}{118, 185, 0}
\newcommand{\myKey}[1]{{\color{keyColor}\textbf{#1}}}

\definecolor{keyColorA}{RGB}{170, 149, 0}
\definecolor{keyColorB}{RGB}{170, 210, 0}
\newcommand{\myKeyA}[1]{\textbf{\color{keyColorA}#1}}
\newcommand{\myKeyB}[1]{\textbf{\color{keyColorB}#1}}

\definecolor{mainColor}{RGB}{170, 0, 170}
\definecolor{minorColor}{RGB}{0, 149, 232}
\newcommand{\mainKey}[1]{\textbf{\color{mainColor}#1}}
\newcommand{\minorKey}[1]{\textbf{\color{minorColor}#1}}
\newcommand{\mainSub}[1]{{\color{mainColor}\hspace{-9mm} \textsf{#1}}}
\newcommand{\minorSub}[1]{{\color{minorColor}\hspace{-9mm} \textsf{#1}}}

\definecolor{hookColor}{RGB}{204, 0, 255}
\newcommand{\hook}[1]{{\color{hookColor}\textbf{#1}}}
\definecolor{flaggedBoxColor}{RGB}{255, 204, 204}
\newcommand{\flagged}[1]{{\color{black}\colorbox{flaggedBoxColor}{#1}}}

\definecolor{lightttColor}{RGB}{69, 69, 80}
\newcommand{\lighttt}[1]{{\color{lightttColor}\texttt{#1}}}
\newcommand{\graytt}[1]{{\color{gray}\texttt{#1}}}
\newcommand{\blacktt}[1]{{\color{black}\texttt{#1}}}

% Red and blue boxes have bright text color.
% Yellow, green, and violet have intentionally muted text colors.
% They used to be black, but it looks slightly better with a tiny bit of color.
\definecolor{redBoxFg}{RGB}{224, 0, 0}
\definecolor{redBoxBg}{RGB}{255, 211, 225}
\newcommand{\redBox}[1]{{\color{redBoxFg}\colorbox{redBoxBg}{#1}}}
\definecolor{yellowBoxFg}{RGB}{89, 89, 0}
\definecolor{yellowBoxBg}{RGB}{255, 232, 0}
\newcommand{\yellowBox}[1]{{\color{yellowBoxFg}\colorbox{yellowBoxBg}{#1}}}
\definecolor{greenBoxFg}{RGB}{0, 89, 63}
\definecolor{greenBoxBg}{RGB}{179, 232, 160}
\newcommand{\greenBox}[1]{{\color{greenBoxFg}\colorbox{greenBoxBg}{#1}}}
\definecolor{blueBoxFg}{RGB}{0, 97, 232}
\definecolor{blueBoxBg}{RGB}{224, 232, 255}
\newcommand{\blueBox}[1]{{\color{blueBoxFg}\colorbox{blueBoxBg}{#1}}}
\definecolor{violetBoxFg}{RGB}{108, 63, 124}
\definecolor{violetBoxBg}{RGB}{218, 204, 255}
\newcommand{\violetBox}[1]{{\color{violetBoxFg}\colorbox{violetBoxBg}{#1}}}

\mdfdefinestyle{MyFrame}{%
    linecolor=black,
    outerlinewidth=0pt,
    linewidth=0pt,
    innertopmargin=2.7pt,
    innerbottommargin=0pt,
    innerrightmargin=0pt,
    innerleftmargin=0pt,
        leftmargin = 0pt,
        rightmargin = 0pt}


\renewcommand*{\theenumi}{\alph{enumi}}
\renewcommand*\labelenumi{(\theenumi)}
\renewcommand*{\theenumii}{\roman{enumii}}
\renewcommand*\labelenumii{\theenumii.}

\sloppy
\newcommand{\abs}[1]{\left \vert #1 \right\vert}

% Some of these links are now dead...
\newcommand{\mbarrier}{\webText{mbarrier}{https://docs.nvidia.com/cuda/parallel-thread-execution/\#parallel-synchronization-and-communication-instructions-mbarrier}}
\newcommand{\cpAsync}{\webText{cp.async}{https://docs.nvidia.com/cuda/parallel-thread-execution/\#data-movement-and-conversion-instructions-cp-async}}
\newcommand{\cpAsyncBulk}{\webText{cp.async.bulk}{https://docs.nvidia.com/cuda/parallel-thread-execution/\#data-movement-and-conversion-instructions-cp-async-bulk}}
\newcommand{\fenceProxyAsync}{\webText{fence.proxy.async}{https://docs.nvidia.com/cuda/parallel-thread-execution/\#asynchronous-warpgroup-level-matrix-async-proxy}}
\newcommand{\wgmma}{\webText{wgmma}{https://docs.nvidia.com/cuda/parallel-thread-execution/\#asynchronous-warpgroup-level-matrix-instructions}}
\newcommand{\hopperBlog}{\webText{NVIDIA Hopper Architecture In-Depth}{https://developer.nvidia.com/blog/nvidia-hopper-architecture-in-depth/}}
\newcommand{\expectTxOperation}{\webText{expect-tx operation}{https://docs.nvidia.com/cuda/parallel-thread-execution/\#parallel-synchronization-and-communication-instructions-mbarrier-expect-tx-operation}}
\newcommand{\completeTxOperation}{\webText{complete-tx operation}{https://docs.nvidia.com/cuda/parallel-thread-execution/\#parallel-synchronization-and-communication-instructions-mbarrier-complete-tx-operation}}
\newcommand{\wgmmaFence}{\webText{wgmma.fence}{https://docs.nvidia.com/cuda/parallel-thread-execution/\#asynchronous-warpgroup-level-matrix-instructions-wgmma-fence}}


\tikzstyle{redsmallnode} = [rectangle, minimum width=1.25cm, minimum height=1cm, text centered, text width=1.25cm, draw=redBoxFg, fill=redBoxBg, text=redBoxFg]
\tikzstyle{bluesmallnode} = [rectangle, minimum width=1.25cm, minimum height=1cm, text centered, text width=1.25cm, draw=blueBoxFg, fill=blueBoxBg, text=blueBoxFg]
\tikzstyle{smallnode} = [rectangle, minimum width=1.25cm, minimum height=1cm, text centered, text width=1.25cm, draw=black, fill=white]
\tikzstyle{smallishnode} = [rectangle, minimum width=2cm, minimum height=1cm, text centered, text width=2cm, draw=black, fill=white]
\tikzstyle{normalnode} = [rectangle, minimum width=3cm, minimum height=1cm, text centered, text width=3cm, draw=black, fill=white]
\tikzstyle{widenode} = [rectangle, minimum width=62mm, minimum height=8mm, text centered, text width=62mm, draw=black, fill=white]
\tikzstyle{bignode} = [rectangle, minimum width=3.5cm, minimum height=2cm, text centered, text width=3cm, draw=black, fill=white]
\tikzstyle{arrow} = [thick,->,>=stealth]
\tikzstyle{line} = [thick]


\begin{document}
\myTitle{Exo GPU (Spork) Collective Algebra}

Thoughts on how to formally model subdividing groups of threads, thread indexing, and thread convergence (e.g. for barriers or warp/warpgroup tensor core MMA).

\filbreak
\mainKey{Collective Unit:} A ``unit of measurement'' for threads arranged in a certain shape.
For example ``32 consecutive threads'' is a collective unit.
This would be a warp if aligned, but in Spork, alignment is not generally a property of a collective unit.
We only check alignment as an additional requirement when matching collective units for lowering \lighttt{instr} and barriers.

\filbreak
Nevertheless, we use familiar terms like ``warp'' for user-facing code; this is only an abuse-of-notation in the rare case that the user chooses to misalign threads.

\filbreak
Collective units need not be contiguous.
For example, a quadpair -- two groups of four threads each, offset by 16 from each other -- is a collective unit, and is distinct in shape from 8 contiguous threads.

\filbreak
\mainKey{Thread Collective:} A grouping of threads dynamically created at runtime.
The shape of the group is described by a collective unit.
For example, threads 0-31, threads 32-63, threads 64-95 ... are all distinct thread collectives that share the collective unit ``warp''.
Threads [0, 3] $\cup$ [16, 19], threads [4, 7] $\cup$ [20, 23], etc. are distinct thread collectives with collective unit ``quadpair''.

\filbreak
A parallel-for loop causes a distinct thread collective to be assigned to execute each iteration.

\filbreak
\mainKey{Top-level Collective:} A CUDA cluster, including the degenerate case of a single CTA if \lighttt{clusterDim = 1} (as is usually the case).
This terminology can generalize to other APIs.

\filbreak
\mainKey{Collective Tiling:} A collective tiling either describes the top-level collective, or describes how the thread collectives for executing statements within a parallel for loop or specialization statement are generated.
Collective tilings are arranged in a parent-child relationship, with each parallel for loop creating a new child that suballocates threads from the parent, and with the top-level parent being the top-level collective.

\filbreak
The primary purpose of this document is to build up to a full description of the collective tiling, and the principles (but not necessarily the imperative details) of how it's constructed.
\textbf{Throughout the Exo codebase -- for both internal and user-facing code -- } we will use ``\blacktt{coll}'' consistently as an abbreviation for ``collective''.
The word ``collective'' is too long, and if we don't create an official abbreviation, the temptation too create incompatible, ad-hoc abbreviations is just too high.

\filbreak
\myTitle{Domain}

The first thing we'll do is arrange the threads of the top-level collective into an N-dimensional ``domain grid''.
The \myKeyA{full domain} describes the size along each dimension of this grid.
For a CUDA cluster, this is (\lighttt{clusterDim}, \lighttt{blockDim}).
This could generalize to non-CUDA APIs, but there's a built-in assumption that the left-most dimension is the ``most significant''.

A domain must consist only of positive integers, and \textit{should} contain only values at least 2.
However, 1 values may result when substituting parameters (as would occur in the common case that \lighttt{clusterDim = 1}).
We can fix this later, as part of domain completion.

\filbreak
The thread count of a domain is the product of its coordinates.
A \myKeyA{partial domain} is a domain whose total thread count is lower than that of the top-level collective.
For example, \lighttt{(blockDim,)} is a partial domain if \lighttt{clusterDim != 1}.

\filbreak
Jumping ahead somewhat, physical threads are arranged in the domain grid by the top-level collective tiling's \myKeyA{intra-box expressions}.
For a CUDA cluster, these expressions are (\lighttt{blockIdx.x \% clusterDim}, \lighttt{threadIdx.x}).

\filbreak
\mainSub{Domain Completion}

Collective units and collective tilings are described in terms of their domain.
Often, their domains are not the same.
We reconcile this with domain completion, translating both to a common domain.
Domain completion consists of (up to) 3 steps in order.

\filbreak
Here, we only describe the steps for modifying the domain.
Whenever we modify the $n^{th}$ coordinate of a domain, we modify in parallel the $n^{th}$ coordinate of each attribute of a collective unit or collective tiling.
We'll describe this later.

\filbreak
\mainKey{Partial Prepend:} If the source domain is partial, we prepend $\frac{t}{s}$, where $t$ is the thread count of the full domain and $s$ is the thread count of the partial domain.
If this is not an integer, this step fails.

Side note, this is the main place where the left-most $\equiv$ ``most significant'' assumption is made.

\filbreak
\mainKey{Coordinate Removal:} Each domain coordinate that's 1 is removed.

\filbreak
\mainKey{Coordinate Splitting:} Domain coordinate values $c$ may be split into two consecutive coordinates $(\frac{c}{f}, f)$, where $f$ is the \myKeyA{splitting factor}.

\filbreak
\mainSub{Domain Completion Example}

Let's say we have two domains

\textbf{A:} (1, 64, 16)\\
\textbf{B:} (16, 128)

with \textbf{A} being a \myKeyA{partial domain} ($s = 1024$ threads) and \textbf{B} being a \myKeyA{full domain} ($t = 2048$ threads).
This is not very realistic for the CUDA use case; it's just for illustration.

\filbreak
\mainKey{Partial Prepend:} Since \textbf{A} is a partial domain, we prepend $2 = \frac{t}{s}$.

$\downarrow$ {\color{grayttColor} (1, 64, 16)} \\
\textbf{A:} (\blueBox{2}, 1, 64, 16)

$\downarrow$ {\color{grayttColor} (16, 128)} \\
\textbf{B:} (16, 128)

\filbreak
\mainKey{Coordinate Removal:} Remove the redundant 1 in \textbf{A}.

$\downarrow$ {\color{grayttColor} (2, \redBox{1}, 64, 16)} \\
\textbf{A:} (2, 64, 16)

$\downarrow$ {\color{grayttColor} (16, 128)} \\
\textbf{B:} (16, 128)

\filbreak
\mainKey{Coordinate Splitting:} Multiple splits are required to reconcile \textbf{A} and \textbf{B}.

$\downarrow$ {\color{grayttColor} (2, \violetBox{64}, 16)} \\
\textbf{A:} (2, \violetBox{8, 8}, 16) \\
Splitting factor: \violetBox{8}

$\downarrow$ {\color{grayttColor} (\blueBox{16}, \greenBox{128})} \\
\textbf{B:} (\blueBox{2, 8}, \greenBox{8, 16}) \\
Splitting factors: \blueBox{8}, \greenBox{16}

\filbreak
\myTitle{Collective Unit}

A collective unit consists of a \myKeyA{domain} and \myKeyA{box} size.
The box is \lighttt{Optional[int]}-valued and has the same dimension as the domain.
If the box contains a None coordinate, it's said to be \myKeyA{agnostic} on that dimension, and can't match any real thread collective.

\filbreak
Otherwise, given a certain domain, the collective unit for a thread collective is defined by the said domain and the shape of the filled axis-aligned box the threads form when arranged in the domain grid.\footnote{If the threads don't form a filled box, no collective unit with the given domain can match the thread collective.}
For example, with the domain being (\lighttt{clusterDim}, \lighttt{blockDim}), the box for a single warp is (1, 32), while the box for ``one warp chosen from each CTA in the cluster'' is (\lighttt{clusterDim}, 32) \textbf{(figure \ref{fig:units})}.

\filbreak
Due to domain completion, the ``warp'' collective unit can be described with domain=(\lighttt{blockDim},), box=(32,).
This is (almost) equivalent to the warp given in the example.

\begin{figure*}[!b]
\sffamily
\begin{tikzpicture}[node distance=2mm]
\node(t000) [smallnode] {0,0};
\node(t001) [smallnode, right=of t000] {0,1};
\node(t002) [smallnode, right=of t001] {0,2};
\node(t031) [smallnode, right=of t002, xshift=4mm] {0,31};
\draw [dotted] (t002) -- (t031);
\node(t032) [bluesmallnode, right=of t031] {0,32};
\node(t033) [bluesmallnode, right=of t032] {0,33};
\node(t034) [bluesmallnode, right=of t033] {0,34};
\node(t063) [bluesmallnode, right=of t034, xshift=4mm] {0,63};
\draw [dotted] (t034) -- (t063);

\node(t100) [redsmallnode, below=of t000] {1,0};
\node(t101) [redsmallnode, right=of t100] {1,1};
\node(t102) [redsmallnode, right=of t101] {1,2};
\node(t131) [redsmallnode, right=of t102, xshift=4mm] {1,31};
\draw [dotted] (t102) -- (t131);
\node(t132) [bluesmallnode, right=of t131] {1,32};
\node(t133) [bluesmallnode, right=of t132] {1,33};
\node(t134) [bluesmallnode, right=of t133] {1,34};
\node(t163) [bluesmallnode, right=of t134, xshift=4mm] {1,63};
\draw [dotted] (t134) -- (t163);

\node(t200) [smallnode, below=of t100] {2,0};
\node(t201) [smallnode, right=of t200] {2,1};
\node(t202) [smallnode, right=of t201] {2,2};
\node(t231) [smallnode, right=of t202, xshift=4mm] {2,31};
\draw [dotted] (t202) -- (t231);
\node(t232) [bluesmallnode, right=of t231] {2,32};
\node(t233) [bluesmallnode, right=of t232] {2,33};
\node(t234) [bluesmallnode, right=of t233] {2,34};
\node(t263) [bluesmallnode, right=of t234, xshift=4mm] {2,63};
\draw [dotted] (t234) -- (t263);

\node(t300) [smallnode, below=of t200] {3,0};
\node(t301) [smallnode, right=of t300] {3,1};
\node(t302) [smallnode, right=of t301] {3,2};
\node(t331) [smallnode, right=of t302, xshift=4mm] {3,31};
\draw [dotted] (t302) -- (t331);
\node(t332) [bluesmallnode, right=of t331] {3,32};
\node(t333) [bluesmallnode, right=of t332] {3,33};
\node(t334) [bluesmallnode, right=of t333] {3,34};
\node(t363) [bluesmallnode, right=of t334, xshift=4mm] {3,63};
\draw [dotted] (t334) -- (t363);

\node(legend) [smallnode, text width=8cm, draw=none, above=of t032] {\lighttt{(blockIdx.x \% clusterDim, threadIdx.x)}};
\end{tikzpicture}
\caption{With \lighttt{clusterDim = 4}, \lighttt{blockDim = 64}, and domain = $(4, 64)$, we highlight examples of thread collectives with collective units \redBox{single warp} (box $(1, 32)$) and \blueBox{warp per CTA} (box $(\lighttt{clusterDim}, 32)$). Note a single row of this grid corresponds to one CTA in a cluster.}
\label{fig:units}
\end{figure*}

\filbreak
\mainSub{Collective Unit Domain Completion}

When doing domain completion, the corresponding steps for the box attribute are:

\filbreak
\mainKey{Partial Prepend:} If we are matching collective units (e.g. to enforce an \lighttt{instr}'s requirements), prepend 1.
Otherwise, prepend None.

\filbreak
\mainKey{Coordinate Removal:} All removed coordinate values must be 1.

\filbreak
\mainKey{Coordinate Splitting:} A None value is split into (None, None).
An integer value $c$ splits into $(1, c)$ if $c < f$; otherwise, it splits into $(\frac{c}{f}, f)$.
If the results are not integers, the domain completion fails.

\filbreak
If we were accepting unaligned thread collectives, collective unit domain completion (as an unintended side-effect) makes the collective unit stricter, as some unaligned thread collectives will no longer form valid boxes in the new domain.
Domain completion won't affect matching of aligned thread collectives.

\end{document}
