% xelatex </dev/null wgmma_dirty_laundry.tex

\documentclass[11pt]{article}

\usepackage[letterpaper, portrait, margin=2.54cm]{geometry}
\usepackage{enumitem}
\usepackage{amsmath}
\usepackage{amssymb}
\usepackage{amsfonts}
\usepackage{placeins}
\usepackage{graphicx}
\usepackage{listings}
\usepackage{caption}
\usepackage{colortbl}
\usepackage[parfill]{parskip}
\usepackage[mathscr]{euscript}
\usepackage[usenames,dvipsnames,svgnames,table,hyperref]{xcolor}
\usepackage[hidelinks]{hyperref}
\usepackage{fontspec}
\usepackage{mdframed}
\usepackage{tikz}
\usetikzlibrary{shapes.geometric, arrows, positioning, calc}

\setsansfont{FreeSans}
\setmonofont{Ubuntu Mono}

\hyphenation{WebGL}

%\definecolor{webColor}{RGB}{0, 124, 204}
\definecolor{webColor}{RGB}{0, 108, 174}
\newcommand{\web}[1]{{\color{webColor} \small \url{#1}}}
\newcommand{\webText}[2]{{\color{webColor} \href{#2}{#1}}}
\newcommand{\email}[2]{{\small \color{webColor} \textsf{\href{mailto:#1@#2}{#1[at]#2}}}}
\definecolor{titleColor}{RGB}{0, 0, 0}
\ifdefined\RESUME
\newcommand{\myTitle}[1]{{\large \color{titleColor} \hspace{-12mm} \textbf{\textsf{#1}}}}
\else
\newcommand{\myTitle}[1]{{ \vspace{2mm} \large \color{titleColor} \hspace{-12mm} \textbf{\textsf{#1}} \vspace{2mm}}}
\fi
\definecolor{subColor}{RGB}{170, 149, 0}
\newcommand{\mySub}[1]{{\color{subColor}\hspace{-9mm} \textsf{#1}}}
\definecolor{keyColor}{RGB}{118, 185, 0}
\newcommand{\myKey}[1]{{\color{keyColor}\textbf{#1}}}

\definecolor{keyColorA}{RGB}{170, 149, 0}
\definecolor{keyColorB}{RGB}{170, 210, 0}
\newcommand{\myKeyA}[1]{\textbf{\color{keyColorA}#1}}
\newcommand{\myKeyB}[1]{\textbf{\color{keyColorB}#1}}

\definecolor{mainColor}{RGB}{170, 0, 170}
\definecolor{minorColor}{RGB}{0, 149, 232}
\newcommand{\mainKey}[1]{\textbf{\color{mainColor}#1}}
\newcommand{\minorKey}[1]{\textbf{\color{minorColor}#1}}
\newcommand{\mainSub}[1]{{\color{mainColor}\hspace{-9mm} \textsf{#1}}}
\newcommand{\minorSub}[1]{{\color{minorColor}\hspace{-9mm} \textsf{#1}}}

\definecolor{hookColor}{RGB}{204, 0, 255}
\newcommand{\hook}[1]{{\color{hookColor}\textbf{#1}}}
\definecolor{flaggedBoxColor}{RGB}{255, 204, 204}
\newcommand{\flagged}[1]{{\color{black}\colorbox{flaggedBoxColor}{#1}}}

\definecolor{lightttColor}{RGB}{69, 69, 80}
\definecolor{grayttColor}{RGB}{144, 144, 160}
\newcommand{\lighttt}[1]{{\color{lightttColor}\texttt{#1}}}
\newcommand{\graytt}[1]{{\color{grayttColor}\texttt{#1}}}
\newcommand{\blacktt}[1]{{\color{black}\texttt{#1}}}

% Red and blue boxes have bright text color.
% Yellow, green, and violet have intentionally muted text colors.
% They used to be black, but it looks slightly better with a tiny bit of color.
\definecolor{redBoxFg}{RGB}{224, 0, 0}
\definecolor{redBoxBg}{RGB}{255, 211, 225}
\newcommand{\redBox}[1]{{\color{redBoxFg}\colorbox{redBoxBg}{#1}}}
\definecolor{yellowBoxFg}{RGB}{89, 89, 0}
\definecolor{yellowBoxBg}{RGB}{255, 232, 0}
\newcommand{\yellowBox}[1]{{\color{yellowBoxFg}\colorbox{yellowBoxBg}{#1}}}
\definecolor{greenBoxFg}{RGB}{0, 89, 63}
\definecolor{greenBoxBg}{RGB}{179, 232, 160}
\newcommand{\greenBox}[1]{{\color{greenBoxFg}\colorbox{greenBoxBg}{#1}}}
\definecolor{blueBoxFg}{RGB}{0, 97, 232}
\definecolor{blueBoxBg}{RGB}{224, 232, 255}
\newcommand{\blueBox}[1]{{\color{blueBoxFg}\colorbox{blueBoxBg}{#1}}}
\definecolor{violetBoxFg}{RGB}{108, 63, 124}
\definecolor{violetBoxBg}{RGB}{218, 204, 255}
\newcommand{\violetBox}[1]{{\color{violetBoxFg}\colorbox{violetBoxBg}{#1}}}

\definecolor{codecommentColor}{RGB}{124, 170, 0}
\newcommand{\codecomment}[1]{{\color{codecommentColor}\textsf{#1}}}

\mdfdefinestyle{MyFrame}{%
    linecolor=black,
    outerlinewidth=0pt,
    linewidth=0pt,
    innertopmargin=2.7pt,
    innerbottommargin=0pt,
    innerrightmargin=0pt,
    innerleftmargin=0pt,
        leftmargin = 0pt,
        rightmargin = 0pt}


\renewcommand*{\theenumi}{\alph{enumi}}
\renewcommand*\labelenumi{(\theenumi)}
\renewcommand*{\theenumii}{\roman{enumii}}
\renewcommand*\labelenumii{\theenumii.}

\newcommand{\fixminipage}{\raggedright \setlength{\parskip}{0.3\baselineskip}}
\newcommand{\codeminipage}{\raggedright \setlength{\parskip}{0\baselineskip}}

\sloppy
\newcommand{\abs}[1]{\left \vert #1 \right\vert}

% Some of these links are now dead...
\newcommand{\mbarrier}{\webText{mbarrier}{https://docs.nvidia.com/cuda/parallel-thread-execution/\#parallel-synchronization-and-communication-instructions-mbarrier}}
\newcommand{\cpAsync}{\webText{cp.async}{https://docs.nvidia.com/cuda/parallel-thread-execution/\#data-movement-and-conversion-instructions-cp-async}}
\newcommand{\cpAsyncBulk}{\webText{cp.async.bulk}{https://docs.nvidia.com/cuda/parallel-thread-execution/\#data-movement-and-conversion-instructions-cp-async-bulk}}
\newcommand{\fenceProxyAsync}{\webText{fence.proxy.async}{https://docs.nvidia.com/cuda/parallel-thread-execution/\#asynchronous-warpgroup-level-matrix-async-proxy}}
\newcommand{\wgmma}{\webText{wgmma}{https://docs.nvidia.com/cuda/parallel-thread-execution/\#asynchronous-warpgroup-level-matrix-instructions}}
\newcommand{\hopperBlog}{\webText{NVIDIA Hopper Architecture In-Depth}{https://developer.nvidia.com/blog/nvidia-hopper-architecture-in-depth/}}
\newcommand{\expectTxOperation}{\webText{expect-tx operation}{https://docs.nvidia.com/cuda/parallel-thread-execution/\#parallel-synchronization-and-communication-instructions-mbarrier-expect-tx-operation}}
\newcommand{\completeTxOperation}{\webText{complete-tx operation}{https://docs.nvidia.com/cuda/parallel-thread-execution/\#parallel-synchronization-and-communication-instructions-mbarrier-complete-tx-operation}}
\newcommand{\wgmmaFence}{\webText{wgmma.fence}{https://docs.nvidia.com/cuda/parallel-thread-execution/\#asynchronous-warpgroup-level-matrix-instructions-wgmma-fence}}


\begin{document}

\mainSub{``abstract''}

To generate optimal code, \lighttt{ptxas} has to be able to \textit{statically} verify that your usage of wgmma is correct.
How \lighttt{ptxas} does this isn't documented.
I sarcastically try to explain what \lighttt{ptxas}'s expectations are.
I don't work for NVIDIA (anymore), so this is all based on personal experience and comes with NO WARRANTY OF MERCHANTABILITY, et cetera.

\myTitle{wgmma (sm\_90a MMA) and its Dirty Laundry}

\hfill\textsf{(How do you \textit{really} program an H100 with PTX?!!?)}

Let's just say hypothetically that you're creating your own PL targetting tensor cores (wgmma) on the H100, so you have to program with PTX and can't take NVIDIA's self-serving advice to ``strongly [prefer] that device kernels utilize this complex feature set through CUTLASS, a collection of CUDA C++ template abstractions...''.
This advice is self-serving because it gives NVIDIA an excuse to poorly document PTX.
You may think you can read the PTX guide to understand how to generate wgmma code.
\lightsf{WRONG}.

If you are like me, you may follow all the requirements specified by the PTX doc, and still quickly run into issues where \lighttt{ptxas} generates this warning:

\filbreak
\lighttt{Potential Performance Loss: \redBox{wgmma.mma\_async instructions are serialized due to} non wgmma instructions defining accumulator registers of a wgmma between start and end of the pipeline stage in the function ...}

(other reasons may also be given)

and if you look at the SASS (\lighttt{cuobjdump --dump-sass my-file.o > my-file.sass}), you will see performance-killing \lighttt{WARPGROUP.DEPBAR} and \lighttt{WARPGROUP.ARRIVE} instructions for each \lighttt{HGMMA} (wgmma) instruction.

\filbreak
I will not be covering swizzling (yet) in this document, which is a topic numerous people have complained about.
Instead I'll be addressing the above warning.

\filbreak
\mainSub{Trivial Fixes}

You will find hits for the above warning that suggest these fixes; this may be enough to fix your problem, but was not enough for mine:
\begin{itemize}
  \item Use nvcc 12.3 or higher due to 12.1 and 12.2 compiler bugs
  \item Enable the optimizer with \lighttt{-O2} (you don't say...)
  \item Define \lighttt{NDEBUG}
\end{itemize}
Furthermore, if the reason for the serialized instructions mentions ``insufficient register resources'', the problem is fundamentally different from what I'm about to describe, and you have little choice but to cut register usage.

\filbreak
\myTitle{Background: Why is this happening?}

\mainSub{Pipeline Stage}

Most messages mention a ``pipeline stage''; this crucial concept is seemingly not documented.
A ``pipeline stage'' consists of
\begin{itemize}
  \item \lighttt{wgmma.fence.sync.aligned;} [begin pipeline stage]
  \item A (hopefully unrolled) loop of \lighttt{wgmma.mma\_async} instructions
  \item \lighttt{wgmma.commit\_group.sync.aligned;} [end pipeline stage]
\end{itemize}
The compiler must be able to \textit{statically} recognize this pipeline stage pattern.

\filbreak
\minorSub{Big Problem 1/2: Register Allocation (Optional Reading)}

If your pipeline stage pattern is correct, and you don't alter the accumulator registers between the pipeline stage begin and end, then the warning that there are ``non wgmma instructions defining accumulator registers of a wgmma between start and end of the pipeline stage'' acts as a misleading catch-all warning that ``static analysis failed somewhere''.

\filbreak
The core expectations fail here, in my opinion, is that when you program at an ISA/assembly level, typically, the consequences for violating correct usage is that your program has incorrect behavior (e.g. for wgmma, incorrect usage could be a race condition on a wgmma register operand).
In other words, as long as your program \textit{dynamically} uses the ISA as documented, things will work correctly, even if it's not proven that usage is correct (i.e. statically proven to be correct).

\filbreak
However, PTX is not an ISA, SASS is, and the consequences of using wgmma ``incorrectly'' (as defined by \lighttt{ptxas}'s static analysis) is that \lighttt{ptxas} generates a suboptimal binary (with the \lighttt{WARPGROUP.ARRIVE}), as opposed to an incorrect binary.
I \textbf{speculate} this static analysis is needed because of register allocation. \lighttt{ptxas} has to be able to reason about the full lifetime (including usage by not-yet-retired asynchronous instructions) of each virtual PTX register in order to allocate values to physical registers, and if \lighttt{ptxas} can't statically analyze your usage pattern, it would risk causing a race condition on a physical register that does not occur at the PTX level.

\filbreak
\minorSub{Big Problem 2/2: scale-d is a LIE (Optional Reading)}

Another hugely misleading feature is the \lighttt{scale-d} operand to \lighttt{wgmma.mma\_async}.
This controls whether zero-initialization occurs:
\begin{align*}
  & D = AB & \lightsf{(scale-d = 0)} \\
  & D = AB + D & \lightsf{(scale-d = 1)}
\end{align*}
where the expected dependency chain is that we start with a single \lightsf{scale-d=0} \lighttt{wgmma.mma\_async} instruction that initializes the accumulator, then we accumulate with \lightsf{scale-d=1}.

\filbreak
This \lighttt{scale-d} operand is given as a PTX predicate, which makes it look cheap.
This is not the case.
In reality, the two cases lower to different SASS instructions (examples operands given):
\begin{align*}
  & \lighttt{HGMMA.64x128x8.F32.TF32 \blueBox{R88}, gdesc[UR12], \redBox{RZ}, !UPT ;} & \lightsf{(scale-d = 0)} \\
  & \lighttt{HGMMA.64x128x8.F32.TF32 \blueBox{R88}, gdesc[UR12], \blueBox{R88} ;} & \lightsf{(scale-d = 1)}
\end{align*}
As a result, \lighttt{ptxas} seems to give the (100\% misleading) ``non wgmma instructions defining blah blah blah'' warning when it's unable to validate either of these two conditions:
\begin{itemize}
  \item It's not able to resolve \lightsf{scale-d} to a constant.
  \item It's unable to prove that the \lightsf{scale-d = 0} case (i.e. the usage of \redBox{\texttt{RZ}}) only occurs as the first instruction in a dependency chain of \lighttt{wgmma.mma\_async} instructions.
\end{itemize}
The latter condition isn't even required by the PTX docs and seems to be an entirely undocumented expectation.

\filbreak
\myTitle{Failure Modes \& Fixes}

I catalog all ``wgmma.mma\_async instructions are serialized'' failures I've encountered and suggested fixes.
This is based on analyzing this code: \webText{cutlass/include/cutlass/gemm/collective/sm90\_mma\_tma\_gmma\_ss\_warpspecialized.hpp}{https://github.com/NVIDIA/cutlass/blob/main/include/cutlass/gemm/collective/sm90_mma_tma_gmma_ss_warpspecialized.hpp}.

\filbreak
\mainSub{Arrays are Evil}

(This section only applies if you are using inline PTX in C++)

wgmma requires a huge number of registers.
It's tempting to e.g. define a tile of 128 registers as\\
\lighttt{struct D\_Matrix \{ float d[128]; \} d; }\\
and then pass the tile as\\
\lighttt{asm volatile ("wgmma.mma\_async ..." :: "+f"(d.d[0]), "+f"(d.d[1]), ...);}

\filbreak
With this approach, I experienced that nvcc would materialize the array in local memory, and generate loads and stores as needed to populate the register operands for \lighttt{wgmma.mma\_async}.
In this case, the compiler's accusation that ``non wgmma instructions [define] accumulator registers...'' is accurate at the PTX level, but not obvious at the C++ level.

\filbreak
If you are authoring the entire PTX file and not just using inline \lighttt{asm}, then this can't happen to you (unknowingly).
Otherwise, to fix this, write out each register manually...\\
\lighttt{struct D\_Matrix \{ float d0, d1, d2, ..., d127; \} d; }\\
\lighttt{asm volatile ("wgmma.mma\_async ..." :: "+f"(d.d0), "+f"(d.d1), ...);}

\filbreak
It doesn't seem like CuTe and CUTLASS follow my proposed rule 100\%, but at least for me, this was necessary.
Why take an unnecessary risk?

\filbreak
\textbf{\redBox{Caveat:}} I noticed this change (array to scalars) also has an effect on how the accumulator tile is written from registers to GMEM.
I am not sure if this is good.
There may be ways to use C++ arrays and not trigger the code lowering issue, but I am not sure how (\lighttt{pragma unroll}-ing everything wasn't enough, at least for me).

\filbreak
\mainSub{Special Casing Iteration 0}

Because of the issue with \lightsf{scale-d} and the \redBox{\texttt{RZ}} register, pseudocode of the form
{\color{lightttColor}
\begin{verbatim}
m, n = ...
scale_d = 0
for k1 in seq(0, SIZE_K/K1):
    wgmma.fence.sync.aligned()
    for k0 in seq(0, K1/K0):
        k = k1 * K1 + k0 * K0
        wgmma.mma_async(D[m:,n:], A[m:,k:], B[n:,k:], scale_d)
        scale_d = 1
    wgmma.commit_group.sync.aligned()
\end{verbatim}
}
\textbf{should} be transformed such that the initial iteration of the outer (\lighttt{k1}) loop is handled specially:
{\color{lightttColor}
\begin{verbatim}
m, n = ...
scale_d = 0
if True:                                # Special case for k1 = 0
    k1 = 0
    wgmma.fence.sync.aligned()
    for k0 in seq(0, K1/K0):            # Compiler knows scale_d = 0 iff k0 == 0
        k = k1 * K1 + k0 * K0           # Unrolling allows compiler to lower special RZ case
        wgmma.mma_async(D[m:,n:], A[m:,k:], B[n:,k:], scale_d)
        scale_d = 1
    wgmma.commit_group.sync.aligned()   # Compiler knows scale_d is always 1 now

for k1 in seq(1, SIZE_K/K1):            # Main loop for k1 >= 1
    wgmma.fence.sync.aligned()
    for k0 in seq(0, K1/K0):            # Unrolled, no usage of RZ
        k = k1 * K1 + k0 * K0
        wgmma.mma_async(D[m:,n:], A[m:,k:], B[n:,k:], scale_d)
        scale_d = 1
    wgmma.commit_group.sync.aligned()
\end{verbatim}
}

I say ``\textbf{should}'' because I've noticed nvcc will try to do this optimization implicitly, but only sometimes, causing code to mysteriously fail for no clear reason if you didn't know about this requirement.
It's better to just do it yourself (and, if you're authoring raw PTX, you have no choice but to do this yourself anyway).

\filbreak
\mainSub{Mandatory wait\_group 0}

The \lighttt{wgmma.wait\_group.sync.aligned \textit{N}} instruction waits for the $N^{th}$ prior executed wgmma pipeline stage to finish.
($N = 0$ waits for the just-executed \lighttt{wgmma.commit\_group.sync.aligned}).
Typically $N > 0$ so that you actually get asynchrony, and you don't issue this for the first $N$-many wgmma pipeline stages.
If $N = 1$, this special case may be combined with the special case for \lightsf{scale-d = 0}, but don't conflate the two problems.

\filbreak
At the end of the accumulation, though, you will probably isuse a \lighttt{wgmma.wait\_group.sync.aligned \textbf{0}} instruction to wait for the final result.
This needs to be done in a completely straightforward fashion -- \lighttt{ptxas} needs to be 110\% assured that all wgmma instructions have retired.
So don't do something clever like putting ``if this is the final k iteration, then wait-group 0'' inside your k-loop.
Just slap it down in unconditional code at the end of your loop.

\end{document}
