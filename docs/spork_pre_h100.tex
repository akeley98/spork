\documentclass[11pt]{article}

\usepackage[letterpaper, portrait, margin=2.54cm]{geometry}
\usepackage{enumitem}
\usepackage{amsmath}
\usepackage{amssymb}
\usepackage{amsfonts}
\usepackage{placeins}
\usepackage{graphicx}
\usepackage{listings}
\usepackage{caption}
\usepackage{colortbl}
\usepackage[parfill]{parskip}
\usepackage[mathscr]{euscript}
\usepackage[usenames,dvipsnames,svgnames,table,hyperref]{xcolor}
\usepackage[hidelinks]{hyperref}
\usepackage{fontspec}
\usepackage{mdframed}
\usepackage{tikz}
\usetikzlibrary{shapes.geometric, arrows, positioning, calc}

\setsansfont{FreeSans}
\setmonofont{Ubuntu Mono}

\hyphenation{WebGL}

%\definecolor{webColor}{RGB}{0, 124, 204}
\definecolor{webColor}{RGB}{0, 108, 174}
\newcommand{\web}[1]{{\color{webColor} \small \url{#1}}}
\newcommand{\webText}[2]{{\color{webColor} \href{#2}{#1}}}
\newcommand{\email}[2]{{\small \color{webColor} \textsf{\href{mailto:#1@#2}{#1[at]#2}}}}
\definecolor{titleColor}{RGB}{0, 0, 0}
\ifdefined\RESUME
\newcommand{\myTitle}[1]{{\large \color{titleColor} \hspace{-12mm} \textbf{\textsf{#1}}}}
\else
\newcommand{\myTitle}[1]{{ \vspace{2mm} \large \color{titleColor} \hspace{-12mm} \textbf{\textsf{#1}} \vspace{2mm}}}
\fi
\definecolor{subColor}{RGB}{170, 149, 0}
\newcommand{\mySub}[1]{{\color{subColor}\hspace{-9mm} \textsf{#1}}}
\definecolor{keyColor}{RGB}{118, 185, 0}
\newcommand{\myKey}[1]{{\color{keyColor}\textbf{#1}}}

\definecolor{keyColorA}{RGB}{170, 149, 0}
\definecolor{keyColorB}{RGB}{170, 210, 0}
\newcommand{\myKeyA}[1]{\textbf{\color{keyColorA}#1}}
\newcommand{\myKeyB}[1]{\textbf{\color{keyColorB}#1}}

\definecolor{mainColor}{RGB}{170, 0, 170}
\definecolor{minorColor}{RGB}{0, 149, 232}
\newcommand{\mainKey}[1]{\textbf{\color{mainColor}#1}}
\newcommand{\minorKey}[1]{\textbf{\color{minorColor}#1}}
\newcommand{\mainSub}[1]{{\color{mainColor}\hspace{-9mm} \textsf{#1}}}
\newcommand{\minorSub}[1]{{\color{minorColor}\hspace{-9mm} \textsf{#1}}}

\definecolor{hookColor}{RGB}{204, 0, 255}
\newcommand{\hook}[1]{{\color{hookColor}\textbf{#1}}}
\definecolor{flaggedBoxColor}{RGB}{255, 204, 204}
\newcommand{\flagged}[1]{{\color{black}\colorbox{flaggedBoxColor}{#1}}}

\definecolor{lightttColor}{RGB}{69, 69, 80}
\definecolor{grayttColor}{RGB}{144, 144, 160}
\newcommand{\lighttt}[1]{{\color{lightttColor}\texttt{#1}}}
\newcommand{\graytt}[1]{{\color{grayttColor}\texttt{#1}}}
\newcommand{\blacktt}[1]{{\color{black}\texttt{#1}}}

% Red and blue boxes have bright text color.
% Yellow, green, and violet have intentionally muted text colors.
% They used to be black, but it looks slightly better with a tiny bit of color.
\definecolor{redBoxFg}{RGB}{224, 0, 0}
\definecolor{redBoxBg}{RGB}{255, 211, 225}
\newcommand{\redBox}[1]{{\color{redBoxFg}\colorbox{redBoxBg}{#1}}}
\definecolor{yellowBoxFg}{RGB}{89, 89, 0}
\definecolor{yellowBoxBg}{RGB}{255, 232, 0}
\newcommand{\yellowBox}[1]{{\color{yellowBoxFg}\colorbox{yellowBoxBg}{#1}}}
\definecolor{greenBoxFg}{RGB}{0, 89, 63}
\definecolor{greenBoxBg}{RGB}{179, 232, 160}
\newcommand{\greenBox}[1]{{\color{greenBoxFg}\colorbox{greenBoxBg}{#1}}}
\definecolor{blueBoxFg}{RGB}{0, 97, 232}
\definecolor{blueBoxBg}{RGB}{224, 232, 255}
\newcommand{\blueBox}[1]{{\color{blueBoxFg}\colorbox{blueBoxBg}{#1}}}
\definecolor{violetBoxFg}{RGB}{108, 63, 124}
\definecolor{violetBoxBg}{RGB}{218, 204, 255}
\newcommand{\violetBox}[1]{{\color{violetBoxFg}\colorbox{violetBoxBg}{#1}}}

\definecolor{codecommentColor}{RGB}{124, 170, 0}
\newcommand{\codecomment}[1]{{\color{codecommentColor}\textsf{#1}}}

\mdfdefinestyle{MyFrame}{%
    linecolor=black,
    outerlinewidth=0pt,
    linewidth=0pt,
    innertopmargin=2.7pt,
    innerbottommargin=0pt,
    innerrightmargin=0pt,
    innerleftmargin=0pt,
        leftmargin = 0pt,
        rightmargin = 0pt}


\renewcommand*{\theenumi}{\alph{enumi}}
\renewcommand*\labelenumi{(\theenumi)}
\renewcommand*{\theenumii}{\roman{enumii}}
\renewcommand*\labelenumii{\theenumii.}

\newcommand{\fixminipage}{\raggedright \setlength{\parskip}{0.3\baselineskip}}
\newcommand{\codeminipage}{\raggedright \setlength{\parskip}{0\baselineskip}}

\sloppy
\newcommand{\abs}[1]{\left \vert #1 \right\vert}

% Some of these links are now dead...
\newcommand{\mbarrier}{\webText{mbarrier}{https://docs.nvidia.com/cuda/parallel-thread-execution/\#parallel-synchronization-and-communication-instructions-mbarrier}}
\newcommand{\cpAsync}{\webText{cp.async}{https://docs.nvidia.com/cuda/parallel-thread-execution/\#data-movement-and-conversion-instructions-cp-async}}
\newcommand{\cpAsyncBulk}{\webText{cp.async.bulk}{https://docs.nvidia.com/cuda/parallel-thread-execution/\#data-movement-and-conversion-instructions-cp-async-bulk}}
\newcommand{\fenceProxyAsync}{\webText{fence.proxy.async}{https://docs.nvidia.com/cuda/parallel-thread-execution/\#asynchronous-warpgroup-level-matrix-async-proxy}}
\newcommand{\wgmma}{\webText{wgmma}{https://docs.nvidia.com/cuda/parallel-thread-execution/\#asynchronous-warpgroup-level-matrix-instructions}}
\newcommand{\hopperBlog}{\webText{NVIDIA Hopper Architecture In-Depth}{https://developer.nvidia.com/blog/nvidia-hopper-architecture-in-depth/}}
\newcommand{\expectTxOperation}{\webText{expect-tx operation}{https://docs.nvidia.com/cuda/parallel-thread-execution/\#parallel-synchronization-and-communication-instructions-mbarrier-expect-tx-operation}}
\newcommand{\completeTxOperation}{\webText{complete-tx operation}{https://docs.nvidia.com/cuda/parallel-thread-execution/\#parallel-synchronization-and-communication-instructions-mbarrier-complete-tx-operation}}
\newcommand{\wgmmaFence}{\webText{wgmma.fence}{https://docs.nvidia.com/cuda/parallel-thread-execution/\#asynchronous-warpgroup-level-matrix-instructions-wgmma-fence}}


\tikzstyle{smallnode} = [rectangle, minimum width=1.25cm, minimum height=1cm, text centered, text width=1.25cm, draw=black, fill=white]
\tikzstyle{smallishnode} = [rectangle, minimum width=2cm, minimum height=1cm, text centered, text width=2cm, draw=black, fill=white]
\tikzstyle{normalnode} = [rectangle, minimum width=3cm, minimum height=1cm, text centered, text width=3cm, draw=black, fill=white]
\tikzstyle{widenode} = [rectangle, minimum width=62mm, minimum height=8mm, text centered, text width=62mm, draw=black, fill=white]
\tikzstyle{bignode} = [rectangle, minimum width=3.5cm, minimum height=2cm, text centered, text width=3cm, draw=black, fill=white]
\tikzstyle{smemnode} = [rectangle, minimum width=3cm, minimum height=1cm, text centered, text width=3cm, draw=keyColorB, fill=white]
\tikzstyle{gmemnode} = [rectangle, minimum width=3cm, minimum height=1cm, text centered, text width=3cm, draw=keyColorA, fill=white]
\tikzstyle{smallishsmemnode} = [rectangle, minimum width=2cm, minimum height=1cm, text centered, text width=2cm, draw=keyColorB, fill=white]
\tikzstyle{arrow} = [thick,->,>=stealth]
\tikzstyle{line} = [thick]

\begin{document}
\myTitle{Exo GPU (Spork) Status 2025-04-22}

As of today, I have a pretty much working prototype of Exo-GPU that supports A100 (sm\_80) features (non-bulk async copies of 4, 8, or 16 bytes, and warp-level tensor cores).
It is ``working'' to the extent that it's able to compile correct programs, but there is no synchronization checking, and limited checking for other kinds of mistakes.

The prototype's support for H100 (sm\_90a) features (TMA and wgmma.mma\_async) is fairly broken right now.
I'm able to compile a basic gemm for H100, but its performance is much worse than hand-written code.

\filbreak
\mainSub{GPU-specific Features Implemented}

\mainKey{Nascent Type Systems:} Nothing formal or anything, but the Exo CUDA backend is able to statically analyze for each statement:
\begin{itemize}
  \item \myKeyA{Collective Unit (coll\_unit):} arrangement of cooperating threads, e.g. \lighttt{cuda\_warp} (32 aligned threads); \lighttt{cuda\_warpgroup} (128 aligned threads).
  A single instance of a collective unit is a \myKeyA{thread collective}.
  \item \myKeyA{Collective Tiling (CollTiling):} tiling of thread collectives across threads of the cluster/CTA; may be non-trivial due to masked-out threads (warp specialization).
  \item \myKeyA{Actor Kind:} cpu, cuda\_classic, or one of the CudaAsync actor kinds.
  \item \myKeyA{Actor Signature:} Each read/write is annotated with an actor signature.
\end{itemize}

\filbreak
We enforce actor kind and collective unit requirements:
\begin{itemize}
  \item Memory types specify read, write, and allocation permissions per actor kind.
  \item Non-\lighttt{instr} writes and reduces must use a single thread.
  \item An \lighttt{instr} specifies its own custom requirements, and an \myKeyA{actor signature} per parameter.
\end{itemize}

\filbreak
\mainKey{CudaDeviceFunction Block:} Subtree compiled to a CUDA device function, with \lighttt{blockDim}, \lighttt{clusterDim}, and \lighttt{blocks\_per\_sm} (occupancy) specified.
The LoopIR-to-C compiler automatically handles compiling a CUDA device function, and passing arguments from host code to device code.
This changes the \myKeyA{actor kind} from CPU to cuda\_classic for child statements.

\filbreak
\mainKey{Grid Constants:} Scalars or fixed-size arrays copied from the CPU to the device function.

\filbreak
\mainKey{Shared Memory:} SMEM requires special compiler support to allocate.
I currently have a stack allocator, where allocate/free means increment/decrement the stack pointer (with special handling for the case that frees aren't in LIFO order).
The compiler deduces the maximum SMEM usage and requests that amount at CUDA device function launch time.
We will have to consider the synchronization challenges this poses -- shared memory must not be freed until all usages have retired.

\filbreak
\mainKey{CudaAsync Blocks:} Changes the actor kind from cuda\_classic to one of the async actor kinds.

\filbreak
\mainKey{cuda\_tasks Loops:} Distribute independent work items across clusters or CTAs.

\filbreak
\mainKey{cuda\_threads Loops:} Subdivide thread collective into child thread collectives with the specified collective unit.
This changes the \myKeyA{collective tiling} of child statements.

\filbreak
\mainKey{Split Barriers:} I can lower \lighttt{Arrive} and \lighttt{Await} statements to \myKeyA{commit group} or \myKeyA{mbarrier} synchronization, for both sm\_80 and sm\_90a features.

\filbreak
\mainKey{sm\_80 (A100) Features:} Non-bulk \lighttt{cp.async} and warp-level MMA.
I have a working gemm for sm\_80 using these features and \myKeyA{mbarrier}.

\filbreak
\mainKey{sm\_90a (H100) Features:} \lighttt{cp.async.bulk} (TMA) and \lighttt{wgmma.mma\_async}.
I only have a tiny subset of instructions (just for tf32) implemented.
TMA mostly works okay (minus required error checking), wgmma not so much.
I have a rant (wgmma\_dirty\_laundry.pdf) about some of the wgmma problems.

\filbreak
\mainKey{New Output Files:} If any CUDA device functions are compiled, the compiler additionally emits \lighttt{.cu} and \lighttt{.cuh} files.
The implementation of the CUDA device functions goes into the \lighttt{.cuh} (header) file, so the user may extract and re-use the generated device functions as they see fit, independent of Exo-generated CPU code
(dynamic linking support for CUDA is very poor).
In the likely case that the user doesn't want to DIY, they may rely on stub code in the \lighttt{.c} and \lighttt{.cu} files to launch the CUDA device functions for them.

\filbreak
\minorSub{Non-GPU-specific Features Implemented (Optional Reading)}

These features were added to support Exo-GPU, but conceivably could be useful in other contexts.

\filbreak
\minorKey{Loop Mode Object:} The binary \lighttt{seq}/\lighttt{par} choice for loop mode is generalized and externalized as \lighttt{LoopMode} objects, which themselves can be parameterized (e.g. the \lighttt{unit} for \lighttt{cuda\_threads} loops).

\filbreak
\minorKey{Pragma Unroll:} Building off the above feature, \lighttt{Seq} loop modes may have an optional \lighttt{pragma\_unroll} parameter, which causes the generated loop to include \lighttt{\#pragma unroll}.

\filbreak
\minorKey{Per-Memory Window Type:} Exo 1 generated window structs for each (primitive type $\times$ dimensionality $\times$ mutability) combination.
For example, for (f32, 2D, read-only), it generates:
{\color{lightttColor}
\begin{verbatim}
struct exo_win_2f32c{
    const float * const data;
    const int_fast32_t strides[2];
};
\end{verbatim}
}
\filbreak
Essentially, Exo 1 assumes all memory can be addressed with a C pointer (I call this the window \myKeyA{dataptr}) and freely-specified strides (I call this the window \myKeyA{layout}).
This assumption breaks down for the GPU, and is even invalid for Exo 1 (e.g. for AVX).

\filbreak
I made an effort to allow memory types to generate their own custom window structs (defining both a \myKeyA{dataptr} and a \myKeyA{layout}).
Memory and window definitions are put in the header (\lighttt{.h}) file iff the memory type is exposed in the header file; otherwise, the definitions go into implementation files (\lighttt{.c} and \lighttt{.cuh}).

\filbreak
\minorKey{Separate windowptr:} Custom window types may specify that the dataptr and layout be created as separate variables, instead of together as a struct.
We need this for CUtensorMap windows: the CUtensorMap needs to be in grid constant memory, while associated layout information needs to be in registers (local variables).
This is a bit fragile: it's a rarely used feature, and the C backend has to be aware of this possibility everywhere.

\filbreak
\minorKey{SpecialWindow:} Memory types are now generalized to MemWin types (Memory or SpecialWindow).
A SpecialWindow type is like a Memory type in that it's used to constrain the parameters for an \lighttt{instr}.
However, variables constructed as a SpecialWindow must be created from existing allocations as a window statement (TBD replace WindowStmt?), instead of allocated.

\filbreak
This exists for now to support CUtensorMap, as in the following example:
{\color{lightttColor}
\begin{verbatim}
@proc
def xgemm_Sm90_wgmma(M: size, N: size, K: size,
                     A: f32[M,K] @ CudaGmemLinear,
                     B: f32[N,K] @ CudaGmemLinear,
                     C: f32[N,M] @ CudaGmemLinear):
    A_tensorMap = A[:,:] @ Sm90_tensorMap(128, smem_m, smem_k)
    B_tensorMap = B[:,:] @ Sm90_tensorMap(128, smem_n, smem_k)
    # ...
\end{verbatim}
}

\filbreak
Note the \lighttt{@ Sm90\_tensorMap(...)} annotation distinguishes this case from an ordinary WindowStmt, which does not override the MemWin type.
Similar to memory types, there are per-actor-kind creation, read, and write permissions.
The CUtensorMap can only be created in CPU code, and read and written through TMA code.

\filbreak
\minorKey{MemWin Template:} You can now define parameterized MemWin types with functions that return an inner MemWin class.
For example, the above \lighttt{Sm90\_tensorMap} example is defined as
{\color{lightttColor}
\begin{verbatim}
@memwin_template
def Sm90_tensorMap(swizzle, *smem_box):
    # ...
    class CUtensorMap(SpecialWindow):
        # define MemWin class as usual
    return CUtensorMap
\end{verbatim}
}
Where in the earlier example, \lighttt{swizzle = 128} and \lighttt{smem\_box = (smem\_m, smem\_k)} or \lighttt{(smem\_n, smem\_k)}.
This is cached: identically parameterized MemWins will be identical Python types.

\filbreak
\minorKey{Instr Classes:} Since CUDA instructions have so many parameters (e.g. per-parameter actor signatures), and we may have to generate a large number of similar instructions for different permutations of matrix sizes, I've implemented an alternative class-based interface for \lighttt{@instr}.
The class defines two meaningful functions:
\begin{enumerate}
  \item \lighttt{behavior}: This is parsed as Exo code, defining the body of the \lighttt{instr} proc, in the same manner as the current function-based \lighttt{@instr} interface.
  \filbreak
  \item \lighttt{instance}: This is an executed Python member function, taking \lighttt{self} and a subset of parameters used in \lighttt{behavior}.
\end{enumerate}

\filbreak
All parameters for \lighttt{instance} must be constants in Exo object code.
The \lighttt{instance} function defines
\begin{enumerate}
  \item The Python format string for the C instruction, and C globals (pre-existing \lighttt{@instr} functionality)
  \filbreak
  \item Required include files and utility code for \lighttt{.cuh} file
  \filbreak
  \item Instruction required collective unit \& actor kind
  \filbreak
  \item Per-parameter MemWin type and actor signature
\end{enumerate}

\filbreak
See \lighttt{LoopIR.InstrInfo}.

\filbreak
\minorKey{With Context:} Overloaded \lighttt{with <ctx>:} statement.
For metaprogramming, we reserve \lighttt{with python:} and \lighttt{with exo:};
anything else is evaluated as a with-context object, which could be \lighttt{CudaDeviceFunction}, \lighttt{CudaWarps}, or \lighttt{CudaAsync}.

\filbreak
There is currently no LoopIR node for this.
Maybe we should add one, but that would entail auditing a huge amount of WIP code (e.g. the new analysis).

\filbreak
What I do for now is wrap the with-context object inside a \lighttt{LoopIR.Const}, which is wrapped as the \lighttt{cond} of \lighttt{LoopIR.If}.
I detect this arrangement with the special helper \lighttt{is\_if\_holding\_with}.
This hack works because all with-context objects are truthy, and most of the code treats this as \lighttt{if True}.
I just have special detection where the with-context is relevant, and also in \lighttt{simplify}, so the ``redundant'' \lighttt{if True} isn't simplified away.

\filbreak
\myTitle{Roadmap \& Time Management}

\end{document}
