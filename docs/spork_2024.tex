\documentclass[12pt]{article}

\usepackage[paperheight=148mm, paperwidth=263mm, margin=10mm]{geometry}
\usepackage{enumitem}
\usepackage{amsmath}
\usepackage{amssymb}
\usepackage{amsfonts}
\usepackage{placeins}
\usepackage{graphicx}
\usepackage{listings}
\usepackage{caption}
\usepackage{colortbl}
\usepackage[parfill=0pt]{parskip}
\usepackage[mathscr]{euscript}
\usepackage[usenames,dvipsnames,svgnames,table,hyperref]{xcolor}
\usepackage[hidelinks]{hyperref}
\usepackage{fontspec}
\usepackage{mdframed}
\usepackage{tikz}
\usetikzlibrary{shapes.geometric, arrows, positioning, calc}

\setsansfont{FreeSans}
\setmonofont{Ubuntu Mono}
\renewcommand{\familydefault}{\sfdefault}

\hyphenation{WebGL}

\definecolor{webColor}{RGB}{0, 108, 174}
\newcommand{\web}[1]{{\color{webColor} \small \url{#1}}}
\newcommand{\webText}[2]{{\color{webColor} \href{#2}{#1}}}
\newcommand{\email}[2]{{\small \color{webColor} \textsf{\href{mailto:#1@#2}{#1[at]#2}}}}
\definecolor{titleColor}{RGB}{179, 0, 149}
\definecolor{extraTitleColor}{RGB}{0, 149, 224}
\newcommand{\myTitle}[1]{{\large \color{titleColor} \hspace{-4mm} \textbf{\textsf{#1}}}}
\newcommand{\myTitleExtra}[1]{{\large \color{extraTitleColor} \hspace{-4mm} \textbf{\textsf{#1}}}}
\newcommand{\myBiggerTitle}[1]{{\huge \color{titleColor} \hspace{-4mm} \textbf{\textsf{#1}}}}
\newcommand{\myBiggerTitleExtra}[1]{{\huge \color{extraTitleColor} \hspace{-4mm} \textbf{\textsf{#1}}}}
\definecolor{subColor}{RGB}{179, 0, 149}
\newcommand{\mySub}[1]{\textsf{\color{subColor}$\blacktriangleright$ #1}}
\definecolor{keyColor}{RGB}{170, 149, 0}  % legacy, = keyColorA
\definecolor{keyColorA}{RGB}{170, 149, 0}
\definecolor{keyColorB}{RGB}{170, 210, 0}
\newcommand{\myKey}[1]{\textbf{\color{keyColor}#1}}
\newcommand{\myKeyA}[1]{\textbf{\color{keyColor}#1}}
\newcommand{\myKeyB}[1]{\textbf{\color{keyColorB}#1}}

% Red and blue boxes have bright text color.
% Yellow, green, and violet have intentionally muted text colors.
% They used to be black, but it looks slightly better with a tiny bit of color.
\definecolor{redBoxFg}{RGB}{224, 0, 0}
\definecolor{redBoxBg}{RGB}{255, 211, 225}
\newcommand{\redBox}[1]{{\color{redBoxFg}\colorbox{redBoxBg}{#1}}}
\definecolor{yellowBoxFg}{RGB}{89, 89, 0}
\definecolor{yellowBoxBg}{RGB}{255, 232, 0}
\newcommand{\yellowBox}[1]{{\color{yellowBoxFg}\colorbox{yellowBoxBg}{#1}}}
\definecolor{greenBoxFg}{RGB}{0, 89, 63}
\definecolor{greenBoxBg}{RGB}{179, 232, 160}
\newcommand{\greenBox}[1]{{\color{greenBoxFg}\colorbox{greenBoxBg}{#1}}}
\definecolor{blueBoxFg}{RGB}{0, 97, 232}
\definecolor{blueBoxBg}{RGB}{224, 232, 255}
\newcommand{\blueBox}[1]{{\color{blueBoxFg}\colorbox{blueBoxBg}{#1}}}
\definecolor{violetBoxFg}{RGB}{108, 63, 124}
\definecolor{violetBoxBg}{RGB}{218, 204, 255}
\newcommand{\violetBox}[1]{{\color{violetBoxFg}\colorbox{violetBoxBg}{#1}}}

\mdfdefinestyle{MyFrame}{%
    linecolor=black,
    outerlinewidth=0pt,
    linewidth=0pt,
    innertopmargin=2.7pt,
    innerbottommargin=0pt,
    innerrightmargin=0pt,
    innerleftmargin=0pt,
        leftmargin = 0pt,
        rightmargin = 0pt}


\definecolor{lightttColor}{RGB}{69, 69, 80}
\newcommand{\lighttt}[1]{{\color{lightttColor}\texttt{#1}}}
\newcommand{\blacktt}[1]{{\color{black}\texttt{#1}}}
\definecolor{grayttColor}{RGB}{144, 144, 160}
\newcommand{\graytt}[1]{{\color{grayttColor}\texttt{#1}}}
\definecolor{emphttColor}{RGB}{170, 149, 0}
\newcommand{\emphtt}[1]{{\color{emphttColor}\texttt{#1}}}

\renewcommand*\labelenumi{(\theenumi)}
\renewcommand*{\theenumii}{\roman{enumii}}
\renewcommand*\labelenumii{\theenumii.}

\newcommand{\fixminipage}{\raggedright \setlength{\parskip}{0.3\baselineskip}}
\newcommand{\codeminipage}{\raggedright \setlength{\parskip}{0\baselineskip}}
\sloppy
%\pagenumbering{gobble}


\begin{document}

{\hfill \LARGE \textbf{\textsf{Project Spork: EXO GPU}}}

\includegraphics[width=\linewidth]{usda_spork.jpg}

\newpage

\myTitle{Project Goals}

\begin{minipage}[t]{0.5\textwidth}\fixminipage
\myKey{Exo Today:} Python-embedded USL with ``imperative'' rewrite-based scheduling

\texttt{@proc} decorator captures annotated Python AST; procs translated (almost) 1:1 to CPU C code

Rewrite with proc$\to$proc scheduling functions
\begin{itemize}
  \item Example: loop division (into inner/outer loops)
  \item \myKey{Goal:} rewrite rules for CUDA parallelism
\end{itemize}
Custom accelerator instructions defined by the user (or stdlib)
\begin{itemize}
  \item Behavior declared as Python AST (\texttt{@instr})
  \item \texttt{replace} rewrite rule substitutes matching code blocks with custom instr.
  \item \myKey{Goal:} expand feature to express special CUDA instructions (async copies, wgmma)
\end{itemize}
Safety Checking
\begin{itemize}
  \item Exo checks that \textit{each} rewrite preserves program behavior.
  \item \myKey{Goal:} check parallelized program matches original intended sequential program's behavior (synchronization check)
\end{itemize}
\end{minipage} %
\hfill
\begin{minipage}[t]{0.5\textwidth}\fixminipage
\mySub{Unscheduled Example Program: Naïve Code}
\vspace{3mm}
{
\tiny
\begin{verbatim}
@proc
def original_gemm(M: size, N: size, K: size, A: f32[M, K] @ DRAM,
                  B: f32[K, N] @ DRAM, C: f32[M, N] @ DRAM):
    assert M % 16 == 0
    assert N % 16 == 0
    assert K % 8 == 0
    for m in seq(0, M):
        for n in seq(0, N):
            accum: f32 @ DRAM
            accum = 0
            for k in seq(0, K):
                accum += A[m, k] * B[k, n]
            C[m, n] = accum
\end{verbatim}
}
\vspace{1cm}
\mySub{Compiled C Code}
\vspace{3mm}
{
\tiny
\begin{verbatim}
void original_gemm(void *ctxt, int_fast32_t M, int_fast32_t N, int_fast32_t K,
                   const float* A, const float* B, float* C ) {
EXO_ASSUME(M % 16 == 0);
EXO_ASSUME(N % 16 == 0);
EXO_ASSUME(K % 8 == 0);
for (int_fast32_t m = 0; m < M; m++) {
  for (int_fast32_t n = 0; n < N; n++) {
    float accum;
    accum = 0.0f;
    for (int_fast32_t k = 0; k < K; k++) {
      accum += A[m * K + k] * B[k * N + n];
    }
    C[m * N + n] = accum;
  }
}
\end{verbatim}
}
\end{minipage} %
\vfill
% Intro to Exo
\newpage
\myTitle{Exo CPU GEMM Example: Original Code}

\begin{minipage}[t]{0.5\textwidth}\codeminipage
Imagine we want to reschedule this matrix multiply function to target a hypothetical CPU matrix tile accelerator that operates on $16 \times 16 \times 8$ tiles ($M \times N \times K$):
\vspace{6mm}
\tiny
\begin{verbatim}
@proc
def original_gemm(M: size, N: size, K: size, A: f32[M,K], B: f32[K,N], C: f32[M,N]):
    # Avoid requiring predication
    assert M % m_tile == 0
    assert N % n_tile == 0
    assert K % k_tile == 0

    for m in seq(0, M):
        for n in seq(0, N):
            accum : f32
            accum = 0
            for k in seq(0, K):
                accum += A[m,k] * B[k,n]
            C[m,n] = accum
\end{verbatim}
\end{minipage}
\begin{minipage}[t]{0.5\textwidth}\fixminipage
\vspace{3cm}
We specify the custom instruction's behavior and start writing a \texttt{proc}$\to$\texttt{proc} scheduling function \texttt{schedule\_gemm}.
\vspace{6mm}
\tiny
\begin{verbatim}
@instr("example_mma_tile({C_tile_data}, {A_tile_data}, {B_tile_data});")
def example_mma_tile(C_tile: [f32][m_tile,n_tile] @ ExampleAcceleratorTile,
                     A_tile: [f32][m_tile,k_tile] @ ExampleAcceleratorTile,
                     B_tile: [f32][k_tile,n_tile] @ ExampleAcceleratorTile):
    for m in seq(0, m_tile):
        for n in seq(0, n_tile):
            for k in seq(0, k_tile):
                C_tile[m,n] += A_tile[m,k] * B_tile[k,n]


def schedule_gemm(p, new_name):
    p = rename(p, new_name)
    return p  # TODO

exo_cpu_gemm = schedule_gemm(original_gemm, "exo_gpu_gemm")
\end{verbatim}
\end{minipage}
\newpage
\myTitle{Exo CPU GEMM Example: Divide and Reorder Loops}

\begin{minipage}[t]{0.5\textwidth}\codeminipage
\tiny
\begin{verbatim}
def exo_cpu_gemm(M: size, N: size, K: size, A: f32[M, K] @ DRAM,
                 B: f32[K, N] @ DRAM, C: f32[M, N] @ DRAM):
    assert M % 16 == 0
    assert N % 16 == 0
    assert K % 8 == 0
\end{verbatim}
\begin{mdframed}[style=MyFrame, backgroundcolor=yellowBoxBg]
\color{yellowBoxFg}
\begin{verbatim}
    for mo in seq(0, M / 16):
        for mi in seq(0, 16):
\end{verbatim}
\end{mdframed}
\begin{mdframed}[style=MyFrame, backgroundcolor=greenBoxBg]
\color{greenBoxFg}
\begin{verbatim}
            for no in seq(0, N / 16):
                for ni in seq(0, 16):
\end{verbatim}
\end{mdframed}
\begin{verbatim}
                    accum: f32 @ DRAM
                    accum = 0
\end{verbatim}
\begin{mdframed}[style=MyFrame, backgroundcolor=violetBoxBg]
\color{violetBoxFg}
\begin{verbatim}
                    for ko in seq(0, K / 8):
                        for ki in seq(0, 8):
\end{verbatim}
\end{mdframed}
\begin{verbatim}
                            accum += A[16 * mo + mi, 8 * ko +
                                       ki] * B[8 * ko + ki, 16 * no + ni]
                    C[16 * mo + mi, 16 * no + ni] = accum
\end{verbatim}
\vspace{5mm}
\begin{verbatim}
def exo_cpu_gemm(M: size, N: size, K: size, A: f32[M, K] @ DRAM,
                 B: f32[K, N] @ DRAM, C: f32[M, N] @ DRAM):
    assert M % 16 == 0
    assert N % 16 == 0
    assert K % 8 == 0
    for mo in seq(0, M / 16):
\end{verbatim}
\begin{mdframed}[style=MyFrame, backgroundcolor=blueBoxBg]
\color{blueBoxFg}
\begin{verbatim}
        for no in seq(0, N / 16):
            for mi in seq(0, 16):
\end{verbatim}
\end{mdframed}
\begin{verbatim}
                for ni in seq(0, 16):
                    accum: f32 @ DRAM
                    accum = 0
                    for ko in seq(0, K / 8):
                        for ki in seq(0, 8):
                            accum += A[16 * mo + mi, 8 * ko +
                                       ki] * B[8 * ko + ki, 16 * no + ni]
                    C[16 * mo + mi, 16 * no + ni] = accum
\end{verbatim}
\end{minipage}
\hfill
\begin{minipage}[t]{0.5\textwidth}\codeminipage
The rescheduling divides the $M$, $N$, and $K$ loops into outer and inner loops, with the inner loops matching the custom instruction's tile dimensions.
We then move the \texttt{no} (n-outer) loop outwards.
Eventually the \texttt{ko} (k-outer) loop should move too, but we can't do that yet.
\vspace{1cm}
\tiny
\begin{verbatim}
def schedule_gemm(p, new_name):
\end{verbatim}
\begin{mdframed}[style=MyFrame, backgroundcolor=yellowBoxBg]
\color{yellowBoxFg}
\begin{verbatim}
    p = divide_loop(p, "m", m_tile, ("mo", "mi"), perfect = True)
\end{verbatim}
\end{mdframed}
\begin{mdframed}[style=MyFrame, backgroundcolor=greenBoxBg]
\color{greenBoxFg}
\begin{verbatim}
    p = divide_loop(p, "n", n_tile, ("no", "ni"), perfect = True)
\end{verbatim}
\end{mdframed}
\begin{mdframed}[style=MyFrame, backgroundcolor=violetBoxBg]
\color{violetBoxFg}
\begin{verbatim}
    p = divide_loop(p, "k", k_tile, ("ko", "ki"), perfect = True)
\end{verbatim}
\end{mdframed}
\begin{mdframed}[style=MyFrame, backgroundcolor=blueBoxBg]
\color{blueBoxFg}
\begin{verbatim}
    p = reorder_loops(p, "mi no")
\end{verbatim}
\end{mdframed}
\begin{verbatim}
    # ...
\end{verbatim}
\end{minipage}
\newpage
\myTitle{Exo CPU GEMM Example: Expand and Lift Accumulator}

\begin{minipage}[t]{0.6\textwidth}\codeminipage
\tiny
\begin{verbatim}
def exo_cpu_gemm(M: size, N: size, K: size, A: f32[M, K] @ DRAM,
                 B: f32[K, N] @ DRAM, C: f32[M, N] @ DRAM):
    assert M % 16 == 0
    assert N % 16 == 0
    assert K % 8 == 0
    for mo in seq(0, M / 16):
        for no in seq(0, N / 16):
            for mi in seq(0, 16):
                for ni in seq(0, 16):
\end{verbatim}
\begin{mdframed}[style=MyFrame, backgroundcolor=yellowBoxBg]
\color{yellowBoxFg}
\begin{verbatim}
                    accum: f32[16, 16] @ DRAM   # <<< c_accum_alloc
                    accum[mi, ni] = 0
\end{verbatim}
\end{mdframed}
\begin{verbatim}
                    for ko in seq(0, K / 8):
                        for ki in seq(0, 8):
\end{verbatim}
\begin{mdframed}[style=MyFrame, backgroundcolor=yellowBoxBg]
\color{yellowBoxFg}
\begin{verbatim}
                            accum[mi, ni] += \
\end{verbatim}
\end{mdframed}
\begin{verbatim}
                                A[16 * mo + mi, 8 * ko + ki] * B[8 * ko + ki, 16 * no + ni]
                    C[16 * mo + mi, 16 * no + ni] = \
\end{verbatim}
\begin{mdframed}[style=MyFrame, backgroundcolor=yellowBoxBg]
\color{yellowBoxFg}
\begin{verbatim}
                        accum[mi, ni]
\end{verbatim}
\end{mdframed}
\vspace{5mm}
\begin{verbatim}
def exo_cpu_gemm(M: size, N: size, K: size, A: f32[M, K] @ DRAM,
                 B: f32[K, N] @ DRAM, C: f32[M, N] @ DRAM):
    assert M % 16 == 0
    assert N % 16 == 0
    assert K % 8 == 0
    for mo in seq(0, M / 16):
        for no in seq(0, N / 16):
\end{verbatim}
\begin{mdframed}[style=MyFrame, backgroundcolor=blueBoxBg]
\color{blueBoxFg}
\begin{verbatim}
            accum: f32[16, 16] @ DRAM  # <<< c_accum_alloc
            for mi in seq(0, 16):
                for ni in seq(0, 16):
\end{verbatim}
\end{mdframed}
\begin{verbatim}
                    accum[mi, ni] = 0
\end{verbatim}

\begin{mdframed}[style=MyFrame, backgroundcolor=violetBoxBg]
\color{violetBoxFg}
\begin{verbatim}
                    for ko in seq(0, K / 8):  # Still in the wrong place
\end{verbatim}
\end{mdframed}
\begin{verbatim}
                        for ki in seq(0, 8):
                            accum[mi, ni] += \
                                A[16 * mo + mi, 8 * ko + ki] * B[8 * ko + ki, 16 * no + ni]
                    C[16 * mo + mi, 16 * no + ni] = \
                        accum[mi, ni]
\end{verbatim}
\end{minipage}
\hfill
\begin{minipage}[t]{0.4\textwidth}\codeminipage
We now need to expand \texttt{accum} into a $16 \times 16$ tile, indexed by \texttt{mi,ni}.
This prepares the accumulator to be used as the destination operand of the hardware accelerator.
\vspace{6mm}
{
\tiny
\begin{verbatim}
def schedule_gemm(p, new_name):
    # ...
    c_accum_alloc = p.find("accum : f32")  # Cursor
\end{verbatim}
\begin{mdframed}[style=MyFrame, backgroundcolor=yellowBoxBg]
\color{yellowBoxFg}
\begin{verbatim}
    p = expand_dim(p, c_accum_alloc, n_tile, 'ni')
    p = expand_dim(p, c_accum_alloc, m_tile, 'mi')
\end{verbatim}
\end{mdframed}
\begin{mdframed}[style=MyFrame, backgroundcolor=blueBoxBg]
\color{blueBoxFg}
\begin{verbatim}
    p = lift_alloc(p, c_accum_alloc, n_lifts = 2)
\end{verbatim}
\end{mdframed}
\begin{verbatim}
    # ...
\end{verbatim}
}
\vspace{6mm}
We further hoist \texttt{accum}'s allocation out of the inner loops.
This will be needed later when we fix the \violetBox{\texttt{ko}} loop being in the wrong place.
\end{minipage}

\newpage
\myTitle{Exo CPU GEMM Example: Loop Fission}

\begin{minipage}[t]{0.5\textwidth}\codeminipage
\tiny
\begin{verbatim}
def exo_cpu_gemm(M: size, N: size, K: size, A: f32[M, K] @ DRAM,
                 B: f32[K, N] @ DRAM, C: f32[M, N] @ DRAM):
    assert M % 16 == 0
    assert N % 16 == 0
    assert K % 8 == 0
    for mo in seq(0, M / 16):
        for no in seq(0, N / 16):
            accum: f32[16, 16] @ DRAM
\end{verbatim}
\begin{mdframed}[style=MyFrame, backgroundcolor=yellowBoxBg]
\color{yellowBoxFg}
\begin{verbatim}
            for mi in seq(0, 16):
                for ni in seq(0, 16):
                    accum[mi, ni] = 0
                    # (gap cursor) c_accum_zero.after()
\end{verbatim}
\end{mdframed}
\begin{mdframed}[style=MyFrame, backgroundcolor=violetBoxBg]
\color{violetBoxFg}
\begin{verbatim}
            for ko in seq(0, K / 8):
\end{verbatim}
\end{mdframed}
\begin{mdframed}[style=MyFrame, backgroundcolor=blueBoxBg]
\color{blueBoxFg}
\begin{verbatim}
                for mi in seq(0, 16):
                    for ni in seq(0, 16):
                        for ki in seq(0, 8):
                            accum[mi,
                                  ni] += A[16 * mo + mi, 8 * ko +
                                           ki] * B[8 * ko + ki, 16 * no + ni]
\end{verbatim}
\end{mdframed}
\begin{mdframed}[style=MyFrame, backgroundcolor=greenBoxBg]
\color{greenBoxFg}
\begin{verbatim}
            for mi in seq(0, 16):
                for ni in seq(0, 16):
                    # (gap cursor) c_accum_export.before()
                    C[16 * mo + mi, 16 * no + ni] = accum[mi, ni]
\end{verbatim}
\end{mdframed}
\end{minipage}
\hfill
\begin{minipage}[t]{0.5\textwidth}\codeminipage
We now fission the inner loops (\texttt{mi,ni}) before and after the \texttt{+=} operation.
This factors out the \blueBox{loop nest (blue)} for later targetting with the custom MMA instruction.

\vspace{6mm}
{\tiny
\begin{verbatim}
def schedule_gemm(p, new_name):
    # ...
    c_accum_zero = p.find("accum = 0")
    c_accum_export = p.find("_ = accum")
\end{verbatim}
\begin{mdframed}[style=MyFrame, backgroundcolor=yellowBoxBg]
\color{yellowBoxFg}
\begin{verbatim}
    p = fission(p, c_accum_zero.after(), n_lifts = 2)
\end{verbatim}
\end{mdframed}
\begin{mdframed}[style=MyFrame, backgroundcolor=greenBoxBg]
\color{greenBoxFg}
\begin{verbatim}
    p = fission(p, c_accum_export.before(), n_lifts = 2)
\end{verbatim}
\end{mdframed}
\begin{mdframed}[style=MyFrame, backgroundcolor=violetBoxBg]
\color{violetBoxFg}
\begin{verbatim}
    p = reorder_loops(p, "ni ko")
    p = reorder_loops(p, "mi ko")
\end{verbatim}
\end{mdframed}
\begin{verbatim}
    # ...
\end{verbatim}
}
\vspace{6mm}
We now also move out the \violetBox{\texttt{ko}} loop to its proper place outside the \texttt{mi},\texttt{ni} loops.
\end{minipage}
\newpage
\myTitle{Exo CPU GEMM Example: Stage Memory}

\begin{minipage}[t]{0.4\textwidth}\codeminipage
\tiny
\begin{verbatim}
def exo_cpu_gemm(M: size, N: size, K: size, A: f32[M, K] @ DRAM,
                 B: f32[K, N] @ DRAM, C: f32[M, N] @ DRAM):
    assert M % 16 == 0
    assert N % 16 == 0
    assert K % 8 == 0
    for mo in seq(0, M / 16):
        for no in seq(0, N / 16):
            accum: f32[16, 16] @ DRAM
            for mi in seq(0, 16):
                for ni in seq(0, 16):
                    accum[mi, ni] = 0
            for ko in seq(0, K / 8):
\end{verbatim}
\begin{mdframed}[style=MyFrame, backgroundcolor=yellowBoxBg]
\color{yellowBoxFg}
\begin{verbatim}
                A_tile: f32[16, 8] @ DRAM
                for i0 in seq(0, 16):
                    for i1 in seq(0, 8):
                        A_tile[i0, i1] = A[i0 + 16 * mo, i1 + 8 * ko]
\end{verbatim}
\end{mdframed}
\begin{mdframed}[style=MyFrame, backgroundcolor=greenBoxBg]
\color{greenBoxFg}
\begin{verbatim}
                B_tile: f32[8, 16] @ DRAM
                for i0 in seq(0, 8):
                    for i1 in seq(0, 16):
                        B_tile[i0, i1] = B[i0 + 8 * ko, i1 + 16 * no]
\end{verbatim}
\end{mdframed}
\begin{mdframed}[style=MyFrame, backgroundcolor=blueBoxBg]
\color{blueBoxFg}
\begin{verbatim}
                for mi in seq(0, 16):         # <<< c_tile_reduce
\end{verbatim}
\end{mdframed}
\begin{verbatim}
                    for ni in seq(0, 16):     # <<< parent().parent()
                        for ki in seq(0, 8):  # <<< parent()
\end{verbatim}
\begin{mdframed}[style=MyFrame, backgroundcolor=yellowBoxBg]
\color{yellowBoxFg}
\begin{verbatim}
                            accum[mi, ni] += A_tile[mi, ki] \
\end{verbatim}
\end{mdframed}
\begin{mdframed}[style=MyFrame, backgroundcolor=greenBoxBg]
\color{greenBoxFg}
\begin{verbatim}
                                             * B_tile[ki, ni]
\end{verbatim}
\end{mdframed}
\begin{verbatim}
            for mi in seq(0, 16):
                for ni in seq(0, 16):
                    C[mi + 16 * mo, ni + 16 * no] = accum[mi, ni]
\end{verbatim}
\end{minipage}
\hfill
\begin{minipage}[t]{0.6\textwidth}\codeminipage
We now construct a \blueBox{``cursor''} to the root of the future location for the matrix tile instruction,
and ask Exo to stage $16 \times 16$ tiles of the input matrices (\texttt{A\_tile}, \texttt{B\_tile}).
This
\begin{itemize}
  \item Causes \texttt{A\_tile} and \texttt{B\_tile} to be prepared outside the scope pointed-to by the cursor.
  \item Causes \texttt{A\_tile} and \texttt{B\_tile} to be substituted (with updated indexing) for uses of \texttt{A} and \texttt{B} inside the scope of the cursor.
\end{itemize}
\vspace{6mm}
{\tiny
\begin{verbatim}
def schedule_gemm(p, new_name):
    # ...
\end{verbatim}
\begin{mdframed}[style=MyFrame, backgroundcolor=blueBoxBg]
\color{blueBoxFg}
\begin{verbatim}
    c_tile_reduce = p.find("accum += _").parent().parent().parent()
\end{verbatim}
\end{mdframed}
\begin{mdframed}[style=MyFrame, backgroundcolor=yellowBoxBg]
\color{yellowBoxFg}
\begin{verbatim}
    p = stage_mem(p, c_tile_reduce,
                  f"A[mo*{m_tile}:(mo+1)*{m_tile}, ko*{k_tile}:(ko+1)*{k_tile}]", "A_tile")
\end{verbatim}
\end{mdframed}
\begin{mdframed}[style=MyFrame, backgroundcolor=greenBoxBg]
\color{greenBoxFg}
\begin{verbatim}
    p = stage_mem(p, c_tile_reduce,
                  f"B[ko*{k_tile}:(ko+1)*{k_tile}, no*{n_tile}:(no+1)*{n_tile}]", "B_tile")
\end{verbatim}
\end{mdframed}
\begin{verbatim}
    p = simplify(p)
    # ...
\end{verbatim}
}
\end{minipage}
\newpage
\myTitle{Exo CPU GEMM Example: Custom Accelerator}

\begin{minipage}[t]{0.5\textwidth}\codeminipage
\tiny
\begin{verbatim}
def exo_cpu_gemm(M: size, N: size, K: size, A: f32[M, K] @ DRAM,
                 B: f32[K, N] @ DRAM, C: f32[M, N] @ DRAM):
    assert M % 16 == 0
    assert N % 16 == 0
    assert K % 8 == 0
    for mo in seq(0, M / 16):
        for no in seq(0, N / 16):
\end{verbatim}
\begin{mdframed}[style=MyFrame, backgroundcolor=violetBoxBg]
\color{violetBoxFg}
\begin{verbatim}
            accum: f32[16, 16] @ ExampleAcceleratorTile
\end{verbatim}
\end{mdframed}
\begin{verbatim}
            for mi in seq(0, 16):
                for ni in seq(0, 16):
                    accum[mi, ni] = 0
            for ko in seq(0, K / 8):
\end{verbatim}
\begin{mdframed}[style=MyFrame, backgroundcolor=yellowBoxBg]
\color{yellowBoxFg}
\begin{verbatim}
                A_tile: f32[16, 8] @ ExampleAcceleratorTile
\end{verbatim}
\end{mdframed}
\begin{verbatim}
                for i0 in seq(0, 16):
                    for i1 in seq(0, 8):
                        A_tile[i0, i1] = A[i0 + 16 * mo, i1 + 8 * ko]
\end{verbatim}
\begin{mdframed}[style=MyFrame, backgroundcolor=greenBoxBg]
\color{greenBoxFg}
\begin{verbatim}
                B_tile: f32[8, 16] @ ExampleAcceleratorTile
\end{verbatim}
\end{mdframed}
\begin{verbatim}
                for i0 in seq(0, 8):
                    for i1 in seq(0, 16):
                        B_tile[i0, i1] = B[i0 + 8 * ko, i1 + 16 * no]
\end{verbatim}
\begin{mdframed}[style=MyFrame, backgroundcolor=blueBoxBg]
\color{blueBoxFg}
\begin{verbatim}
                example_mma_tile(accum[0:16, 0:16], A_tile[0:16, 0:8],
                                 B_tile[0:8, 0:16])
\end{verbatim}
\end{mdframed}
\begin{verbatim}
            for mi in seq(0, 16):
                for ni in seq(0, 16):
                    C[mi + 16 * mo, ni + 16 * no] = accum[mi, ni]
\end{verbatim}
\end{minipage}
\hfill
\begin{minipage}[t]{0.5\textwidth}\codeminipage
Finally, we update the memory types for the tiles.
This makes it possible to substitute the custom instruction for the \texttt{mi}, \texttt{ni}, \texttt{ki} loop nest.
\vspace{6mm}
{\tiny
\begin{verbatim}
@instr("example_mma_tile({C_tile_data}, {A_tile_data}, {B_tile_data});")
def example_mma_tile(
\end{verbatim}
\begin{mdframed}[style=MyFrame, backgroundcolor=violetBoxBg]
\color{violetBoxFg}
\begin{verbatim}
                     C_tile: [f32][m_tile,n_tile] @ ExampleAcceleratorTile,
\end{verbatim}
\end{mdframed}
\begin{mdframed}[style=MyFrame, backgroundcolor=yellowBoxBg]
\color{yellowBoxFg}
\begin{verbatim}
                     A_tile: [f32][m_tile,k_tile] @ ExampleAcceleratorTile,
\end{verbatim}
\end{mdframed}
\begin{mdframed}[style=MyFrame, backgroundcolor=greenBoxBg]
\color{greenBoxFg}
\begin{verbatim}
                     B_tile: [f32][k_tile,n_tile] @ ExampleAcceleratorTile):
\end{verbatim}
\end{mdframed}
\begin{verbatim}
    for m in seq(0, m_tile):
        for n in seq(0, n_tile):
            for k in seq(0, k_tile):
                C_tile[m,n] += A_tile[m,k] * B_tile[k,n]

def schedule_gemm(p, new_name):
    # ...
    c_accum_alloc = p.find("accum : f32")  # Cursor
    # ...
\end{verbatim}
\begin{mdframed}[style=MyFrame, backgroundcolor=violetBoxBg]
\color{violetBoxFg}
\begin{verbatim}
    p = set_memory(p, c_accum_alloc, ExampleAcceleratorTile)
\end{verbatim}
\end{mdframed}
\begin{mdframed}[style=MyFrame, backgroundcolor=yellowBoxBg]
\color{yellowBoxFg}
\begin{verbatim}
    p = set_memory(p, "A_tile", ExampleAcceleratorTile)
\end{verbatim}
\end{mdframed}
\begin{mdframed}[style=MyFrame, backgroundcolor=greenBoxBg]
\color{greenBoxFg}
\begin{verbatim}
    p = set_memory(p, "B_tile", ExampleAcceleratorTile)
\end{verbatim}
\end{mdframed}
\begin{mdframed}[style=MyFrame, backgroundcolor=blueBoxBg]
\color{blueBoxFg}
\begin{verbatim}
    p = replace(p, c_tile_reduce, example_mma_tile)
\end{verbatim}
\end{mdframed}
\begin{verbatim}
    return p
\end{verbatim}
}
\vspace{6mm}
A real accelerator would likely also need custom load and store instructions for \texttt{ExampleAcceleratorTile}, but we skip this for the example.
\end{minipage}
\newpage
\myTitle{Intro to CUDA}
% Intro to CUDA

% CUDA gemm pseudocode

% Exo parallelism: loop mode, fence

% Exo Example: gemm with CUDA

% Correctness Model: Sychronization Environment

% GPU gemm correctness example

% CUDA Async Instructions

% CUDA Advanced Synchronization

% TMA

% wgmma

% Actor Kind

% Actor Signature

\end{document}
