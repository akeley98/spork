% exocc b_samples.py && python3 code_to_tex.py b_samples.py b_samples && xelatex spork_b.tex </dev/null
\documentclass[11pt]{article}

\usepackage[letterpaper, portrait, margin=2.54cm]{geometry}
\usepackage{enumitem}
\usepackage{amsmath}
\usepackage{amssymb}
\usepackage{amsfonts}
\usepackage{placeins}
\usepackage{graphicx}
\usepackage{listings}
\usepackage{caption}
\usepackage{colortbl}
\usepackage[parfill]{parskip}
\usepackage[mathscr]{euscript}
\usepackage[usenames,dvipsnames,svgnames,table,hyperref]{xcolor}
\usepackage[hidelinks]{hyperref}
\usepackage{fontspec}
\usepackage{mdframed}
\usepackage{tikz}
\usetikzlibrary{shapes.geometric, arrows, positioning, calc}

\setsansfont{FreeSans}
\setmonofont{Ubuntu Mono}

\hyphenation{WebGL}

%\definecolor{webColor}{RGB}{0, 124, 204}
\definecolor{webColor}{RGB}{0, 108, 174}
\newcommand{\web}[1]{{\color{webColor} \small \url{#1}}}
\newcommand{\webText}[2]{{\color{webColor} \href{#2}{#1}}}
\newcommand{\email}[2]{{\small \color{webColor} \textsf{\href{mailto:#1@#2}{#1[at]#2}}}}
\definecolor{titleColor}{RGB}{0, 0, 0}
\ifdefined\RESUME
\newcommand{\myTitle}[1]{{\large \color{titleColor} \hspace{-12mm} \textbf{\textsf{#1}}}}
\else
\newcommand{\myTitle}[1]{{ \vspace{2mm} \large \color{titleColor} \hspace{-12mm} \textbf{\textsf{#1}} \vspace{2mm}}}
\fi
\definecolor{subColor}{RGB}{170, 149, 0}
\newcommand{\mySub}[1]{{\color{subColor}\hspace{-9mm} \textsf{#1}}}
\definecolor{keyColor}{RGB}{118, 185, 0}
\newcommand{\myKey}[1]{{\color{keyColor}\textbf{#1}}}

\definecolor{keyColorA}{RGB}{170, 149, 0}
\definecolor{keyColorB}{RGB}{170, 210, 0}
\newcommand{\myKeyA}[1]{\textbf{\color{keyColorA}#1}}
\newcommand{\myKeyB}[1]{\textbf{\color{keyColorB}#1}}

\definecolor{mainColor}{RGB}{170, 0, 170}
\definecolor{minorColor}{RGB}{0, 149, 232}
\newcommand{\mainKey}[1]{\textbf{\color{mainColor}#1}}
\newcommand{\minorKey}[1]{\textbf{\color{minorColor}#1}}
\newcommand{\mainSub}[1]{{\color{mainColor}\hspace{-9mm} \textsf{#1}}}
\newcommand{\minorSub}[1]{{\color{minorColor}\hspace{-9mm} \textsf{#1}}}

\definecolor{hookColor}{RGB}{204, 0, 255}
\newcommand{\hook}[1]{{\color{hookColor}\textbf{#1}}}
\definecolor{flaggedBoxColor}{RGB}{255, 204, 204}
\newcommand{\flagged}[1]{{\color{black}\colorbox{flaggedBoxColor}{#1}}}

\definecolor{lightttColor}{RGB}{69, 69, 80}
\newcommand{\lighttt}[1]{{\color{lightttColor}\texttt{#1}}}
\newcommand{\graytt}[1]{{\color{gray}\texttt{#1}}}
\newcommand{\blacktt}[1]{{\color{black}\texttt{#1}}}

% Red and blue boxes have bright text color.
% Yellow, green, and violet have intentionally muted text colors.
% They used to be black, but it looks slightly better with a tiny bit of color.
\definecolor{redBoxFg}{RGB}{224, 0, 0}
\definecolor{redBoxBg}{RGB}{255, 211, 225}
\newcommand{\redBox}[1]{{\color{redBoxFg}\colorbox{redBoxBg}{#1}}}
\definecolor{yellowBoxFg}{RGB}{89, 89, 0}
\definecolor{yellowBoxBg}{RGB}{255, 232, 0}
\newcommand{\yellowBox}[1]{{\color{yellowBoxFg}\colorbox{yellowBoxBg}{#1}}}
\definecolor{greenBoxFg}{RGB}{0, 89, 63}
\definecolor{greenBoxBg}{RGB}{179, 232, 160}
\newcommand{\greenBox}[1]{{\color{greenBoxFg}\colorbox{greenBoxBg}{#1}}}
\definecolor{blueBoxFg}{RGB}{0, 97, 232}
\definecolor{blueBoxBg}{RGB}{224, 232, 255}
\newcommand{\blueBox}[1]{{\color{blueBoxFg}\colorbox{blueBoxBg}{#1}}}
\definecolor{violetBoxFg}{RGB}{108, 63, 124}
\definecolor{violetBoxBg}{RGB}{218, 204, 255}
\newcommand{\violetBox}[1]{{\color{violetBoxFg}\colorbox{violetBoxBg}{#1}}}

\mdfdefinestyle{MyFrame}{%
    linecolor=black,
    outerlinewidth=0pt,
    linewidth=0pt,
    innertopmargin=2.7pt,
    innerbottommargin=0pt,
    innerrightmargin=0pt,
    innerleftmargin=0pt,
        leftmargin = 0pt,
        rightmargin = 0pt}


\renewcommand*{\theenumi}{\alph{enumi}}
\renewcommand*\labelenumi{(\theenumi)}
\renewcommand*{\theenumii}{\roman{enumii}}
\renewcommand*\labelenumii{\theenumii.}

\sloppy
\newcommand{\abs}[1]{\left \vert #1 \right\vert}

% Some of these links are now dead...
\newcommand{\mbarrier}{\webText{mbarrier}{https://docs.nvidia.com/cuda/parallel-thread-execution/\#parallel-synchronization-and-communication-instructions-mbarrier}}
\newcommand{\cpAsync}{\webText{cp.async}{https://docs.nvidia.com/cuda/parallel-thread-execution/\#data-movement-and-conversion-instructions-cp-async}}
\newcommand{\cpAsyncBulk}{\webText{cp.async.bulk}{https://docs.nvidia.com/cuda/parallel-thread-execution/\#data-movement-and-conversion-instructions-cp-async-bulk}}
\newcommand{\fenceProxyAsync}{\webText{fence.proxy.async}{https://docs.nvidia.com/cuda/parallel-thread-execution/\#asynchronous-warpgroup-level-matrix-async-proxy}}
\newcommand{\wgmma}{\webText{wgmma}{https://docs.nvidia.com/cuda/parallel-thread-execution/\#asynchronous-warpgroup-level-matrix-instructions}}
\newcommand{\hopperBlog}{\webText{NVIDIA Hopper Architecture In-Depth}{https://developer.nvidia.com/blog/nvidia-hopper-architecture-in-depth/}}
\newcommand{\expectTxOperation}{\webText{expect-tx operation}{https://docs.nvidia.com/cuda/parallel-thread-execution/\#parallel-synchronization-and-communication-instructions-mbarrier-expect-tx-operation}}
\newcommand{\completeTxOperation}{\webText{complete-tx operation}{https://docs.nvidia.com/cuda/parallel-thread-execution/\#parallel-synchronization-and-communication-instructions-mbarrier-complete-tx-operation}}
\newcommand{\wgmmaFence}{\webText{wgmma.fence}{https://docs.nvidia.com/cuda/parallel-thread-execution/\#asynchronous-warpgroup-level-matrix-instructions-wgmma-fence}}


\begin{document}

% Kernel launch \& loops:
% CudaDeviceFunction defines clusterDim & blockDim
% Loops: seq, cuda_task, cuda_threads (loop mode)
% nest of cuda_tasks loops in CudaDeviceFunction. Inner-most body is device task; one instance of the body is assigned to one cluster.
% Within the device task, each instance of a stmt is executed by a set of threads within the cluster (thread collective)
% cuda_threads(0, c_{hi}, unit=...) loop subdivides its executing thread collective into c_{hi}-many disjoint thread collectives, guided by the unit parameter.
% Uniform execution encoded in language.
%
% Collective Types
% cuda_threads loop takes a collective unit from which a collective type \delta is unpacked (section link).
% Collective type encodes number and arrangement of threads (e.g. warp, CTA, CTA pair).
% In typical case, the thread collectives assigned to iteration is described by \delta (section link).
% Local thread index, thread collective need not be contiguous range of local thread indices.
% Lexicographical ordering wrt domain.
% Box and so on.
%
% Distributed Memory
% Thread pitch is crucial concept.
% Thread pitch of iterator.
% Distributed memory; shard mapped to thread collectives described by \delta
% Thread pitch tuple, describes residency.
%
% Synchronization
% Read|Write -> Fence|Arrive
% Read|Write -> Await, when trailing bar
% Fence|Await -> Read
% Fence|Await -> Write
% Arrive -> Await


\section{Overview}
\label{sec:Overview}

{\sffamily

\myChapterLink{sec:CudaDeviceFunction}{Cuda Device Function}

\myChapterLink{sec:CollectiveTypes}{Collective Units \& Collective Types}

\myChapterLink{sec:CollectiveTiling}{Collective Tiling}

\myChapterLink{sec:DistributedMemory}{Distributed Memory}

\myChapterLink{sec:Synchronization}{Synchronization}

\myChapterLink{sec:Glossary}{Glossary}

}

This document assumes knowledge of CPU-only Exo.
We describe the concepts of the GPU extension.
All Exo code continues to be CPU code by default (we say that such code is at \myKeyA{CPU-scope}).
To move code to the GPU, wrap it inside a \lighttt{with CudaDeviceFunction} block (Section~\ref{sec:CudaDeviceFunction}).
The body (which is at \myKeyA{CUDA-scope}) must consist of a single statement: a nest of one or more \lighttt{cuda\_tasks} loops.
The body of the inner-most \lighttt{cuda\_tasks} loop is a \myKeyA{device task}; each is assigned to a CUDA cluster for execution.

Within the device task, \lighttt{cuda\_threads} loops may be used to assign work to threads within the cluster.
A cornerstone of Exo-GPU is that we perform static analysis (collective analysis) on which threads are used to execute which statement instances\footnote{A \emph{statement} is a syntactic construct, while a \emph{statement instance} is a single interpretation/``execution'' of a statement. For example, ``\lighttt{for i in seq(0, 10): $s_1$; $s_2$}'' is a loop containing two statements: $s_1$ and $s_2$, and when the loop is executed, 20 statement instances are created (10 for $s_1$ and 10 for $s_2$).}.
We annotate each statement at CUDA-scope with a \myKeyA{collective tiling} (Section~\ref{sec:CollectiveTiling}), partially describing a mapping between the control environment and the set of threads assigned to execute instances of that statement (i.e. which loop iterations execute with which threads in a cluster).
We call this assigned set of threads a \myKeyA{thread collective}.
This description is partial in the sense that the static analysis only identifies threads by their index within a cluster (the \myKeyA{local thread index}) and not which cluster is used.

We define collective units $\tau_u$ (Section~\ref{sec:CollectiveTypes}) that describe a certain grouping of threads without a concrete index (e.g. single thread, warp, warpgroup, CTA (thread block), cluster).
We say a statement is at $\tau_u$-scope when the thread collectives that execute instances of that statement are accurately described by $\tau_u$.
Statements that are not at single-thread-scope are executed cooperatively by multiple threads; this statically encodes uniform execution in the frontend language.
This does \emph{not} imply a convergence\footnote{Threads are uniform when their control flow are identical; convergence implies additional ``lockstep''/synchronization guarantees beyond uniform execution} guarantee; in particular, \lighttt{cuda\_threads} loops do \emph{not} imply fork-join semantics.
Statements outside single-thread-scope must not access data variables except through instructions that require uniform execution (e.g. warp or warpgroup MMA).

We use the collective tiling downstream to
\begin{itemize}
  \item Map shards of a tensor into different threads for storage; this is \myKeyA{distributed memory} (Section~\ref{sec:DistributedMemory}).
  \item Check correct synchronization (Section~\ref{sec:Synchronization}).
\end{itemize}

\section{Cuda Device Function \& Warp Specialization}
\label{sec:CudaDeviceFunction}

Wrap code with a \lighttt{with CudaDeviceFunction(...):} statement to transform it to CUDA.
The body of the \lighttt{CudaDeviceFunction} statement must consist of exactly one statement: a nest of one or more \lighttt{cuda\_tasks} loops.
The body of the inner-most \lighttt{cuda\_tasks} loop is a \myKeyA{device task}; each is assigned to a CUDA cluster for execution.
We implement a persistent-kernel design, so multiple tasks may be co-located on the same cluster.
The shape of the \lighttt{cuda\_tasks} iteration space must be a cuboid, i.e., the loop bounds of one \lighttt{cuda\_tasks} loop must not be dependent on another \lighttt{cuda\_tasks} loop.

The \lighttt{CudaDeviceFunction} object is a Python object, containing attributes
\begin{itemize}
  \item \lighttt{clusterDim} (default 1), number of CTAs per cluster.
  \item \lighttt{blocks\_per\_sm} (default 1), number of CTAs concurrently executing per hardware SM.
  \item \lighttt{blockDim}, number of threads per CTA.
  \item \lighttt{warp\_config}, list of \lighttt{CudaWarpConfig} objects.
\end{itemize}
Exactly one of \lighttt{blockDim} or \lighttt{warp\_config} must be given.
The latter is intended for kernels with warp specialization, where we partition the warps in the CTA into named groups of warps, possibly with a different number of registers each.
Each \lighttt{CudaWarpConfig} defines a \myKeyA{warp variable}, and has attributes
\begin{itemize}
  \item \lighttt{name: str}, the name of the warp variable.
  \item \lighttt{count: int}, number of warps.
  \item \lighttt{setmaxnreg\_dec: Optional[int]}, registers per thread; regs allocated by \lighttt{setmaxnreg.dec}.
  \item \lighttt{setmaxnreg\_inc: Optional[int]}, registers per thread; regs allocated by \lighttt{setmaxnreg.inc}.
\end{itemize}
The \lighttt{blockDim} of the CTA is implicitly 32 times the sum of the number of warps defined.
Within the device task, a \lighttt{with CudaWarps(name=<str>)} statement may be used to restrict the body of the statement to only execute on the subset of warps named.

\subsection{CudaDeviceFunction Sample}

\input{b_samples/CudaDeviceFunction.0.tex}

\subsection{Scheduling Functions}

\input{b_samples/CudaDeviceFunction_scheduling.0.tex}

\section{Collective Units \& Collective Types}
\label{sec:CollectiveTypes}

We use collective types $\delta$ to describe a quantity and arrangement of threads within a cluster, such as ``single thread'', ``warp'', ``CTA'', ``one warp from a pair of CTAs''.
These are unpacked from a collective unit $\tau_u$ defined in the frontend language (Section~\ref{sec:CollectiveUnit}).
A collective type consists of two equal-length tuples: a domain and a box.
The dimension $M$ of the collective type is the length of these tuples.
The \myKeyA{domain} ($\delta.D_0...\delta.D_{M-1}$): $\mathbb{N}_{\ge2}^M$ describes an organization of the threads in a cluster into an $M$-dimensional space.
The \myKeyA{box} ($\delta.B_0...\delta.B_{M-1}$): $\mathbb{N}_\bot^M$ describes the number of threads on each dimension to select.

We first define a linear ordering of threads in a cluster, then extend to multidimensional coordinates.
The local thread index of a thread is
\[
    \lighttt{cluster\_ctarank * blockDim.x + threadIdx.x}
\]
i.e., the threads in a cluster are numbered in (\lighttt{cluster\_ctarank, threadIdx.x})-lexicographical order (Exo-GPU parallelizes on the x dimension only).
For a given domain, we derive the \myKeyA{dimension thread pitch} $\delta.P_i$:
\begin{align*}
    \delta.P_i = \prod_{k=i+1}^{M-1} \delta.D_k
\end{align*}
and we define the mapping $\mathsf{toLocal}: \mathbb{N}^M \to \mathbb{N}^M \to \mathbb{N}$, which converts a domain and coordinates to a local thread index, as
\begin{align*}
    \mathsf{toLocal}((\delta.D_0,...,\delta.D_{M-1}), (c_0,...,c_{M-1})) \mapsto \sum_{k=0}^{M-1} c_k \delta.P_k
\end{align*}
i.e. the coordinates $[0, \delta.D_0-1]_\mathbb{N} \times ... \times [0, \delta.D_{M-1}-1]_\mathbb{N}$ get mapped to local thread indices in lexicographical order.
The product of the domain coordinates $\delta.D_0 \times ... \times \delta.D_{M-1}$ must be equal to the number of threads in the cluster (\lighttt{clusterDim.x * blockDim.x}).

A thread collective is described by a collective type $\delta$ when all threads are in the same cluster, and, with $\mu: \mathcal{P}(\mathbb{N})$ being the set of local thread indices of the threads, there exist sets $C_0 ... C_{M-1}: \mathcal{P}(\mathbb{N})$ such that
\begin{itemize}
  \item $C_i \subseteq [0, \delta.D_i - 1]$
  \item $\delta.B_i \ne \bot \implies \exists x \mid C_i = [x, x + \delta.B_i - 1]_\mathbb{N}$.
  \item $\mu = \{ \mathsf{toLocal}(\delta.D, c) \mid c \in C_0 \times ... \times C_{M-1}\}$
\end{itemize}

\subsection{Collective Unit to Collective Type}
\label{sec:CollectiveUnit}

Collective units are also parameterized by a pair of $M$-tuples (domain and box), with coordinates being integer expressions of \lighttt{blockDim} and \lighttt{clusterDim}, or $\bot$ in the case of the box.
We convert to 
\begin{itemize}
  \item (fail if any coordinate is not a natural number)
\end{itemize}


\subsection{Reshape}

\section{Collective Tiling}
\label{sec:CollectiveTiling}

The \lighttt{cuda\_tasks} loops assign work to different clusters on the system, and the user has no control (yet) over the mapping between device tasks and clusters.
On the other hand, the \lighttt{cuda\_threads} loop, which assigns work to threads within a cluster, provides the user with tight control over this work mapping.

The deduced collective tiling of each CUDA-scope statement describes a mapping between the control environment and the local thread indices of the thread collective assigned to execute a statement instance.

A statement's collective tiling describes an organization of the threads in a cluster into a multidimensional grid of threads, and the effect that each control environment variable has on the set of threads assigned to execute the statement instance.
Let $M$ denote the dimensionality of the collective tiling.
The collective tiling consists of a tuple of \myKeyA{collective dimensions} $\mathcal{D}_0, ..., \mathcal{D}_{M-1}$; each $\mathcal{D}_i$ contains
\begin{itemize}
  \item $D_i$: dimension extent
  \item $\mathcal{O}_i$: dimension operators
\end{itemize}


Commonly, a \lighttt{cuda\_threads} loop will fail to compile because the user requests more threads than is available in the scope, e.g., a \lighttt{cuda\_threads(0, 40, unit=cuda\_thread)} loop at warp-scope, which would require 40 of the 32 available threads.
More infrequently, the loop will fail to compile because the bounds are not of the form (0, $c_\text{hi}$), or because the compiler is unable to deduce which collective dimension to tile on.

\section{Distributed Memory}
\label{sec:DistributedMemory}

\section{Synchronization}
\label{sec:Synchronization}

\section{Glossary}
\label{sec:Glossary}

\subsection{Statement Instance}
\label{def:StatementInstance}

\end{document}
