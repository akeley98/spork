\magicSection{Synchronization Usage}{sec:SyncUsage}

We view each statement instance (def~\ref{sec:gStmtInstance}) as issuing reads or mutates to a set of array elements; the reads and mutates are executed by a thread collective (def~\ref{sec:gThreadCollective}) with a certain \myKeyA{initial qualitative timeline} (Section~\ref{sec:InstrTL}), which varies based on the issued instruction.
The purpose of the qualitative timeline is to model that the read or mutate is only synchronized by a subset of possible synchronization statements.
These synchronization statements (def~\ref{sec:gSyncStmt}) are parameterized with a \myKeyA{first synchronization timeline} $\tau_s^\mathrm{pre}$ and/or a \myKeyA{second synchronization timeline} $\tau_s^\mathrm{post}$, like so:

\hphantom{spacing}
\texttt{Fence($\tau_s^\mathrm{pre}, \tau_s^\mathrm{post}$)}
\hfill
\texttt{Arrive($\tau_s^\mathrm{pre}$) >}\texttt{> $e$}
\hfill
\texttt{Await($e, \tau_s^\mathrm{post}, n$)}
\hphantom{spacing}

where $e$ is a barrier expression (def~\ref{sec:gBarrierExpr}) and $n$ is an integer, which controls \lighttt{Arrive}/\lighttt{Await} pairing (Section~\ref{sec:ArriveAwaitPairing}).

The synchronization timeline (def~\ref{sec:gSyncTL}) contains a \myKeyA{full timeline set} $\tau_s.\mathrm{full}$ and \myKeyA{temporal timeline set} $\tau_s.\mathrm{temp}$ (both sets of \textsf{QualTL}), which filter (by qualitative timeline) the reads and mutates that the synchronization statement interacts with.

For the purposes of reasoning about correct synchronization, each executed read or mutate has additional attributes
\begin{itemize}
  \item extended qualitative timelines (set of \textsf{QualTL}), Section~\ref{sec:InstrTL}
  \item atomic qualitative timelines (set of \textsf{QualTL}), non-empty for atomic accesses, Section~\ref{sec:AtomicInstr}
  \item convergence flag, Section~\ref{sec:InstrConvergentAccess}.
  \item out-of-order flag, Section~\ref{sec:InstrOOO}.
\end{itemize}
If the convergence flag is true, then the thread collective for each read and mutate executed by the statement instance is the thread collective assigned to the statement instance itself.
Otherwise (i.e. the access is non-convergent), we view each read and mutate as being done by one thread only, repeated for each thread in the thread collective (Section~\ref{sec:VisRecordCreation}).
This models the uncertainty in there being no way to ascribe responsibility for that access to a particular thread, rather than that the access is literally repeated in the underlying hardware.

In Exo-GPU, the user's job is to insert sufficient synchronization so that \myKeyA{sequential-parallel equivalence} is upheld (def~\ref{sec:gSeqParEquivalence}), that is, for a given input, the value computed by interpreting the Exo-GPU program with sequential Exo value semantics will be the same as that computed by the parallelized CUDA program.
The sync-check step (whose ``abstract machine semantics'' are specified in Section~\ref{sec:SyncSemantics}) validates this.

Although this is not how the abstract machine formally works, it's useful to think of Exo-GPU synchronization in terms of dependency edges in a graph formed from a sequential trace of the Exo-GPU program.
The nodes are executed reads to array elements, executed mutates to array elements, and executed synchronization statement instances (with read/mutate nodes generated from statement instances as described above).
The following subsections will specify intended usage patterns of Exo-GPU.
These consist of two nodes appearing in the sequential trace, with dependency edges being formed between nodes involved in an intended usage pattern.

It must be possible to find a connection using dependency edges between one read/mutate to an array element and another read/mutate to that same array element, except when one of the following apply:
\begin{itemize}
    \item the accesses are both reads
    \item the array element is stored in sync-exempt memory (def~\ref{sec:gSyncExempt})
    \item the accesses are both from the same statement instance (due to non-convergent access)
\end{itemize}
While describing the intended usage patterns, we will also jump ahead and reference the rules in Section~\ref{sec:SyncSemantics} that are involved in detecting that pattern for sync-check.
This information is not required for understanding this section.

\magicSubsection{Intended Usage Patterns -- Access Before Synchronization}{sec:AccessBeforeSync}

\begin{equation*}
    \texttt{Read|Mutate} \to \texttt{Arrive(}\tau_s^\mathrm{pre}\texttt{)\codecomment{...} | Fence(}\tau_s^\mathrm{pre},\codecomment{...}\texttt{)}
\end{equation*}

A dependency edge forms between a read or mutate followed by an \lighttt{Arrive} or \lighttt{Fence} statement instance when the initial qualitative timeline $q$ of the read or mutate is a member of $\tau_s^\mathrm{pre}.\mathrm{full}$, and there is a thread in common executing the two.
The synchronization timeline $\tau_s^\mathrm{pre}$ (def~\ref{sec:gSyncTL}) and barrier mechanism (def~\ref{sec:gBarrierMechanism}) to use varies depending on the instruction issuing the read or mutate:
\begin{itemize}
  \item For non-async cuda instructions, use \lighttt{cuda\_in\_order} and a \lighttt{Fence} (Section~\ref{sec:FenceUsage}) or mbarrier (Section~\ref{sec:MbarrierUsage}).
  \item For sm\_80 non-bulk \lighttt{cp.async} instructions (Ampere), use \lighttt{Sm80\_cp\_async} and a commit group (Section~\ref{sec:CommitGroupUsage}) or mbarrier (Section~\ref{sec:MbarrierUsage}).
  \item For TMA instructions copying into SMEM, use mbarriers (Section~\ref{sec:MbarrierUsage}).
  \item For wgmma instructions and TMA instructions copying out of SMEM, use commit groups (Section~\ref{sec:CommitGroupUsage}).
\end{itemize}
This pattern is implemented by storing a new \textsf{VisRecord} (Section~\ref{sec:VisRecordCreation}) containing a timeline signature with one of the executing threads, the initial qualitative timeline $q$, and visibility flag $\mathsf{VF_{issue}}$.
The later \lighttt{Arrive} or \lighttt{Fence} will witness (Section~\ref{sec:Witness}) this timeline signature.


\magicSubsection{Intended Usage Patterns -- Access After Synchronization}{sec:AccessAfterSync}

\begin{equation*}
    \texttt{Await(\codecomment{...}, $\tau_s^\mathrm{post}$, \codecomment{...})|Fence(\codecomment{...}, $\tau_s^\mathrm{post}$)} \to \texttt{Read|Mutate}
\end{equation*}

A dependency edge forms between an \lighttt{Await} or \lighttt{Fence} statement instance and a subsequent read or mutate when there is a thread in common executing the two, and any of
\begin{itemize}
  \item The access is a read, or a read-write mutate (Section~\ref{sec:InstrMemoryAccess}), and there is a qualitative timeline $q$ in common between $\tau_s^\mathrm{post}.\mathrm{full}$ and the extended qualitative timelines (Section~\ref{sec:InstrTL}) of the read or mutate.
  \item The access is a write-only mutate (Section~\ref{sec:InstrMemoryAccess}), and there is a qualitative timeline $q$ in common between $\tau_s^\mathrm{post}.\mathrm{temp}$ and the extended qualitative timelines (Section~\ref{sec:InstrTL}) of the mutate.
  This is ``temporal-only'' synchronization, which may elide proxy fences.
  \item The ``access'' is as a result of freeing shared memory (Section~\ref{sec:ChecksOnFree}).
  This is needed to safely alias future allocations onto the freed physical SMEM, whose effects we overapproximate as a non-convergent write-only mutate by each thread in the cluster.
\end{itemize}

The sychronization timeline $\tau_s^\mathrm{post}$ (def~\ref{sec:gSyncTL}) to use varies depending on the instruction issuing the read or mutate:
\begin{itemize}
  \item \lighttt{cuda\_in\_order} if the following reads and mutates are only issued by generic proxy instructions: non-async CUDA instructions and sm\_80 non-bulk \lighttt{cp.async}.
  These instructions include \lighttt{cuda\_in\_order\_ram\_qual} (def~\ref{sec:gQualTL}) in their extended timeline set for non-register parameters.
  \item \lighttt{cuda\_generic\_and\_async\_proxy} if any async proxy instructions are involved (TMA, wgmma, tcgen05); these instructions include \lighttt{cuda\_async\_proxy\_retired} (def~\ref{sec:gQualTL}) in their extended timeline set for non-register parameters.
  \item \lighttt{cuda\_temporal} may also be used if only temporal-only synchronization is required.
\end{itemize}
As a special case, \lighttt{Fence(wgmma\_fence\_1, wgmma\_fence\_2)} at \lighttt{cuda\_warpgroup}-scope (def~\ref{sec:gCollUnit}) should be used before any accesses to wgmma registers, even if no prior access occurs (this is \emph{not} enforced by sync-check, but will result in a ptxas warning and suboptimal CUDA code if violated).

With $(\iota, n)$ denoting a thread in common, and with
\begin{itemize}
    \item $\mathsf{vf} = \mathsf{VF_{temp}}$ (def~\ref{sec:gVisFlag}) and $q$ denoting a qualitative timeline in common to $\tau_s^\mathrm{post}.\mathrm{temp}$ and the extended timeline set, in the case of temporal-only synchronization.
    \item $\mathsf{vf} = \mathsf{VF_{full}}$ for non-atomic acceses or $\mathsf{vf} = \mathsf{VF_{atom}}$ for atomics, and $q$ denoting a qualitative timeline in common to $\tau_s^\mathrm{post}.\mathrm{full}$ and the extended timeline set, in the other case.
\end{itemize}
this pattern is implemented in sync-check by the augment step (Section~\ref{sec:Augment}) adding the timeline signature $((\iota, n), q, \mathsf{vf})$ to visibility records, which satisfy the checking requirements for the subsequent read (Section~\ref{sec:ChecksOnRead}) or mutate (Section~\ref{sec:ChecksOnMutate}).



\magicSubsection{Intended Usage Patterns -- Arrive-Await Pairing, Trailing Barrier Expr}{sec:ArriveAwaitPairing}

\begin{gather*}
    \texttt{Arrive(\codecomment{...}) >> $e_a$} \to \texttt{Await(\codecomment{...}, $e_w$, $n$)} \\
    \texttt{Read|Mutate >}\texttt{> } e_t \to \texttt{Await(\codecomment{...}, $e_w$, $n$)}
\end{gather*}

In the Exo-GPU abstraction, each barrier array element contains its own arrive count and await count, both initially 0 (Section~\ref{sec:SyncEnv}).
The \lighttt{Arrive} statement may reference multiple barrier array elements (say, $B$-many), using multiple barrier expressions (def~\ref{sec:gBarrierExpr}) separated by \texttt{>}\texttt{>}.
Each ``batch'' of $B$-many instances of such an \lighttt{Arrive} statement (issued with $B$-many different thread collectives) results in incrementing the arrive count of each referenced barrier array element by 1.
This includes the common case where $B=1$.

The \lighttt{Await} statement must contain only a single barrier expression $e_w$, and it must reference only one barrier array element.
The behavior of the \lighttt{Await} changes based on $n$.

If $n \ge 0$, then this is the ``arrive-indexed'' case.
With $a$ being the arrive count of the referenced barrier array element, the eligible batches are the first $(a-n)$-many batches of \lighttt{Arrive} statement instances that reference the same barrier array element.
There are dependency edges from the eligible batches to the await.

If $n = \texttt{\textasciitilde lag}$ with \texttt{\textasciitilde} being the 2's complement and $\texttt{lag} \ge 0$, then this is the ``await-indexed'' case.
With $w$ being the await count of the referenced barrier, the eligible batches are the first $(w - \texttt{lag} + 1)$-many batches of \lighttt{Arrive} statement instances that reference the same barrier array element.
There are dependency edges from the eligible batches to the await.
Afterwards, the await count of the referenced barrier is incremented by 1.

In both cases, additional dependency edges are formed from reads and mutates executed by a statement instance with a trailing barrier expression $e_t$, when $e_t$ and $e_w$ reference the same barrier array element, and the read or mutate appears in the trace prior to at least one eligible batch of \lighttt{Arrive} statement instances.

This pattern is implemented in sync-check through the shared pending await state between \lighttt{Arrive} (Section~\ref{sec:ArriveSemantics}) and \lighttt{Await} (Section~\ref{sec:AwaitSemantics}), and the initial pending awaits added for reads/mutates that include a trailing barrier expression (Section~\ref{sec:VisRecordCreation}).
The to-be-described convergence requirements for multiple barriers (Section~\ref{sec:BarrierMulticast}) are required to ensure batches of arrives execute as intended.

Scheduling Functions:
\begin{itemize}
  \item \texttt{insert\_arrive(\codecomment{...}, $\tau_s^\mathrm{pre}$, $e^*$)}
  \item \texttt{insert\_await(\codecomment{...}, $e$, $\tau_s^\mathrm{post}$)}
  \item \texttt{set\_trailing\_barrier\_expr(\codecomment{...}, $e$)}
\end{itemize}


\magicSubsection{Intended Usage Patterns -- Multiple Memory Accesses}{sec:MultipleMemoryAccesses}

\begin{equation*}
    \texttt{Read|Mutate} \to \texttt{Read|Mutate}
\end{equation*}

A thread may read/mutate an array element and then read/mutate that element again without intervening synchronization when the first access is not out-of-order and the initial timeline signature of the first access is included in the extended timeline signature set of the second access.
This pattern is implemented in sync-check through a timeline signature (def~\ref{sec:gTlSig}) with $\mathsf{VF_{full}}$ (def~\ref{sec:gVisFlag}) being added upon interpreting a non-out-of-order read or mutate (Section~\ref{sec:VisRecordCreation}).

Two atomic mutates may also access the same array element concurrently as long as their atomic qualitative timeline sets (Section~\ref{sec:InstrTL}) intersect.
Sequential-parallel equivalence (def~\ref{sec:gSeqParEquivalence}) is only possible in this case since Exo atomics don't return the updated value.
This pattern is implemented in sync-check through a timeline signature with $\mathsf{VF_{atom}}$ that is added upon the first atomic access (Section~\ref{sec:VisRecordCreation}) and used to pass the check in the second atomic access (Section~\ref{sec:ChecksOnMutate}).

% atomic, non-ooo


\magicSubsection{Intended Usage Patterns -- Multiple Synchronization Statements (Transitivity)}{sec:Transitivity}

\begin{equation*}
    \texttt{Await(\codecomment{...}, $\tau_s^\mathrm{post}$, \codecomment{...})|Fence(\codecomment{...}, $\tau_s^\mathrm{post}$)} \to
    \texttt{Fence($\tau_s^\mathrm{pre}$, \codecomment{...})|Arrive($\tau_s^\mathrm{pre}$)\codecomment{...}}
\end{equation*}

A dependency edge forms between an \lighttt{Await} or \lighttt{Fence} statement instance and a subsequent \lighttt{Arrive} or \lighttt{Fence} statement instance when the thread collectives executing the two statement instances have a thread in common, and there exists a qualitative timeline $q$ in the intersection of the first statement instance's $\tau_s^\mathrm{post}$ and the second statement instance's $\tau_s^\mathrm{pre}$ (def~\ref{sec:gSyncTL}).

With $(\iota, n)$ denoting a thread in common, and with $q$ as described above, this pattern is implemented in sync-check through the timeline signature $((\iota, n), q, \mathsf{VF_{issue}})$ that is added by the augment (Section~\ref{sec:Augment}) for the first statement instance and detected by the witness (Section~\ref{sec:Witness}) of the second statement instance.

% reason for async proxy retired thing


\magicSubsection{Fence Usage}{sec:FenceUsage}

\texttt{Fence($\tau_s^\mathrm{pre}$, $\tau_s^\mathrm{post}$)}

\texttt{insert\_fence(\codecomment{...}, $\tau_s^\mathrm{pre}$, $\tau_s^\mathrm{post}$)}

Besides the special case \lighttt{Fence(wgmma\_fence\_1, wgmma\_fence\_2)}, which must be at \lighttt{cuda\_warpgroup}-scope (def~\ref{sec:gCollUnit}) and must appear before other usages of wgmma registers, all \lighttt{Fence} statements must be of the form \lighttt{Fence($\tau_s^\mathrm{pre}$, $\tau_s^\mathrm{post}$)} with
\begin{itemize}
    \item the statement appearing at \lighttt{cuda\_thread}, \lighttt{cuda\_warp}, \lighttt{cuda\_cta\_in\_cluster}, or \lighttt{cuda\_cluster} scope (def~\ref{sec:gCollUnit}).
    \item $\tau_s^\mathrm{pre}$ being \lighttt{Sm80\_generic}, or another synchronization timeline whose timeline sets are subsets of the respective timeline sets of \lighttt{Sm80\_generic} (def~\ref{sec:gSyncTL}).
    \item $\tau_s^\mathrm{post}$ being \lighttt{cuda\_generic\_and\_async\_proxy}, or another synchronization timeline whose timeline sets are subsets of the respective timeline sets of \lighttt{cuda\_generic\_and\_async\_proxy} (def~\ref{sec:gSyncTL}).
\end{itemize}
These non-wgmma \lighttt{Fence} configurations are referred to as \myKeyA{garden-variety fences}.
% Referenced in glossary entry gGardenVarietyFence.

If the \lighttt{Fence} statement appears at \lighttt{cuda\_cluster}-scope but not at \lighttt{cuda\_cta\_in\_cluster}-scope, then it is subject to the solitary barrier requirement (Section~\ref{sec:SolitaryBarrier}).


\magicSubsection{Cluster Sync Usage}{sec:ClusterSyncUsage}

A barrier variable $z$ with cluster sync barrier mechanism may be declared with

\texttt{$z$: barrier @ CudaClusterSync}

\texttt{insert\_barrier\_alloc(\codecomment{...}, $z$, None, [], CudaClusterSync)}

The valid synchronization timelines (def~\ref{sec:gSyncTL}) for the \lighttt{Arrive} and \lighttt{Await} using the cluster sync barrier mechanism are the same as for garden-variety fences (Section~\ref{sec:FenceUsage}).

\mainKey{Statically-checked Requirements:}
\begin{itemize}
  \item Split barrier basic requirements (Section~\ref{sec:SplitBarrierBasic})
  \item Barrier guarding requirement (Section~\ref{sec:BarrierGuarding}) with only the arrive-first configuration allowed
  \item Solitary barrier requirement (Section~\ref{sec:SolitaryBarrier})
  \item Barrier expressions (def~\ref{sec:gBarrierExpr}) must use only point indices, not intervals.
  \item \lighttt{Await} must have $n=0$.
  \item \lighttt{Arrive} and \lighttt{Await} must be at \lighttt{cuda\_cluster}-scope (def~\ref{sec:gCollUnit}).
  \item The \lighttt{Arrive} and \lighttt{Await} statements for a given barrier array element must be executed by the same \myKeyA{thread collective} (def~\ref{sec:gThreadCollective}).
    This is enforced by requiring base thread equality (Section~\ref{sec:DistributedMemoryBaseThreads}) between collective indexing pairs collected both from \lighttt{Arrive} and \lighttt{Await} statements for $z$.
\end{itemize}

% solitary, sync-exempt, barrier variable vs CUDA implicit state


\magicSubsection{Commit Group Usage}{sec:CommitGroupUsage}

A barrier variable $z$ with commit group barrier mechanism may be declared with

\texttt{$z$: barrier[$e^*$] @ CudaCommitGroup}

\texttt{insert\_barrier\_alloc(\codecomment{...}, $z$, None, [$e^*$], CudaCommitGroup)}

where the array size $e^*$ is subject to distributed memory analysis (Section~\ref{sec:DistributedMemory}).
In CUDA, the commit group is implicit state local to a thread, warp, or warpgroup; however in Exo-GPU we explicitly shard this barrier state onto the implicit thread state with distributed memory analysis.
This motivates the solitary barrier requirement (Section~\ref{sec:SolitaryBarrier}).

The synchronization timelines (def~\ref{sec:gSyncTL}) $\tau_s^\mathrm{pre}$ for the \lighttt{Arrive} and $\tau_s^\mathrm{post}$ for the \lighttt{Await} must match one of the rows in this table, which also defines the expected collective unit $\tau_u$ (def~\ref{sec:gCollUnit}).

\begin{tabular}{|r|r|r|}
$\tau_s^\mathrm{pre}$ & $\tau_s^\mathrm{post}$ & $\tau_u$ \\
\texttt{Sm80\_cp\_async} & \texttt{cuda\_in\_order} & \texttt{cuda\_thread} \\
\texttt{tma\_to\_gmem\_async} (Section~\ref{sec:Tma}) & \texttt{cuda\_generic\_or\_async\_proxy} & \texttt{cuda\_warp} \\
\texttt{wgmma\_async} & \texttt{cuda\_generic\_or\_async\_proxy} & \texttt{cuda\_warpgroup}
\end{tabular}

\mainKey{Statically-checked Requirements:}
\begin{itemize}
  \item Split barrier basic requirements (Section~\ref{sec:SplitBarrierBasic})
  \item Solitary barrier requirement (Section~\ref{sec:SolitaryBarrier})
  \item Barrier expressions (def~\ref{sec:gBarrierExpr}) must use only point indices, not intervals.
  \item \lighttt{Await} must have $n \ge 0$, i.e. this is an arrive-indexed barrier (Section~\ref{sec:ArriveAwaitPairing}, Section~\ref{sec:AwaitSemantics}).
  \item \lighttt{Arrive} and \lighttt{Await} must be at $\tau_u$-scope as defined in the above table (def~\ref{sec:gCollUnit}).
  \item The \lighttt{Arrive} and \lighttt{Await} statements for a given barrier array element must be executed by the same \myKeyA{thread collective} (def~\ref{sec:gThreadCollective}).
    This is enforced by requiring base thread equality (Section~\ref{sec:DistributedMemoryBaseThreads}) between collective indexing pairs collected both from \lighttt{Arrive} and \lighttt{Await} statements for $z$.
\end{itemize}

% solitary, sync-exempt, barrier variable vs CUDA implicit state


\magicSubsection{Mbarrier Usage}{sec:MbarrierUsage}

A barrier variable $z_a$ with mbarrier barrier mechanism may be declared with

\texttt{$z_a$: barrier[$e^*$] @ CudaMbarrier}

\texttt{insert\_barrier\_alloc(\codecomment{...}, $z_a$, None, [$e^*$], CudaMbarrier)}

\texttt{$z_a$: barrier($z_g$) [$e^*$] @ CudaMbarrier}

\texttt{insert\_barrier\_alloc(\codecomment{...}, $z_a$, $z_g$, [$e^*$], CudaMbarrier)}

where the array size $e^*$ is subject to distributed memory analysis (Section~\ref{sec:DistributedMemory}), and the $z_g$ variable, if present, means that $z_a$ is explicitly guarded-by $z_g$ (def~\ref{sec:gGuardedBy}).
This is used in the barrier guarding requirement (Section~\ref{sec:BarrierGuarding}); in short, \lighttt{Arrive} statements using $z_a$ must be matched with \lighttt{Await} statements using $z_g$.

Each barrier array element is implemented as a \emph{ring buffer} of CUDA mbarrier objects (def~\ref{sec:gMbarrierRingBuffer}), resident in a single CTA.
You cannot implement ring buffering explicitly in Exo-GPU (the indexing pattern required will not pass distributed memory analysis), which is an intentional break from Exo's imperative programming style in order to simplify sync-check.

The first synchronization timeline (def~\ref{sec:gSyncTL}) $\tau_s^\mathrm{pre}$ for the \lighttt{Arrive} should be \lighttt{cuda\_in\_order}, \lighttt{Sm80\_cp\_async}, or \lighttt{cuda\_temporal}.
The second synchronization timeline $\tau_s^\mathrm{post}$ for the \lighttt{Await} should be \lighttt{cuda\_in\_order}, \lighttt{cuda\_generic\_or\_async\_proxy}, or \lighttt{cuda\_temporal}.
(Avoid using \lighttt{cuda\_generic\_or\_async\_proxy} unless required; in particular guarding against write-after-read (WAR) hazards for a producer warp using TMA requires only \lighttt{cuda\_temporal} for the producer warp's \lighttt{Await}).
For TMA-to-SMEM instructions, use a trailing barrier expression for the instruction directly (Section~\ref{sec:Tma}).

\mainKey{Statically-checked Requirements:}
\begin{itemize}
  \item Split barrier basic requirements (Section~\ref{sec:SplitBarrierBasic}).
  \item Barrier guarding requirement, either arrive-first or await-first (Section~\ref{sec:BarrierGuarding}).
  \item Barrier expressions (def~\ref{sec:gBarrierExpr}) must meet the barrier multicast requirements (Section~\ref{sec:BarrierMulticast}).
  \item Each \lighttt{Await} for a given variable $z_a$ must have the same $n$ value, and $n < 0$, i.e. this is an await-indexed barrier (Section~\ref{sec:ArriveAwaitPairing}, Section~\ref{sec:AwaitSemantics}).
  \item Distributed memory analysis (Section~\ref{sec:DistributedMemory}) must be able to map each barrier array element into only a single CTA (TODO, explain how?).
  \item Each \lighttt{Arrive} for a given barrier array element must be executed by the same \myKeyA{thread collective} (def~\ref{sec:gThreadCollective}).
    This is enforced by requiring base thread equality (Section~\ref{sec:DistributedMemoryBaseThreads}) between collective indexing pairs collected both from all \lighttt{Arrive} statements for $z_a$.
  \item The above requirement applies separately for all \lighttt{Await} statements for $z_a$.
  \item The deduced ring buffer size must be positive (def~\ref{sec:gMbarrierRingBuffer}).
\end{itemize}

% multicast, guarding, hidden ring buffer & why ...


\magicSubsection{Barrier Multicast}{sec:BarrierMulticast}

% thread pitch multiple of blockDim, multicast convergence, home barrier expr

The \lighttt{Arrive} statement may contain multiple barrier expressions (Section~\ref{sec:gBarrierExpr}), some containing intervals instead of points.
For example,
\begin{equation*}
    \texttt{Arrive(\codecomment{...}) >> $z$[m, :] >> $z$[:, n]}
\end{equation*}
There must be a unique \myKeyA{home barrier} (def~\ref{sec:gHomeBarrier}), which is the unique barrier in the intersection of all barrier expressions.
This is determined from a deduced home barrier expression (in this case, \texttt{z[m, n]}).

In the context of a given \lighttt{Arrive} statement, an iterator (control variable) $y$ is a \myKeyA{multicast iterator} if it appears at some index $j$ of a barrier expression, and another barrier expression has an interval at index $j$.
For example,
\begin{equation*}
    \texttt{Arrive(\codecomment{...}) >> $z$[m, n] >> $z$[m, :]}
\end{equation*}
has only \texttt{n} being a multicast iterator.

Currently only \lighttt{CudaMbarrier}-mechanism split barriers (Section~\ref{sec:MbarrierUsage}) support multicasting.
Each multicast iterator must have a thread pitch (def~\ref{sec:gThreadPitch}) that is a multiple of \lighttt{blockDim} (Section~\ref{sec:CudaDeviceFunction}).
That is, multicast iterators must correspond to \lighttt{cuda\_threads} loops that iterate in the CTA-in-cluster dimension (Section~\ref{sec:CollTiling}).

We impose a conservative convergence requirement for multicast iterators.
This is the only case in Exo-GPU where convergence is enforced other than through scoping.
Each multicast iterator is defined by some \lighttt{cuda\_threads} loop (Section~\ref{sec:CollTilingCudaThreads}).
For a given \lighttt{Arrive} statement, it must not have any parent \lighttt{if} or \emph{sequential} \lighttt{for} loop statements that are not also parents of each multicast iterator's defining \lighttt{cuda\_threads} loop (Figure~\ref{fig:BadMulticastExample}).

\begin{figure}[t]
\codehrule
\input{b_samples/BadMulticastExample.0.tex}
\caption{Multicast Convergence Requirement Violation}
\label{fig:BadMulticastExample}
\codehrule
\end{figure}


\magicSubsection{Split Barrier Basic Requirements}{sec:SplitBarrierBasic}

For a given barrier variable $z$,
\begin{itemize}
  \item At least one \lighttt{Arrive} using $z$ must exist, each using the same $\tau_s^\mathrm{pre}$ (def~\ref{sec:gSyncTL}).
  \item At least one \lighttt{Await} using $z$ must exist, each using the same $\tau_s^\mathrm{post}$ (def~\ref{sec:gSyncTL}).
  \item \lighttt{Arrive} statements must have a valid home barrier expression (def~\ref{sec:gHomeBarrier}).
  \item \lighttt{Await} statements must have only a single barrier expression (def~\ref{sec:gBarrierExpr}) with all indices being points.
  \item Two different barrier variables must not be guarded-by the same barrier variable (def~\ref{sec:gGuardedBy}).
  \item The multicasts (def~\ref{sec:gMulticasts}) of all \lighttt{Arrive} statements using $z$ must be the same.
\end{itemize}


\magicSubsection{Barrier Guarding Requirement}{sec:BarrierGuarding}

This section applies only to barrier mechanisms that list it as a requirement.

Each barrier variable is guarded-by some other barrier variable:
\begin{itemize}
    \item Declare a barrier variable $z$ with no explictly-guarded-by variable, and with no other barrier variable being explicitly-guarded-by $z$ (def~\ref{sec:gBarrierVariable}) to have $z$ be guarded by itself.
    \item Declare a barrier variable $z_0$ with no explictly-guarded-by variable, and declare $z_1$ to be explicitly guarded-by $z_0$, and no other barrier variable being explicitly guarded-by $z_1$ to have $z_0$ and $z_1$ guard each other.
    \item See the reference for ``guarded-by'' (def~\ref{sec:gGuardedBy}) for other use cases.
\end{itemize}
For each barrier variable $z_a$ subject to the barrier guarding requirement, the Exo-GPU program must pass some static analysis, specified in the reference (def~\ref{sec:gBarrierGuardingRequirement}),  that guarantees that each thread alternates between executing instances of
\begin{itemize}
    \item \lighttt{Arrive} statements involving $z_a$ (if this is the first stmt in the cycle, it's \myKeyA{arrive-first} usage)
    \item \lighttt{Await} statements involving $z_g$ (if this is the first stmt in the cycle, it's \myKeyA{await-first} usage)
\end{itemize}
as occurs in Figure~\ref{fig:mbarrier_2_cycle}.

\begin{figure}[t]
\codehrule
\input{b_samples/mbarrier_2_cycle.0.tex}
\caption{Correct barrier guarding for a barrier guard cycle of length 2 (def~\ref{sec:gBarrierGuardCycle}).}
\label{fig:mbarrier_2_cycle}
\codehrule
\end{figure}

If any instruction calls have trailing barrier expressions (def~\ref{sec:gTrailingBarrierExpr}) that use $z_a$, these calls must occur after the \lighttt{Await} and before the \lighttt{Arrive} in the above cycle.
Furthermore, the barrier array elements referenced by the trailing barrier expression must be a non-strict subset of the barrier array elements referenced in the \lighttt{Arrive}.

The await-first usage must be used for any barrier variable that is not guarded-by itself.


\magicSubsection{Solitary Barrier Requirement}{sec:SolitaryBarrier}

There must not be two barrier variables with the same lowered barrier type live in the same scope, if the lowered barrier type of the barrier appears in the following list:
\begin{itemize}
  \item \lighttt{cluster\_sync}, all cluster-sync mechanism split barriers (Section~\ref{sec:ClusterSyncUsage}).
  \item \lighttt{Sm80\_commit\_group}, commit-group mechanism split barriers for sm\_80 non-bulk \lighttt{cp.async} (Section~\ref{sec:CommitGroupUsage}).
  \item \lighttt{tma\_to\_gmem\_commit\_group}, commit-group mechanism split barriers for TMA copies out of SMEM (Section~\ref{sec:CommitGroupUsage}).
  \item \lighttt{wgmma\_commit\_group}, commit-group mechanism split barriers for wgmma (Section~\ref{sec:CommitGroupUsage}).
\end{itemize}
Additionally, a \lighttt{Fence} statement (Section~\ref{sec:FenceUsage}) at \lighttt{cuda\_cluster} scope, but not at \lighttt{cuda\_cta\_in\_cluster} scope (def~\ref{sec:gCollUnit}), must not appear in a scope with a live barrier variable with \lighttt{cluster\_sync} lowered barrier type.

\FloatBarrier
\newpage

