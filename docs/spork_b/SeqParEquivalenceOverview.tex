\magicSubsection{Sequential-Parallel Equivalence Overview}{sec:SeqParEquivalenceOverview}

The output values of an Exo-GPU program are defined by \myKeyA{sequential semantics}, where
\begin{itemize}
\item All loops, regardless of loop mode (def~\ref{sec:gLoopMode}), have sequential step semantics (with \textsc{LoopStep}/\textsc{LoopDone} rules).
\item Synchronization statements (Section~\ref{sec:SyncUsage}) are no-ops.
\item \lighttt{with CudaDeviceFunction} (Section~\ref{sec:CudaDeviceFunction}) and \lighttt{with CudaWarps} (Section~\ref{sec:CollTilingCudaWarps}) are like \lighttt{if True}.
\item The sequential semantics of instructions are a function only of its behavior (Section~\ref{sec:InstrBehavior}), not its memory access information (Section~\ref{sec:InstrMemoryAccess}).

\end{itemize}
Rewrite operations guarantee functional equivalence under sequential semantics.
The user starts by defining a simple initial procedure $p_1$, then schedules the procedure by issuing a series of rewrite operations $R_1 ... R_{N-1}$ to create transformed procedures $p_2 ... p_N$.
Key to checking functional equivalence between $p_1$ and the generated CUDA program is our notion of \myKeyA{sequential-parallel equivalence}:
the \emph{final} procedure $p_N$, interpreted with sequential semantics, must be functionally equivalent to \emph{itself} interpreted as a parallelized CUDA program.
This equivalence is not implemented by the compiler;
instead, the user (or scheduling library) should insert synchronization, then run sync-check (Section~\ref{sec:SyncSemantics}) to check that this correctly upholds sequential-parallel equivalence.
We currently only implement sync-check for given concrete problem sizes.

The combination of scheduling rewrites and sequential-parallel equivalence transitively ensures functional equivalence of the initial procedure $p_1$ and the generated CUDA program:

\noindent
\begin{tabular}{l c c c c c c c c}
Rewrites: & $p_1$ & $\xrightarrow{R_1}$ & $p_2$ & ... & $\xrightarrow{R_{N-1}}$ & $p_N$ & \footnotesize $\text{(sync check)}$ & $p_N$ \\[2pt]
Behavior: & $\textsf{seq}\qc{p_1}$ & $\equiv$ & $\textsf{seq}\qc{p_2}$ & ... & $\equiv$ & $\textsf{seq}\qc{p_N}$ & $\equiv$ & $\textsf{par}\qc{p_N}$
\end{tabular}

\hphantom{xyzzy} where $\textsf{seq}\qc{p}$ means ``$p$ interpreted with sequential semantics''\\
\hphantom{xyzzy} and $\textsf{par}\qc{p}$ means ``$p$ interpreted as parallelized CUDA''.

This design allows us to continue to use Exo's scheduling libraries unmodified.
New Exo-GPU rewrite operations that introduce parallelism and asynchrony don't require any checking during scheduling.
An additional benefit of this design is that Exo-GPU delivers value even if the user prefers to handwrite the full program, rather than use the scheduling rewrites.
In this case, we still have a guarantee that the handwritten program, interpreted under sequential semantics, will be functionally equivalent to the generated CUDA program (at least for the set of problem sizes given to sync check).

Key to this checking is that each statement in CUDA scope (def~\ref{sec:gCudaScope}) is annotated with a \myKeyA{collective tiling} (Section~\ref{sec:CollTiling}), which statically encodes which threads within the CUDA cluster (def~\ref{sec:gCluster}) are used for what work (Figure~\ref{fig:OverviewCollTiling}).
Deducing this is part of the \myKeyA{collective analysis} (def~\ref{sec:gCollAnalysis}) compilation step, which only applies to the final scheduled proc.

Note, the \lighttt{cuda\_tasks} loops (def~\ref{sec:gCudaTasksLoop}), which distribute work to clusters, are not subject to the same level of static analysis due to the nondeterminism inherent in Blackwell cluster launch control and to allow for future non-affine work distribution like Morton swizzling.

\begin{figure}[t!]
\codehrule
\input{b_samples/OverviewCollTiling.0.tex}
\caption{Simple Collective Tilings, which imply the stated thread mapping functions (def~\ref{sec:gThreadMappingFunction})}
\label{fig:OverviewCollTiling}
\codehrule
\end{figure}


