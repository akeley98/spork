\magicSection{Collective Units \& Collective Types}{sec:CollType}

We use collective types $\delta$ to describe a quantity and arrangement of threads within a cluster (def~\ref{sec:gCluster}), such as ``single thread'', ``warp'', ``CTA'', ``one warp from a pair of CTAs''.
These are unpacked from a collective unit $\tau_u$ defined in the frontend language (def~\ref{sec:gCollUnit}).
A collective type consists of two equal-length tuples: a domain and a box.
The dimension $M$ of the collective type is the length of these tuples.
The \myKeyA{domain} ($\delta.D_0...\delta.D_{M-1}$): $\mathbb{N}_{\ge2}^M$ describes an organization of the threads in a cluster into an $M$-dimensional grid.
The \myKeyA{box} ($\delta.B_0...\delta.B_{M-1}$): $\mathbb{N}_\top^M$ describes the number of threads on each dimension to select (the special value $\top$ indicates ``no requirement'').

We first define a linear ordering of threads in a cluster, then extend to multidimensional coordinates.
The local thread index of a thread is \lighttt{cluster\_ctarank * blockDim.x + threadIdx.x}
i.e., the threads in a cluster are numbered in (\lighttt{cluster\_ctarank, threadIdx.x})-lexicographical order (Exo-GPU parallelizes on the x dimension only).

For a given domain, we derive the \myKeyA{dimension thread pitch} $\delta.P_i$:
\begin{align*}
    \delta.P_i = \prod_{k=i+1}^{M-1} \delta.D_k
\end{align*}
and we define the mapping $\mathsf{toLocal}(D, c)$ (def~\ref{sec:gToLocal}), which converts a domain $D$ and coordinates $c$ to a local thread index; the coordinates $[0, \delta.D_0-1]_\mathbb{N} \times ... \times [0, \delta.D_{M-1}-1]_\mathbb{N}$ get mapped to local thread indices in lexicographical order.
The product of the domain coordinates $\delta.D_0 \times ... \times \delta.D_{M-1}$ must be equal to the number of threads in the cluster (\lighttt{clusterDim.x * blockDim.x}).

\magicSubsection{Collective Types \& Thread Collectives}{sec:CollTypeThreadCollective}

A thread collective (def~\ref{sec:gThreadCollective}) is an instance of a collective type $\delta$ when all threads are in the same cluster, and, with $\mu: \mathcal{P}(\mathbb{N})$ being the set of local thread indices of the threads, there exist sets $C_0 ... C_{M-1}: \mathcal{P}(\mathbb{N})$ such that
\begin{itemize}
  \item $C_i \subseteq [0, \delta.D_i - 1]$
  \item $\delta.B_i \ne \top \implies \exists x.\; C_i = [x, x + \delta.B_i - 1]_\mathbb{N}$.
  \item $\mu = \{ \mathsf{toLocal}(\delta.D, c) \mid c \in C_0 \times ... \times C_{M-1}\}$
\end{itemize}
If we have no $\top$ box coordinates, this is specifying that the thread collective forms a cuboid of dimensions $\delta.B_0 \times ... \times \delta.B_{M-1}$ when its threads are arranged in the grid implied by $\delta.D$; therefore, the number of threads in the thread collective is the product of the box coordinates (Figure~\ref{fig:CollTypeExample}).

\begin{figure}[h!]
\sffamily
\hspace{-12mm}
\begin{tikzpicture}[node distance=0mm]
\node(t000a) [CollTypeExampleStyle] {(0, 0)\\0 $ \notin \mu$\\0, 0, 0, 0};
\node(t000b) [CollTypeExampleStyle, right=of t000a, xshift=6mm] {(0, 127)\\127 $ \notin \mu$\\0, 0, 0, 127};
\draw [dotted] (t000a) -- (t000b);
\node(t001a) [CollTypeExampleStyle, right=of t000b, xshift=2mm, yshift=-3mm] {(0, 128)\\128 $ \notin \mu$\\\redBox{0}, 0, 1, 0};
\node(t001b) [CollTypeExampleStyle, right=of t001a, xshift=6mm] {(0, 255)\\255 $ \notin \mu$\\0, 0, 1, 127};
\draw [dotted] (t001a) -- (t001b);
\node(t002a) [CollTypeExampleStyle, right=of t001b, xshift=2mm, yshift=-3mm] {(0, 256)\\256 $ \notin \mu$\\0, 0, \greenBox{2}, \blueBox{0}};
\node(t002b) [CollTypeExampleStyle, right=of t002a, xshift=6mm] {(0, 383)\\383 $ \notin \mu$\\0, 0, \greenBox{2}, \blueBox{127}};
\draw [dotted] (t002a) -- (t002b);
\node(t010a) [CollTypeExampleStyle, below=of t000a, yshift=-2mm] {(1, 0)\\384 $ \notin \mu$\\0, 1, 0, 0};
\node(t010b) [CollTypeExampleStyle, right=of t010a, xshift=6mm] {(1, 127)\\511 $ \notin \mu$\\0, 1, 0, 127};
\draw [dotted] (t010a) -- (t010b);
\node(t011a) [CollTypeExampleStyle, right=of t010b, xshift=2mm, yshift=-3mm] {(1, 128)\\512 $ \notin \mu$\\0, 1, 1, 0};
\node(t011b) [CollTypeExampleStyle, right=of t011a, xshift=6mm] {(1, 255)\\639 $ \notin \mu$\\0, 1, 1, 127};
\draw [dotted] (t011a) -- (t011b);
\node(t012a) [CollTypeExampleStyle, right=of t011b, xshift=2mm, yshift=-3mm] {(1, 256)\\640 $ \notin \mu$\\0, 1, 2, 0};
\node(t012b) [CollTypeExampleStyle, right=of t012a, xshift=6mm] {(1, 383)\\767 $ \notin \mu$\\0, 1, 2, 127};
\draw [dotted] (t012a) -- (t012b);

\node(t100a) [CollTypeExampleStyle, below=of t010a, yshift=-4mm, xshift=-4mm] {(2, 0)\\768 $ \notin \mu$\\\redBox{1}, \yellowBox{0}, 0, 0};
\node(t100b) [CollTypeExampleStyle, right=of t100a, xshift=6mm] {(2, 127)\\895 $ \notin \mu$\\1, 0, 0, 127};
\draw [dotted] (t100a) -- (t100b);
\node(t101a) [CollTypeExampleStyle, right=of t100b, xshift=2mm, yshift=-3mm] {(2, 128)\\896 $ \notin \mu$\\\redBox{1}, 0, 1, 0};
\node(t101b) [CollTypeExampleStyle, right=of t101a, xshift=6mm] {(2, 255)\\1023 $ \notin \mu$\\1, 0, 1, 127};
\draw [dotted] (t101a) -- (t101b);
\node(t102a) [CollTypeExampleStyle, violetstyle, right=of t101b, xshift=2mm, yshift=-3mm] {(2, 256)\\1024 $ \in \mu$\\1, 0, 2, 0};
\node(t102b) [CollTypeExampleStyle, violetstyle, right=of t102a, xshift=6mm] {(2, 383)\\1151 $ \in \mu$\\1, 0, 2, 127};
\draw [dotted] (t102a) -- (t102b);
\node(t110a) [CollTypeExampleStyle, below=of t100a, yshift=-2mm] {(3, 0)\\1152 $ \notin \mu$\\\redBox{1}, \yellowBox{1}, 0, 0};
\node(t110b) [CollTypeExampleStyle, right=of t110a, xshift=6mm] {(3, 127)\\1279 $ \notin \mu$\\1, 1, 0, 127};
\draw [dotted] (t110a) -- (t110b);
\node(t111a) [CollTypeExampleStyle, right=of t110b, xshift=2mm, yshift=-3mm] {(3, 128)\\1280 $ \notin \mu$\\1, 1, 1, 0};
\node(t111b) [CollTypeExampleStyle, right=of t111a, xshift=6mm] {(3, 255)\\1407 $ \notin \mu$\\1, 1, 1, 127};
\draw [dotted] (t111a) -- (t111b);
\node(t112a) [CollTypeExampleStyle, violetstyle, right=of t111b, xshift=2mm, yshift=-3mm] {(3, 256)\\1408 $ \in \mu$\\1, 1, 2, 0};
\node(t112b) [CollTypeExampleStyle, violetstyle, right=of t112a, xshift=6mm] {(3, 383)\\1535 $ \in \mu$\\1, 1, 2, 127};
\draw [dotted] (t112a) -- (t112b);

\node(t200a) [CollTypeExampleStyle, below=of t110a, yshift=-4mm, xshift=-4mm] {(4, 0)\\1536 $ \notin \mu$\\2, 0, 0, 0};
\node(t200b) [CollTypeExampleStyle, right=of t200a, xshift=6mm] {(4, 127)\\1663 $ \notin \mu$\\2, 0, 0, 127};
\draw [dotted] (t200a) -- (t200b);
\node(t201a) [CollTypeExampleStyle, right=of t200b, xshift=2mm, yshift=-3mm] {(4, 128)\\1664 $ \notin \mu$\\\redBox{2}, \yellowBox{0}, 1, 0};
\node(t201b) [CollTypeExampleStyle, right=of t201a, xshift=6mm] {(4, 255)\\1791 $ \notin \mu$\\2, 0, 1, 127};
\draw [dotted] (t201a) -- (t201b);
\node(t202a) [CollTypeExampleStyle, right=of t201b, xshift=2mm, yshift=-3mm] {(4, 256)\\1792 $ \notin \mu$\\2, 0, 2, \blueBox{0}};
\node(t202b) [CollTypeExampleStyle, right=of t202a, xshift=6mm] {(4, 383)\\1919 $ \notin \mu$\\2, 0, 2, \blueBox{127}};
\draw [dotted] (t202a) -- (t202b);
\node(t210a) [CollTypeExampleStyle, below=of t200a, yshift=-2mm] {(5, 0)\\1920 $ \notin \mu$\\2, 1, \greenBox{0}, 0};
\node(t210b) [CollTypeExampleStyle, right=of t210a, xshift=6mm] {(5, 127)\\2047 $ \notin \mu$\\2, 1, 0, 127};
\draw [dotted] (t210a) -- (t210b);
\node(t211a) [CollTypeExampleStyle, right=of t210b, xshift=2mm, yshift=-3mm] {(5, 128)\\2048 $ \notin \mu$\\2, \yellowBox{1}, \greenBox{1}, 0};
\node(t211b) [CollTypeExampleStyle, right=of t211a, xshift=6mm] {(5, 255)\\2175 $ \notin \mu$\\2, 1, 1, 127};
\draw [dotted] (t211a) -- (t211b);
\node(t212a) [CollTypeExampleStyle, right=of t211b, xshift=2mm, yshift=-3mm] {(5, 256)\\2176 $ \notin \mu$\\2, 1, \greenBox{2}, 0};
\node(t212b) [CollTypeExampleStyle, right=of t212a, xshift=6mm] {(5, 383)\\2303 $ \notin \mu$\\2, 1, 2, 127};
\draw [dotted] (t212a) -- (t212b);

\node(t300a) [CollTypeExampleStyle, below=of t210a, yshift=-4mm, xshift=-4mm] {(6, 0)\\2304 $ \notin \mu$\\3, 0, 0, 0};
\node(t300b) [CollTypeExampleStyle, right=of t300a, xshift=6mm] {(6, 127)\\2431 $ \notin \mu$\\3, 0, 0, 127};
\draw [dotted] (t300a) -- (t300b);
\node(t301a) [CollTypeExampleStyle, right=of t300b, xshift=2mm, yshift=-3mm] {(6, 128)\\2432 $ \notin \mu$\\\redBox{3}, 0, 1, 0};
\node(t301b) [CollTypeExampleStyle, right=of t301a, xshift=6mm] {(6, 255)\\2559 $ \notin \mu$\\3, 0, 1, 127};
\draw [dotted] (t301a) -- (t301b);
\node(t302a) [CollTypeExampleStyle, right=of t301b, xshift=2mm, yshift=-3mm] {(6, 256)\\2560 $ \notin \mu$\\3, 0, 2, 0};
\node(t302b) [CollTypeExampleStyle, right=of t302a, xshift=6mm] {(6, 383)\\2687 $ \notin \mu$\\3, 0, 2, 127};
\draw [dotted] (t302a) -- (t302b);
\node(t310a) [CollTypeExampleStyle, below=of t300a, yshift=-2mm] {(7, 0)\\2688 $ \notin \mu$\\3, 1, 0, 0};
\node(t310b) [CollTypeExampleStyle, right=of t310a, xshift=6mm] {(7, 127)\\2815 $ \notin \mu$\\3, 1, 0, 127};
\draw [dotted] (t310a) -- (t310b);
\node(t311a) [CollTypeExampleStyle, right=of t310b, xshift=2mm, yshift=-3mm] {(7, 128)\\2816 $ \notin \mu$\\3, 1, 1, 0};
\node(t311b) [CollTypeExampleStyle, right=of t311a, xshift=6mm] {(7, 255)\\2943 $ \notin \mu$\\3, 1, 1, 127};
\draw [dotted] (t311a) -- (t311b);
\node(t312a) [CollTypeExampleStyle, right=of t311b, xshift=2mm, yshift=-3mm] {(7, 256)\\2944 $ \notin \mu$\\3, 1, 2, 0};
\node(t312b) [CollTypeExampleStyle, right=of t312a, xshift=6mm] {(7, 383)\\3071 $ \notin \mu$\\3, 1, 2, 127};
\draw [dotted] (t312a) -- (t312b);

\draw[line, redstyle] (t100a.north west) -- ($(t100a.north west) - (14mm, 0mm)$);
\draw[line, redstyle] (t110a.north west) -- ($(t110a.north west) - (14mm, 0mm)$);
\draw[line, redstyle] ($(t110a.north west) - (14mm, 0mm)$) -- ($(t100a.north west) - (14mm, 0mm)$);
\node(dim0) [anchor=south west] at($(t100a.north west) - (14mm, 0mm)$) {\redBox{dim 0: $[1, 1]_\mathbb{N}, \delta.B_0=1$}};

\draw[line, yellowstyle] (t100a.south west) -- ($(t100a.south west) - (12mm, 0mm)$);
\draw[line, yellowstyle] (t110a.south west) -- ($(t110a.south west) - (12mm, 0mm)$);
\draw[line, yellowstyle] ($(t110a.south west) - (12mm, 0mm)$) -- ($(t100a.south west) - (12mm, 0mm)$);
\node(dim1) [anchor=north west] at($(t110a.south west) - (12mm, 0mm)$) {\yellowBox{dim 1: $[0, 1]_\mathbb{N}, \delta.B_1=2$}};

\draw[line, greenstyle] (t002a.north west) -- ($(t002a.north west) + (0mm, 15mm)$);
\draw[line, greenstyle] (t002b.north west) -- ($(t002b.north west) + (0mm, 15mm)$);
\draw[line, greenstyle] ($(t002a.north west) + (0mm, 15mm)$) -- ($(t002b.north west) + (0mm, 15mm)$);
\node(dim2) [anchor=south] at($(t002a.north west)!0.5!(t002b.north west)+(0mm, 15mm)$) {\greenBox{dim 2: $[2, 2]_\mathbb{N}, \delta.B_2=1$}};

\draw[line, bluestyle] (t002a.north east) -- ($(t002a.north east) + (0mm, 4mm)$);
\draw[line, bluestyle] (t002b.north east) -- ($(t002b.north east) + (0mm, 4mm)$);
\draw[line, bluestyle] ($(t002a.north east) + (0mm, 4mm)$) -- ($(t002b.north east) + (0mm, 4mm)$);
\node(dim3) [anchor=south east] at($(t002b.north east)+(0mm, 4mm)$) {\blueBox{dim 3: $[0, 127]_\mathbb{N}, \delta.B_3=128$}};

\draw[thick, <->, redstyle] ($(t001a.north west) - (3mm, 0mm)$) -- ($(t311a.south west) - (3mm, 0mm)$);
\node(D0) [anchor=center] at($(t001a.north west)!0.5!(t311a.south west) - (3mm, 0mm)$) {\redBox{$\delta.D_0=4$}};

\draw[thick, <->, yellowstyle] ($(t201a.north east)+(3mm, 0mm)$) -- ($(t211a.south east)+(3mm, 0mm)$);
\node(D1) [anchor=center] at($(t201a.north east)!0.5!(t211a.south east)+(3mm, 0mm)$) {\yellowBox{$\delta.D_1=2$}};

\draw[thick, <->, greenstyle] ($(t210a.south west)-(0mm, 2mm)$) -- ($(t212b.south east)-(0mm, 2mm)$);
\node(D2) [anchor=center] at($(t210a.south west)!0.6!(t212b.south east)-(0mm, 2mm)$) {\greenBox{$\delta.D_2=3$}};

\draw[thick, <->, bluestyle] ($(t202a.south west) + (2mm, -1.3mm)$) -- ($(t202b.south east) + (-2mm, -1.3mm)$);
\node(D3) [anchor=center] at($(t202a.south west)!0.5!(t202b.south east) + (0mm, -1.3mm)$) {\blueBox{$\delta.D_3=128$}};

\node(legend) [rectangle, draw=black, text centered, text width=5cm, above=of t000a, xshift=40mm, yshift=2mm] {(cluster\_ctarank, threadIdx.x)\\local thread index $\in \mu$\\coordinates};

\end{tikzpicture}
\caption{Example of a \violetBox{thread collective} that is an instance of the collective type $\delta$ with $\delta.B = (1, 2, 1, 128)$ and $\delta.D = (4, 2, 3, 128)$, which is a collective type in aligned form (def~\ref{sec:gAlignedForm}).
The domain implies an organization of the threads in the cluster into a 4D \redBox{4} $\times$ \yellowBox{2} $\times$ \greenBox{3} $\times$ \blueBox{128} space (which is tough to illustrate).
\lighttt{clusterDim = 8} and \lighttt{blockDim = 384}. $\mu = \{ \mathsf{toLocal}(\delta.D, c) \mid c \in [1, 1]_\mathbb{N} \times [0, 1]_\mathbb{N} \times [2, 2]_\mathbb{N} \times [0, 127]_\mathbb{N}\}$ (def~\ref{sec:gToLocal}).
This is one warpgroup per CTA of a CTA pair.}
\label{fig:CollTypeExample}
\end{figure}


\magicSubsection{Collective Type Reshape}{sec:CollTypeReshape}

Reshaping a collective type means to apply a series of dimension split operations to yield a new collective type.
We split the $k^{th}$ dimension of a collective type by a factor $f \in \mathbb{N}$ by
\begin{itemize}
  \item Inserting the coordinate pair $(\delta.D_k / f, f)$ in place of $\delta.D_k$.
  \item Inserting the coordinate pair $(\delta.B_k / f, f)$ in place of $\delta.B_k$, if $\delta.B_k \ne \top$ and $\delta.B_k \ge f$.
  \item Inserting the coordinate pair $(1, \delta.B_k)$ in place of $\delta.B_k$, if $\delta.B_k \ne \top$ and $\delta.B_k < f$.
  \item Inserting the coordinate pair $(\top, \top)$ in place of $\delta.B_k$, if $\delta.B_k = \top$.
\end{itemize}
e.g. if $\delta.D = (4, 384)$ and $\delta.B = (1, 128)$, then splitting dimension 1 by $32$ gives $\delta.D = (4, 12, 32)$ and $\delta.B = (1, 4, 32)$.


\magicSubsection{Collective Unit to Collective Type}{sec:CollUnit}

Collective units are also parameterized by a pair of $M$-tuples (domain and box), with coordinates being integer expressions of \lighttt{blockDim} and \lighttt{clusterDim}, or $\top$ in the case of the box.
This is represented in the frontend language as a Python \lighttt{CollUnit} object (the code consistently abbreviates ``collective'' as ``coll''), with $\top$ represented by \lighttt{None}.
We describe this further in def~\ref{sec:gCollUnit}.

\FloatBarrier
\newpage
\useMainSub

