\magicSubsection{Collective Tiling Figure}{sec:CollTilingFigure}

We will illustrate the collective tiling that annotates the inner-most statement of the following example proc.
The illustration is on a separate page.

\filbreak
\input{b_samples/for_CollTiling_figure.0.tex}

where we note that
\begin{itemize}
  \item The \texttt{\violetBox{n\_cta}} loop has a collective unit (def~\ref{sec:gCollUnit}) of \lighttt{4 * cuda\_cta\_in\_cluster\_strided(2)}, indicating that each iteration is executed cooperatively by a thread collective comprising 4 CTAs, with \lighttt{cluster\_ctarank} of the CTAs in the thread collective increasing by 2's.
  \item All \lighttt{unit} parameters are documented (def~\ref{sec:gCollUnit}).
  \item The \lighttt{CudaWarps} statement has the effect of deactivating the 0th warpgroup (def~\ref{sec:gWarpgroup}) of a CTA.
    The Exo-GPU compiler generates a hidden \lighttt{CudaWarps\_consumer\_None\_None} variable associated with this statement.
\end{itemize}

\filbreak
{
\sffamily
\begin{tikzpicture}[node distance=0mm]
\node(t_0_1_0) [CollTilingExampleStyle, fill=lightgray, anchor=north west, xshift=0.000000mm, yshift=0.000000mm] at(0, 0) {384\\$c_0$=0\\$c_1$=1\\$c_2$=0};
\node(t_0_1_1) [CollTilingExampleStyle, fill=lightgray, anchor=north west, xshift=14.000000mm, yshift=0.000000mm] at(t_0_1_0.north west) {385\\$c_0$=0\\$c_1$=1\\$c_2$=1};
\node(t_0_1_127) [CollTilingExampleStyle, fill=lightgray, anchor=north west, xshift=17.600000mm, yshift=0.000000mm] at(t_0_1_1.north west) {511\\$c_0$=0\\$c_1$=1\\$c_2$=127};
\draw[dotted, thick] (t_0_1_1.east) -- (t_0_1_127.west);
\node(t_0_1_128) [CollTilingExampleStyle, fill=lightgray, anchor=north west, xshift=14.000000mm, yshift=0.000000mm] at(t_0_1_127.north west) {512\\$c_0$=0\\$c_1$=1\\$c_2$=128};
\node(t_0_1_129) [CollTilingExampleStyle, fill=lightgray, anchor=north west, xshift=14.000000mm, yshift=0.000000mm] at(t_0_1_128.north west) {513\\$c_0$=0\\$c_1$=1\\$c_2$=129};
\node(t_0_1_255) [CollTilingExampleStyle, fill=lightgray, anchor=north west, xshift=17.600000mm, yshift=0.000000mm] at(t_0_1_129.north west) {639\\$c_0$=0\\$c_1$=1\\$c_2$=255};
\draw[dotted, thick] (t_0_1_129.east) -- (t_0_1_255.west);
\node(t_0_1_256) [CollTilingExampleStyle, fill=lightgray, anchor=north west, xshift=14.000000mm, yshift=0.000000mm] at(t_0_1_255.north west) {640\\$c_0$=0\\$c_1$=1\\$c_2$=256};
\node(t_0_1_257) [CollTilingExampleStyle, fill=lightgray, anchor=north west, xshift=14.000000mm, yshift=0.000000mm] at(t_0_1_256.north west) {641\\$c_0$=0\\$c_1$=1\\$c_2$=257};
\node(t_0_1_383) [CollTilingExampleStyle, fill=lightgray, anchor=north west, xshift=17.600000mm, yshift=0.000000mm] at(t_0_1_257.north west) {767\\$c_0$=0\\$c_1$=1\\$c_2$=383};
\draw[dotted, thick] (t_0_1_257.east) -- (t_0_1_383.west);
\node(cta1) [yellowstyle, anchor=center] at(t_0_1_0.north west) {1$\rightarrow$};
\node(t_1_1_0) [CollTilingExampleStyle, fill=lightgray, anchor=north west, xshift=0.000000mm, yshift=-24.000000mm] at(t_0_1_0.north west) {1152\\$c_0$=1\\$c_1$=1\\$c_2$=0};
\node(t_1_1_1) [CollTilingExampleStyle, fill=lightgray, anchor=north west, xshift=14.000000mm, yshift=0.000000mm] at(t_1_1_0.north west) {1153\\$c_0$=1\\$c_1$=1\\$c_2$=1};
\node(t_1_1_127) [CollTilingExampleStyle, fill=lightgray, anchor=north west, xshift=17.600000mm, yshift=0.000000mm] at(t_1_1_1.north west) {1279\\$c_0$=1\\$c_1$=1\\$c_2$=127};
\draw[dotted, thick] (t_1_1_1.east) -- (t_1_1_127.west);
\node(t_1_1_128) [CollTilingExampleStyle, fill=lightgray, anchor=north west, xshift=14.000000mm, yshift=0.000000mm] at(t_1_1_127.north west) {1280\\$c_0$=1\\$c_1$=1\\$c_2$=128};
\node(t_1_1_129) [CollTilingExampleStyle, fill=lightgray, anchor=north west, xshift=14.000000mm, yshift=0.000000mm] at(t_1_1_128.north west) {1281\\$c_0$=1\\$c_1$=1\\$c_2$=129};
\node(t_1_1_255) [CollTilingExampleStyle, fill=lightgray, anchor=north west, xshift=17.600000mm, yshift=0.000000mm] at(t_1_1_129.north west) {1407\\$c_0$=1\\$c_1$=1\\$c_2$=255};
\draw[dotted, thick] (t_1_1_129.east) -- (t_1_1_255.west);
\node(t_1_1_256) [CollTilingExampleStyle, fill=lightgray, anchor=north west, xshift=14.000000mm, yshift=0.000000mm] at(t_1_1_255.north west) {1408\\$c_0$=1\\$c_1$=1\\$c_2$=256};
\node(t_1_1_257) [CollTilingExampleStyle, fill=lightgray, anchor=north west, xshift=14.000000mm, yshift=0.000000mm] at(t_1_1_256.north west) {1409\\$c_0$=1\\$c_1$=1\\$c_2$=257};
\node(t_1_1_383) [CollTilingExampleStyle, fill=lightgray, anchor=north west, xshift=17.600000mm, yshift=0.000000mm] at(t_1_1_257.north west) {1535\\$c_0$=1\\$c_1$=1\\$c_2$=383};
\draw[dotted, thick] (t_1_1_257.east) -- (t_1_1_383.west);
\node(cta3) [yellowstyle, anchor=center] at(t_1_1_0.north west) {3$\rightarrow$};
\node(t_2_1_0) [CollTilingExampleStyle, fill=lightgray, anchor=north west, xshift=0.000000mm, yshift=-24.000000mm] at(t_1_1_0.north west) {1920\\$c_0$=2\\$c_1$=1\\$c_2$=0};
\node(t_2_1_1) [CollTilingExampleStyle, fill=lightgray, anchor=north west, xshift=14.000000mm, yshift=0.000000mm] at(t_2_1_0.north west) {1921\\$c_0$=2\\$c_1$=1\\$c_2$=1};
\node(t_2_1_127) [CollTilingExampleStyle, fill=lightgray, anchor=north west, xshift=17.600000mm, yshift=0.000000mm] at(t_2_1_1.north west) {2047\\$c_0$=2\\$c_1$=1\\$c_2$=127};
\draw[dotted, thick] (t_2_1_1.east) -- (t_2_1_127.west);
\node(t_2_1_128) [CollTilingExampleStyle, fill=lightgray, anchor=north west, xshift=14.000000mm, yshift=0.000000mm] at(t_2_1_127.north west) {2048\\$c_0$=2\\$c_1$=1\\$c_2$=128};
\node(t_2_1_129) [CollTilingExampleStyle, fill=lightgray, anchor=north west, xshift=14.000000mm, yshift=0.000000mm] at(t_2_1_128.north west) {2049\\$c_0$=2\\$c_1$=1\\$c_2$=129};
\node(t_2_1_255) [CollTilingExampleStyle, fill=lightgray, anchor=north west, xshift=17.600000mm, yshift=0.000000mm] at(t_2_1_129.north west) {2175\\$c_0$=2\\$c_1$=1\\$c_2$=255};
\draw[dotted, thick] (t_2_1_129.east) -- (t_2_1_255.west);
\node(t_2_1_256) [CollTilingExampleStyle, fill=lightgray, anchor=north west, xshift=14.000000mm, yshift=0.000000mm] at(t_2_1_255.north west) {2176\\$c_0$=2\\$c_1$=1\\$c_2$=256};
\node(t_2_1_257) [CollTilingExampleStyle, fill=lightgray, anchor=north west, xshift=14.000000mm, yshift=0.000000mm] at(t_2_1_256.north west) {2177\\$c_0$=2\\$c_1$=1\\$c_2$=257};
\node(t_2_1_383) [CollTilingExampleStyle, fill=lightgray, anchor=north west, xshift=17.600000mm, yshift=0.000000mm] at(t_2_1_257.north west) {2303\\$c_0$=2\\$c_1$=1\\$c_2$=383};
\draw[dotted, thick] (t_2_1_257.east) -- (t_2_1_383.west);
\node(cta5) [yellowstyle, anchor=center] at(t_2_1_0.north west) {5$\rightarrow$};
\node(t_3_1_0) [CollTilingExampleStyle, fill=lightgray, anchor=north west, xshift=0.000000mm, yshift=-24.000000mm] at(t_2_1_0.north west) {2688\\$c_0$=3\\$c_1$=1\\$c_2$=0};
\node(t_3_1_1) [CollTilingExampleStyle, fill=lightgray, anchor=north west, xshift=14.000000mm, yshift=0.000000mm] at(t_3_1_0.north west) {2689\\$c_0$=3\\$c_1$=1\\$c_2$=1};
\node(t_3_1_127) [CollTilingExampleStyle, fill=lightgray, anchor=north west, xshift=17.600000mm, yshift=0.000000mm] at(t_3_1_1.north west) {2815\\$c_0$=3\\$c_1$=1\\$c_2$=127};
\draw[dotted, thick] (t_3_1_1.east) -- (t_3_1_127.west);
\node(t_3_1_128) [CollTilingExampleStyle, fill=lightgray, anchor=north west, xshift=14.000000mm, yshift=0.000000mm] at(t_3_1_127.north west) {2816\\$c_0$=3\\$c_1$=1\\$c_2$=128};
\node(t_3_1_129) [CollTilingExampleStyle, fill=lightgray, anchor=north west, xshift=14.000000mm, yshift=0.000000mm] at(t_3_1_128.north west) {2817\\$c_0$=3\\$c_1$=1\\$c_2$=129};
\node(t_3_1_255) [CollTilingExampleStyle, fill=lightgray, anchor=north west, xshift=17.600000mm, yshift=0.000000mm] at(t_3_1_129.north west) {2943\\$c_0$=3\\$c_1$=1\\$c_2$=255};
\draw[dotted, thick] (t_3_1_129.east) -- (t_3_1_255.west);
\node(t_3_1_256) [CollTilingExampleStyle, fill=lightgray, anchor=north west, xshift=14.000000mm, yshift=0.000000mm] at(t_3_1_255.north west) {2944\\$c_0$=3\\$c_1$=1\\$c_2$=256};
\node(t_3_1_257) [CollTilingExampleStyle, fill=lightgray, anchor=north west, xshift=14.000000mm, yshift=0.000000mm] at(t_3_1_256.north west) {2945\\$c_0$=3\\$c_1$=1\\$c_2$=257};
\node(t_3_1_383) [CollTilingExampleStyle, fill=lightgray, anchor=north west, xshift=17.600000mm, yshift=0.000000mm] at(t_3_1_257.north west) {3071\\$c_0$=3\\$c_1$=1\\$c_2$=383};
\draw[dotted, thick] (t_3_1_257.east) -- (t_3_1_383.west);
\node(cta7) [yellowstyle, anchor=center] at(t_3_1_0.north west) {7$\rightarrow$};
\node(t_0_0_0) [CollTilingExampleStyle, fill=white, anchor=north west, xshift=12.000000mm, yshift=42.000000mm] at(t_0_1_0.north west) {0\\$c_0$=0\\$c_1$=0\\$c_2$=0};
\node(t_0_0_1) [CollTilingExampleStyle, fill=white, anchor=north west, xshift=14.000000mm, yshift=0.000000mm] at(t_0_0_0.north west) {1\\$c_0$=0\\$c_1$=0\\$c_2$=1};
\node(t_0_0_127) [CollTilingExampleStyle, fill=white, anchor=north west, xshift=17.600000mm, yshift=0.000000mm] at(t_0_0_1.north west) {127\\$c_0$=0\\$c_1$=0\\$c_2$=127};
\draw[dotted, thick] (t_0_0_1.east) -- (t_0_0_127.west);
\node(t_0_0_128) [CollTilingExampleStyle, fill=white, anchor=north west, xshift=14.000000mm, yshift=0.000000mm] at(t_0_0_127.north west) {128\\$c_0$=0\\$c_1$=0\\$c_2$=128};
\node(t_0_0_129) [CollTilingExampleStyle, fill=white, anchor=north west, xshift=14.000000mm, yshift=0.000000mm] at(t_0_0_128.north west) {129\\$c_0$=0\\$c_1$=0\\$c_2$=129};
\node(t_0_0_255) [CollTilingExampleStyle, fill=white, anchor=north west, xshift=17.600000mm, yshift=0.000000mm] at(t_0_0_129.north west) {255\\$c_0$=0\\$c_1$=0\\$c_2$=255};
\draw[dotted, thick] (t_0_0_129.east) -- (t_0_0_255.west);
\node(t_0_0_256) [CollTilingExampleStyle, fill=white, anchor=north west, xshift=14.000000mm, yshift=0.000000mm] at(t_0_0_255.north west) {256\\$c_0$=0\\$c_1$=0\\$c_2$=256};
\node(t_0_0_257) [CollTilingExampleStyle, fill=white, anchor=north west, xshift=14.000000mm, yshift=0.000000mm] at(t_0_0_256.north west) {257\\$c_0$=0\\$c_1$=0\\$c_2$=257};
\node(t_0_0_383) [CollTilingExampleStyle, fill=white, anchor=north west, xshift=17.600000mm, yshift=0.000000mm] at(t_0_0_257.north west) {383\\$c_0$=0\\$c_1$=0\\$c_2$=383};
\draw[dotted, thick] (t_0_0_257.east) -- (t_0_0_383.west);
\node(cta0) [yellowstyle, anchor=center] at(t_0_0_0.north west) {0$\rightarrow$};
\node(t_1_0_0) [CollTilingExampleStyle, fill=white, anchor=north west, xshift=0.000000mm, yshift=-24.000000mm] at(t_0_0_0.north west) {768\\$c_0$=1\\$c_1$=0\\$c_2$=0};
\node(t_1_0_1) [CollTilingExampleStyle, fill=white, anchor=north west, xshift=14.000000mm, yshift=0.000000mm] at(t_1_0_0.north west) {769\\$c_0$=1\\$c_1$=0\\$c_2$=1};
\node(t_1_0_127) [CollTilingExampleStyle, fill=white, anchor=north west, xshift=17.600000mm, yshift=0.000000mm] at(t_1_0_1.north west) {895\\$c_0$=1\\$c_1$=0\\$c_2$=127};
\draw[dotted, thick] (t_1_0_1.east) -- (t_1_0_127.west);
\node(t_1_0_128) [CollTilingExampleStyle, fill=white, anchor=north west, xshift=14.000000mm, yshift=0.000000mm] at(t_1_0_127.north west) {896\\$c_0$=1\\$c_1$=0\\$c_2$=128};
\node(t_1_0_129) [CollTilingExampleStyle, fill=white, anchor=north west, xshift=14.000000mm, yshift=0.000000mm] at(t_1_0_128.north west) {897\\$c_0$=1\\$c_1$=0\\$c_2$=129};
\node(t_1_0_255) [CollTilingExampleStyle, fill=white, anchor=north west, xshift=17.600000mm, yshift=0.000000mm] at(t_1_0_129.north west) {1023\\$c_0$=1\\$c_1$=0\\$c_2$=255};
\draw[dotted, thick] (t_1_0_129.east) -- (t_1_0_255.west);
\node(t_1_0_256) [CollTilingExampleStyle, fill=white, anchor=north west, xshift=14.000000mm, yshift=0.000000mm] at(t_1_0_255.north west) {1024\\$c_0$=1\\$c_1$=0\\$c_2$=256};
\node(t_1_0_257) [CollTilingExampleStyle, fill=white, anchor=north west, xshift=14.000000mm, yshift=0.000000mm] at(t_1_0_256.north west) {1025\\$c_0$=1\\$c_1$=0\\$c_2$=257};
\node(t_1_0_383) [CollTilingExampleStyle, fill=white, anchor=north west, xshift=17.600000mm, yshift=0.000000mm] at(t_1_0_257.north west) {1151\\$c_0$=1\\$c_1$=0\\$c_2$=383};
\draw[dotted, thick] (t_1_0_257.east) -- (t_1_0_383.west);
\node(cta2) [yellowstyle, anchor=center] at(t_1_0_0.north west) {2$\rightarrow$};
\node(t_2_0_0) [CollTilingExampleStyle, fill=white, anchor=north west, xshift=0.000000mm, yshift=-24.000000mm] at(t_1_0_0.north west) {1536\\$c_0$=2\\$c_1$=0\\$c_2$=0};
\node(t_2_0_1) [CollTilingExampleStyle, fill=white, anchor=north west, xshift=14.000000mm, yshift=0.000000mm] at(t_2_0_0.north west) {1537\\$c_0$=2\\$c_1$=0\\$c_2$=1};
\node(t_2_0_127) [CollTilingExampleStyle, fill=white, anchor=north west, xshift=17.600000mm, yshift=0.000000mm] at(t_2_0_1.north west) {1663\\$c_0$=2\\$c_1$=0\\$c_2$=127};
\draw[dotted, thick] (t_2_0_1.east) -- (t_2_0_127.west);
\node(t_2_0_128) [CollTilingExampleStyle, fill=white, anchor=north west, xshift=14.000000mm, yshift=0.000000mm] at(t_2_0_127.north west) {1664\\$c_0$=2\\$c_1$=0\\$c_2$=128};
\node(t_2_0_129) [CollTilingExampleStyle, fill=white, anchor=north west, xshift=14.000000mm, yshift=0.000000mm] at(t_2_0_128.north west) {1665\\$c_0$=2\\$c_1$=0\\$c_2$=129};
\node(t_2_0_255) [CollTilingExampleStyle, fill=white, anchor=north west, xshift=17.600000mm, yshift=0.000000mm] at(t_2_0_129.north west) {1791\\$c_0$=2\\$c_1$=0\\$c_2$=255};
\draw[dotted, thick] (t_2_0_129.east) -- (t_2_0_255.west);
\node(t_2_0_256) [CollTilingExampleStyle, fill=white, anchor=north west, xshift=14.000000mm, yshift=0.000000mm] at(t_2_0_255.north west) {1792\\$c_0$=2\\$c_1$=0\\$c_2$=256};
\node(t_2_0_257) [CollTilingExampleStyle, fill=white, anchor=north west, xshift=14.000000mm, yshift=0.000000mm] at(t_2_0_256.north west) {1793\\$c_0$=2\\$c_1$=0\\$c_2$=257};
\node(t_2_0_383) [CollTilingExampleStyle, fill=white, anchor=north west, xshift=17.600000mm, yshift=0.000000mm] at(t_2_0_257.north west) {1919\\$c_0$=2\\$c_1$=0\\$c_2$=383};
\draw[dotted, thick] (t_2_0_257.east) -- (t_2_0_383.west);
\node(cta4) [yellowstyle, anchor=center] at(t_2_0_0.north west) {4$\rightarrow$};
\node(t_3_0_0) [CollTilingExampleStyle, fill=white, anchor=north west, xshift=0.000000mm, yshift=-24.000000mm] at(t_2_0_0.north west) {2304\\$c_0$=3\\$c_1$=0\\$c_2$=0};
\node(t_3_0_1) [CollTilingExampleStyle, fill=white, anchor=north west, xshift=14.000000mm, yshift=0.000000mm] at(t_3_0_0.north west) {2305\\$c_0$=3\\$c_1$=0\\$c_2$=1};
\node(t_3_0_127) [CollTilingExampleStyle, fill=white, anchor=north west, xshift=17.600000mm, yshift=0.000000mm] at(t_3_0_1.north west) {2431\\$c_0$=3\\$c_1$=0\\$c_2$=127};
\draw[dotted, thick] (t_3_0_1.east) -- (t_3_0_127.west);
\node(t_3_0_128) [CollTilingExampleStyle, fill=white, anchor=north west, xshift=14.000000mm, yshift=0.000000mm] at(t_3_0_127.north west) {2432\\$c_0$=3\\$c_1$=0\\$c_2$=128};
\node(t_3_0_129) [CollTilingExampleStyle, fill=white, anchor=north west, xshift=14.000000mm, yshift=0.000000mm] at(t_3_0_128.north west) {2433\\$c_0$=3\\$c_1$=0\\$c_2$=129};
\node(t_3_0_255) [CollTilingExampleStyle, fill=white, anchor=north west, xshift=17.600000mm, yshift=0.000000mm] at(t_3_0_129.north west) {2559\\$c_0$=3\\$c_1$=0\\$c_2$=255};
\draw[dotted, thick] (t_3_0_129.east) -- (t_3_0_255.west);
\node(t_3_0_256) [CollTilingExampleStyle, fill=white, anchor=north west, xshift=14.000000mm, yshift=0.000000mm] at(t_3_0_255.north west) {2560\\$c_0$=3\\$c_1$=0\\$c_2$=256};
\node(t_3_0_257) [CollTilingExampleStyle, fill=white, anchor=north west, xshift=14.000000mm, yshift=0.000000mm] at(t_3_0_256.north west) {2561\\$c_0$=3\\$c_1$=0\\$c_2$=257};
\node(t_3_0_383) [CollTilingExampleStyle, fill=white, anchor=north west, xshift=17.600000mm, yshift=0.000000mm] at(t_3_0_257.north west) {2687\\$c_0$=3\\$c_1$=0\\$c_2$=383};
\draw[dotted, thick] (t_3_0_257.east) -- (t_3_0_383.west);
\node(cta6) [yellowstyle, anchor=center] at(t_3_0_0.north west) {6$\rightarrow$};
\draw[thick, greenstyle] ($(t_0_1_0.north west) + (0.000000mm, -7.800000mm)$) -- ($(t_0_1_0.north west) + (-7.500000mm, -7.800000mm)$);
\draw[thick, greenstyle] ($(t_3_1_0.north west) + (0.000000mm, -7.800000mm)$) -- ($(t_3_1_0.north west) + (-7.500000mm, -7.800000mm)$);
\draw[thick, <->, greenstyle] ($(t_0_1_0.north west) + (-3.750000mm, -7.800000mm)$) -- ($(t_3_1_0.north west) + (-3.750000mm, -7.800000mm)$);
\node(D0) [anchor=west] at($(t_0_1_0.north west)!0.5!(t_3_1_0.north west) + (-7.500000mm, -7.800000mm)$) {\greenBox{$\omega.D_0 = 4$}};
\draw[thick, violetstyle] ($(t_0_1_0.north west) + (0.000000mm, -13.000000mm)$) -- ($(t_0_1_0.north west) + (-7.500000mm, -13.000000mm)$);
\draw[thick, violetstyle] ($(t_0_0_0.north west) + (0.000000mm, -13.000000mm)$) -- ($(t_0_0_0.north west) + (-7.500000mm, -13.000000mm)$);
\draw[thick, <->, violetstyle] ($(t_0_1_0.north west) + (-3.750000mm, -13.000000mm)$) -- ($(t_0_0_0.north west) + (-3.750000mm, -13.000000mm)$);
\node(D1) [anchor=center] at($(t_0_1_0.north west)!0.5!(t_0_0_0.north west) + (-3.750000mm, -13.000000mm)$) {\violetBox{$\omega.D_1 = 2$}};
\draw[thick, bluestyle] ($(t_3_1_0.south) + (0mm, 0mm)$) -- ($(t_3_1_0.south) + (0mm, -7.5000000mm)$);
\draw[thick, bluestyle] ($(t_3_1_383.south) + (0mm, 0mm)$) -- ($(t_3_1_383.south) + (0mm, -7.5000000mm)$);
\draw[thick, <->, bluestyle] ($(t_3_1_0.south) + (0mm, -3.750000mm)$) -- ($(t_3_1_383.south) + (0mm, -3.750000mm)$);
\node(D2) [anchor=center] at($(t_3_1_0.south)!0.5!(t_3_1_383.south) + (0mm, -3.750000mm)$) {\blueBox{$\omega.D_2 = 384$}};
\draw[thick, violetstyle] ($(t_3_0_383.south east)!0.950000!(t_3_1_383.south east)$) -- ($(t_3_0_383.south east)!0.950000!(t_3_1_383.south east) + (7.500000mm, 0mm)$);
\draw[thick, violetstyle] ($(t_3_0_383.south east)!0.550000!(t_3_1_383.south east)$) -- ($(t_3_0_383.south east)!0.550000!(t_3_1_383.south east) + (7.500000mm, 0mm)$);
\draw[thick, violetstyle] ($(t_3_0_383.south east)!0.950000!(t_3_1_383.south east) + (7.500000mm, 0mm)$) -- ($(t_3_0_383.south east)!0.550000!(t_3_1_383.south east) + (7.500000mm, 0mm)$);
\node[anchor=center] at($(t_3_0_383.south east)!0.750000!(t_3_1_383.south east) + (7.500000mm, 0mm)$) {\violetBox{\texttt{n\_cta=1}}};
\draw[thick, violetstyle] ($(t_3_0_383.south east)!0.450000!(t_3_1_383.south east)$) -- ($(t_3_0_383.south east)!0.450000!(t_3_1_383.south east) + (7.500000mm, 0mm)$);
\draw[thick, violetstyle] ($(t_3_0_383.south east)!0.050000!(t_3_1_383.south east)$) -- ($(t_3_0_383.south east)!0.050000!(t_3_1_383.south east) + (7.500000mm, 0mm)$);
\draw[thick, violetstyle] ($(t_3_0_383.south east)!0.450000!(t_3_1_383.south east) + (7.500000mm, 0mm)$) -- ($(t_3_0_383.south east)!0.050000!(t_3_1_383.south east) + (7.500000mm, 0mm)$);
\node[anchor=center] at($(t_3_0_383.south east)!0.250000!(t_3_1_383.south east) + (7.500000mm, 0mm)$) {\violetBox{\texttt{n\_cta=0}}};
\draw[thick, greenstyle] ($(t_0_0_383.north east) + (1.875000mm, 0mm)$) -- ($(t_0_0_383.north east) + (9.375000mm, 0mm)$);
\draw[thick, greenstyle] ($(t_0_0_383.south east) + (1.875000mm, 0mm)$) -- ($(t_0_0_383.south east) + (9.375000mm, 0mm)$);
\draw[thick, greenstyle] ($(t_0_0_383.south east) + (9.375000mm, 0mm)$) -- ($(t_0_0_383.north east) + (9.375000mm, 0mm)$);
\node[anchor=center] at($(t_0_0_383.north east)!0.5!(t_0_0_383.south east) + (9.375000mm, 0mm)$) {\greenBox{\texttt{m\_cta=0}}};
\draw[thick, greenstyle] ($(t_1_0_383.north east) + (1.875000mm, 0mm)$) -- ($(t_1_0_383.north east) + (9.375000mm, 0mm)$);
\draw[thick, greenstyle] ($(t_1_0_383.south east) + (1.875000mm, 0mm)$) -- ($(t_1_0_383.south east) + (9.375000mm, 0mm)$);
\draw[thick, greenstyle] ($(t_1_0_383.south east) + (9.375000mm, 0mm)$) -- ($(t_1_0_383.north east) + (9.375000mm, 0mm)$);
\node[anchor=center] at($(t_1_0_383.north east)!0.5!(t_1_0_383.south east) + (9.375000mm, 0mm)$) {\greenBox{\texttt{m\_cta=1}}};
\draw[thick, greenstyle] ($(t_2_0_383.north east) + (1.875000mm, 0mm)$) -- ($(t_2_0_383.north east) + (9.375000mm, 0mm)$);
\draw[thick, greenstyle] ($(t_2_0_383.south east) + (1.875000mm, 0mm)$) -- ($(t_2_0_383.south east) + (9.375000mm, 0mm)$);
\draw[thick, greenstyle] ($(t_2_0_383.south east) + (9.375000mm, 0mm)$) -- ($(t_2_0_383.north east) + (9.375000mm, 0mm)$);
\node[anchor=center] at($(t_2_0_383.north east)!0.5!(t_2_0_383.south east) + (9.375000mm, 0mm)$) {\greenBox{\texttt{m\_cta=2}}};
\draw[thick, greenstyle] ($(t_3_0_383.north east) + (1.875000mm, 0mm)$) -- ($(t_3_0_383.north east) + (9.375000mm, 0mm)$);
\draw[thick, greenstyle] ($(t_3_0_383.south east) + (1.875000mm, 0mm)$) -- ($(t_3_0_383.south east) + (9.375000mm, 0mm)$);
\draw[thick, greenstyle] ($(t_3_0_383.south east) + (9.375000mm, 0mm)$) -- ($(t_3_0_383.north east) + (9.375000mm, 0mm)$);
\node[anchor=center] at($(t_3_0_383.north east)!0.5!(t_3_0_383.south east) + (9.375000mm, 0mm)$) {\greenBox{\texttt{m\_cta=3}}};
\draw[thick, bluestyle] ($(t_0_0_128.north west) + (1.750000mm, 20.625000mm)$) -- ($(t_0_0_128.north west) + (1.750000mm, 28.125000mm)$);
\draw[thick, bluestyle] ($(t_0_0_383.north east) + (-1.750000mm, 20.625000mm)$) -- ($(t_0_0_383.north east) + (-1.750000mm, 28.125000mm)$);
\draw[thick, bluestyle] ($(t_0_0_128.north west) + (1.750000mm, 28.125000mm)$) -- ($(t_0_0_383.north east) + (-1.750000mm, 28.125000mm)$);
\node[anchor=center] at($(t_0_0_128.north west)!0.5!(t_0_0_383.north east) + (0.000000mm, 28.125000mm)$) {\blueBox{\texttt{CudaWarps\_consumer\_None\_None=0}}};
\draw[thick, bluestyle] ($(t_0_0_128.north west) + (1.750000mm, 11.250000mm)$) -- ($(t_0_0_128.north west) + (1.750000mm, 18.750000mm)$);
\draw[thick, bluestyle] ($(t_0_0_255.north east) + (-1.750000mm, 11.250000mm)$) -- ($(t_0_0_255.north east) + (-1.750000mm, 18.750000mm)$);
\draw[thick, bluestyle] ($(t_0_0_128.north west) + (1.750000mm, 18.750000mm)$) -- ($(t_0_0_255.north east) + (-1.750000mm, 18.750000mm)$);
\node[anchor=center] at($(t_0_0_128.north west)!0.5!(t_0_0_255.north east) + (0.000000mm, 18.750000mm)$) {\blueBox{\texttt{wg=0}}};
\draw[thick, bluestyle] ($(t_0_0_128.north west) + (1.750000mm, 1.875000mm)$) -- ($(t_0_0_128.north west) + (1.750000mm, 9.375000mm)$);
\draw[thick, bluestyle] ($(t_0_0_128.north east) + (-1.750000mm, 1.875000mm)$) -- ($(t_0_0_128.north east) + (-1.750000mm, 9.375000mm)$);
\draw[thick, bluestyle] ($(t_0_0_128.north west) + (1.750000mm, 9.375000mm)$) -- ($(t_0_0_128.north east) + (-1.750000mm, 9.375000mm)$);
\node[anchor=center] at($(t_0_0_128.north west)!0.5!(t_0_0_128.north east) + (0.000000mm, 9.375000mm)$) {\blueBox{\texttt{t=0}}};
\draw[thick, bluestyle] ($(t_0_0_129.north west) + (1.750000mm, 1.875000mm)$) -- ($(t_0_0_129.north west) + (1.750000mm, 9.375000mm)$);
\draw[thick, bluestyle] ($(t_0_0_129.north east) + (-1.750000mm, 1.875000mm)$) -- ($(t_0_0_129.north east) + (-1.750000mm, 9.375000mm)$);
\draw[thick, bluestyle] ($(t_0_0_129.north west) + (1.750000mm, 9.375000mm)$) -- ($(t_0_0_129.north east) + (-1.750000mm, 9.375000mm)$);
\node[anchor=center] at($(t_0_0_129.north west)!0.5!(t_0_0_129.north east) + (0.000000mm, 9.375000mm)$) {\blueBox{\texttt{t=1}}};
\draw[thick, bluestyle] ($(t_0_0_255.north west) + (1.750000mm, 1.875000mm)$) -- ($(t_0_0_255.north west) + (1.750000mm, 9.375000mm)$);
\draw[thick, bluestyle] ($(t_0_0_255.north east) + (-1.750000mm, 1.875000mm)$) -- ($(t_0_0_255.north east) + (-1.750000mm, 9.375000mm)$);
\draw[thick, bluestyle] ($(t_0_0_255.north west) + (1.750000mm, 9.375000mm)$) -- ($(t_0_0_255.north east) + (-1.750000mm, 9.375000mm)$);
\node[anchor=center] at($(t_0_0_255.north west)!0.5!(t_0_0_255.north east) + (0.000000mm, 9.375000mm)$) {\blueBox{\texttt{t=127}}};
\draw[thick, bluestyle] ($(t_0_0_256.north west) + (1.750000mm, 11.250000mm)$) -- ($(t_0_0_256.north west) + (1.750000mm, 18.750000mm)$);
\draw[thick, bluestyle] ($(t_0_0_383.north east) + (-1.750000mm, 11.250000mm)$) -- ($(t_0_0_383.north east) + (-1.750000mm, 18.750000mm)$);
\draw[thick, bluestyle] ($(t_0_0_256.north west) + (1.750000mm, 18.750000mm)$) -- ($(t_0_0_383.north east) + (-1.750000mm, 18.750000mm)$);
\node[anchor=center] at($(t_0_0_256.north west)!0.5!(t_0_0_383.north east) + (0.000000mm, 18.750000mm)$) {\blueBox{\texttt{wg=1}}};
\draw[thick, bluestyle] ($(t_0_0_256.north west) + (1.750000mm, 1.875000mm)$) -- ($(t_0_0_256.north west) + (1.750000mm, 9.375000mm)$);
\draw[thick, bluestyle] ($(t_0_0_256.north east) + (-1.750000mm, 1.875000mm)$) -- ($(t_0_0_256.north east) + (-1.750000mm, 9.375000mm)$);
\draw[thick, bluestyle] ($(t_0_0_256.north west) + (1.750000mm, 9.375000mm)$) -- ($(t_0_0_256.north east) + (-1.750000mm, 9.375000mm)$);
\node[anchor=center] at($(t_0_0_256.north west)!0.5!(t_0_0_256.north east) + (0.000000mm, 9.375000mm)$) {\blueBox{\texttt{t=0}}};
\draw[thick, bluestyle] ($(t_0_0_257.north west) + (1.750000mm, 1.875000mm)$) -- ($(t_0_0_257.north west) + (1.750000mm, 9.375000mm)$);
\draw[thick, bluestyle] ($(t_0_0_257.north east) + (-1.750000mm, 1.875000mm)$) -- ($(t_0_0_257.north east) + (-1.750000mm, 9.375000mm)$);
\draw[thick, bluestyle] ($(t_0_0_257.north west) + (1.750000mm, 9.375000mm)$) -- ($(t_0_0_257.north east) + (-1.750000mm, 9.375000mm)$);
\node[anchor=center] at($(t_0_0_257.north west)!0.5!(t_0_0_257.north east) + (0.000000mm, 9.375000mm)$) {\blueBox{\texttt{t=1}}};
\draw[thick, bluestyle] ($(t_0_0_383.north west) + (1.750000mm, 1.875000mm)$) -- ($(t_0_0_383.north west) + (1.750000mm, 9.375000mm)$);
\draw[thick, bluestyle] ($(t_0_0_383.north east) + (-1.750000mm, 1.875000mm)$) -- ($(t_0_0_383.north east) + (-1.750000mm, 9.375000mm)$);
\draw[thick, bluestyle] ($(t_0_0_383.north west) + (1.750000mm, 9.375000mm)$) -- ($(t_0_0_383.north east) + (-1.750000mm, 9.375000mm)$);
\node[anchor=center] at($(t_0_0_383.north west)!0.5!(t_0_0_383.north east) + (0.000000mm, 9.375000mm)$) {\blueBox{\texttt{t=127}}};


\node(keyText) [anchor=south west, yshift=60mm, xshift=-5mm] at(cta1.north west) {\textbf{KEY:}};
\node(keyThread) [CollTilingExampleStyle, anchor=west] at(keyText.east) {tid\\$c_0$\\$c_1$\\$c_2$};
\node(keyTid) [anchor=west, text width=40mm] at(keyThread.east) {where ``tid'' is the local thread index given by toLocal($\omega.D, (c_0, c_1, c_2)$) (def~\ref{sec:gLocalThreadIndex},~def~\ref{sec:gToLocal}).};
\node(keyCta) [anchor=north west, yellowstyle, yshift=-2mm] at(keyThread.south west) {\texttt{cluster\_ctarank} (def~\ref{sec:gCluster})$\rightarrow$};

\node(ops_2_2_value) [anchor=south east, bluestyle, text width=80mm, yshift=+35mm] at(t_0_0_383.north east)  {\rmfamily $\textit{offset}=0$, $\textit{box}=1$, $\textit{tileCount}=128, \textit{iter}=\texttt{t}$};
\node(ops_2_1_value) [anchor=south east, bluestyle, text width=80mm, yshift=+1mm] at(ops_2_2_value.north east)  {\rmfamily $\textit{offset}=0$, $\textit{box}=128$, $\textit{tileCount}=2, \textit{iter}=\texttt{wg}$};
\node(ops_2_0_value) [anchor=south east, bluestyle, text width=80mm, yshift=+1mm] at(ops_2_1_value.north east)  {\rmfamily $\textit{offset}=128$, $\textit{box}=256$, $\textit{tileCount}=1$,\\$\textit{iter}=\texttt{CudaWarps\_consumer\_None\_None}$};
\node(ops_1_0_value) [anchor=south east, violetstyle, text width=80mm, yshift=+1mm] at(ops_2_0_value.north east) {\rmfamily $\textit{offset}=0$, $\textit{box}=1$, $\textit{tileCount}=2$, $\textit{iter}=\texttt{n\_cta}$};
\node(ops_0_0_value) [anchor=south east, greenstyle, text width=80mm, yshift=+1mm] at(ops_1_0_value.north east) {\rmfamily $\textit{offset}=0$, $\textit{box}=1$, $\textit{tileCount}=4$, $\textit{iter}=\texttt{m\_cta}$};

\node(ops_2_2_label) [anchor=east] at(ops_2_2_value.west) {\rmfamily $\omega_2.\textit{ops}_2$};
\node(ops_2_1_label) [anchor=east] at(ops_2_1_value.west) {\rmfamily $\omega_2.\textit{ops}_1$};
\node(ops_2_0_label) [anchor=east] at(ops_2_0_value.west) {\rmfamily $\omega_2.\textit{ops}_0$};
\node(ops_1_0_label) [anchor=east] at(ops_1_0_value.west) {\rmfamily $\omega_1.\textit{ops}_0$};
\node(ops_0_0_label) [anchor=east] at(ops_0_0_value.west) {\rmfamily $\omega_0.\textit{ops}_0$};

\node(dim2) [anchor=north west, xshift=-50mm] at(ops_2_0_value.north west) {\textbf{Dim 2:} $\omega_2.n = 384$};
\node(dim1) [anchor=north west, xshift=-50mm] at(ops_1_0_value.north west) {\textbf{Dim 1:} $\omega_1.n = 2$};
\node(dim0) [anchor=north west, xshift=-50mm] at(ops_0_0_value.north west) {\textbf{Dim 0:} $\omega_0.n = 4$};
\node(CollTiling) [draw=black, anchor=east, text width=20mm] at(dim1.west) {CollTiling\\$\omega: \Omega$ state};

\node(domain) [anchor=north east, yshift=-2mm, text width=50mm] at(dim2.south east) {The domain $\omega.D = (4, 2, 384)$ is given by ($\omega_0.n$, $\omega_1.n$, $\omega_2.n$) (def~\ref{sec:gDomain}).};

\end{tikzpicture}
}

\FloatBarrier
\newpage
