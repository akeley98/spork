\magicSection{Collective Tiling}{sec:CollTiling}

The \lighttt{cuda\_tasks} loops assign instances of device tasks (def~\ref{sec:gDeviceTask}) to different clusters on the system, and the user has no control (yet) over the mapping between device tasks and clusters.
On the other hand, the \lighttt{cuda\_threads} loop, which assigns work to threads within a cluster, provides the user with tight control over this work mapping.

Each statement at CUDA scope (def~\ref{sec:gCudaScope}) has a deduced \myKeyA{collective tiling} attribute.
The collective tiling describes an arrangement of the threads in the cluster into a multidimensional space, in the same manner as a collective type's domain (def~\ref{sec:gDomain}), and groups \lighttt{cuda\_threads} loop iterators by the dimension they operate on.
The statement's collective tiling may be converted to
\begin{itemize}
  \item an \myKeyA{output collective type}.
    The thread collective assigned to execute an instance of this statement will always be an instance of this collective type (Section~\ref{sec:CollTypeThreadCollective}).
  \item a \myKeyA{thread mapping function} of type $\Sigma \to \mathcal{P}(\mathbb{N})$; this converts the control environment $\sigma:\Sigma$ (i.e. $\sigma: \mathbb{Y} \to \mathbb{Z}$; the per-control-variable values) to the local thread indices (def~\ref{sec:gLocalThreadIndex}) of the thread collective assigned to execute an instance of this statement.
  NB this is using syntax from the PLDI submission.
  \item a per-\lighttt{cuda\_threads} iterator \myKeyA{thread pitch}: $\mathbb{N}$ (def~\ref{sec:gThreadPitch}).
  \item a per-\lighttt{cuda\_threads} iterator \myKeyA{tiled dimension index}: $\mathbb{N}_\bot$.
\end{itemize}
The \myKeyA{domain} and \myKeyA{box} of a collective tiling $\omega$ are that of its output collective type, and for brevity, we denote it as $\omega.D$ and $\omega.B$ respectively.

\magicSubsection{Collective Dimension Descriptor}{sec:CollDimDescriptor}

A collective tiling $\omega: \Omega$ (of some dimensionality $M$) is an $M$-tuple of \myKeyA{collective dimension descriptors} $(\omega_0 ... \omega_{M-1})$.
Each descriptor $\omega_i$ consists of
\begin{itemize}
  \item $\omega_i.n: \mathbb{N}$, dimension extent
  \item $\omega_i.\textit{ops}: \mathcal{O}^*$, tuple of \myKeyA{collective dimension operators}
\end{itemize}
where $\mathcal{O}$ is $\mathbb{Y} \times \mathbb{N}^3$, with each attribute denoted as
\begin{itemize}
  \item $\omega_i.\textit{ops}_j.\textit{iter}: \mathbb{Y}$ (name of a control variable; no two ops in $\omega$ can have the same \textit{iter})
  \item $\omega_i.\textit{ops}_j.\textit{offset}: \mathbb{N}$
  \item $\omega_i.\textit{ops}_j.\textit{box}: \mathbb{N}$
  \item $\omega_i.\textit{ops}_j.\textit{tileCount}: \mathbb{N}$
\end{itemize}


\magicSubsection{Collective Tiling Derived State}{sec:CollTilingDerivedState}

The \myKeyA{output collective type} of $\omega$ is a collective type $\delta = (\omega.D, \omega.B)$ defined by
\begin{itemize}
  \item $\omega.D_i = \omega_i.n$
  \item $\omega.B_i$ is $\omega_i.n$ if $\omega_i.\textit{ops}$ is empty, otherwise it is the $\textit{box}$ of the last op.
  \item (recall dimension thread pitch values $\omega.P_i$ are implicit, as in def~\ref{sec:gThreadPitch}).
\end{itemize}
The \myKeyA{thread mapping function} selects, for each collective dimension descriptor $\omega_i$, an interval of size $\omega.B_i$ based on an affine transform of the values of the iterators associated with that dimension.
These intervals together select a sub-grid of the $M$-dimensional space of threads-in-cluster.

We derive the thread mapping from $\omega$ based on the function
\begin{align*}
    & \textsf{collMap}: \Omega \to \Sigma \to \mathcal{P}(\mathbb{N}) \\
    & (\omega_0, ..., \omega_{M-1}) \mapsto \sigma \mapsto \{ \textsf{toLocal}(\omega.D, c) \mid
      c \in [x_0, x_0 + \omega.B_0 - 1]_\mathbb{N} \times ... \times
            [x_{M-1}, x_{M-1} + \omega.B_{M-1} - 1]_\mathbb{N} \} \\
    & \text{ with } x_i = \sum_{\textit{op} \in \omega_i.\textit{ops}} \textit{op}.\textit{offset} + \sigma(\textit{op}.\textit{iter}) \textit{op}.\textit{box}
\end{align*}
where \textsf{toLocal} is def~\ref{sec:gToLocal} and the actual thread mapping function is $\textsf{collMap}(\omega)$ (partial evaluation).

For a given $y: \mathbb{Y}$, let $\textit{op}_y = \omega_i.\textit{ops}_j$ such that $y = \textit{op}_y.\textit{iter}$, if it exists.
The \myKeyA{thread pitch} and \myKeyA{tiled dimension index} of $y: \mathbb{Y}$ as defined by $\omega$ is
\begin{itemize}
  \item 0 and $i$, if $\textit{op}_y$ exists and $\textit{op}_y.\textit{tileCount} \le 1$
  \item $(\omega.P_i)(\textit{op}_y.\textit{box})$ and $i$, if $\textit{op}_y$ exists and $\textit{op}_y.\textit{tileCount} > 1$
  \item 0 and $\bot$ if $\textit{op}_y$ does not exist.
\end{itemize}


\magicSubsection{Collective Tiling Reshape \& Domain Completion}{sec:CollTilingReshape}

Similar to collective types (Section~\ref{sec:CollTypeReshape}), collective tilings may be reshaped by splitting dimensions.
We split the $k^{th}$ dimension of $\omega$ by a factor $f: \mathbb{N}$ by replacing the collective dimension descriptor $\omega_k$ with a pair $(\omega_\text{hi}, \omega_\text{lo})$ defined by
\begin{itemize}
  \item $\omega_\text{hi}.n = \omega_k.n / f$
  \item $\omega_\text{hi}.\textit{ops}$ is all $\textit{op} \in \omega_k.\textit{ops}$ with $\textit{op}.\textit{box} \ge f$, modified by dividing both $\textit{op}.\textit{box}$ and $\textit{op}.\textit{offset}$ by $f$. This fails if any division gives a non-integer.
  \item $\omega_\text{lo}.n = f$
  \item $\omega_\text{lo}.\textit{ops}$ is all $\textit{op} \in \omega_k.\textit{ops}$ with $\textit{op}.\textit{box} < f$, unmodified.
\end{itemize}

This mirrors Section~\ref{sec:CollTypeReshape}, in that this results in the single domain coordinate $\omega.D_k$ being replaced with $\omega.D_k / f$ and $f$.
We use the function $\textsf{domainCompletion}: \Omega \times \Delta \to \Omega \times \Delta$ (def~\ref{sec:gDomainCompletion}) to reshape a collective tiling and a collective type so that they have the same domain.


\magicSubsection{Collective Type Matching}{sec:CollTilingCollTypeMatching}

Given a collective type $\delta_0$ and collective tiling $\omega_0$, we can query if $\omega_0$ \myKeyA{matches} $\delta_0$.
Let $(\omega, \delta) = \mathsf{domainCompletion}(\omega_0, \delta_0)$.
Then the query returns true if domain completion succeeded and $\omega.B = \delta.B$.
This certifies that statements with collective tiling $\omega_0$ are at $\delta_0$-scope (def~\ref{sec:gCollType}).


\magicSubsection{Derived Collective Tilings}{sec:DerivedCollTiling}

The collective tiling of each statement is the same as that of its parent, except that \lighttt{with CudaDeviceFunction}, \lighttt{for cuda\_threads}, and \lighttt{with CudaWarps} assign a new collective tiling for their children.
The latter two are defined based on the \textsf{deriveCollTiling} function (def~\ref{sec:gDeriveCollTiling}); here we just summarize the collective tilings.


\magicSubsection{CudaDeviceFunction}{sec:CollTilingCudaDeviceFunction}

If \lighttt{clusterDim = 1}, then $\omega$ is one-dimensional, with $\omega_0.n = \lighttt{blockDim}$.
Otherwise, $\omega$ is two-dimensional, with $\omega_0.n = \lighttt{clusterDim}$ and $\omega_1.n = \lighttt{blockDim}$.
In both cases, $\omega_*.\textit{ops}$ is empty.


\magicSubsection{cuda\_threads Loops}{sec:CollTilingCudaThreads}

A \lighttt{cuda\_threads} loop must be in the form \lighttt{for $y$ in cuda\_threads(0, $c_\text{hi}$, unit=$\tau_u$)}, with $c_\text{hi}$ a positive constant integer.
Let $\omega_\text{raw}$ be the collective tiling of the loop statement, and let $\delta_\text{raw}$ be the collective type unpacked from $\tau_u$ without alignment and without 1-padding (def~\ref{sec:gCollUnit}).
Let $(\omega, \delta) = \textsf{domainCompletion}(\omega_\text{raw}, \delta_\text{raw})$ (def~\ref{sec:gDomainCompletion}) so that $\omega$ and $\delta$ have the same domain.

The tiled dimension index $k$ is the value such that $\delta.B_k \notin \{ \top, \omega.B_k, \omega.D_k \}$.
If no such $k$ exists, then the collective tiling of the child statements is $\omega$, and $c_\text{hi}$ must be 1 (trivial tiling).

If multiple such $k$ exist, then the loop is ill-formed (ambiguous tiling).

If $k$ exists uniquely, then we must have $c_\text{hi} \delta.B_k \le \omega.B_k$, otherwise the loop is ill-formed (not enough threads).
In this case, the collective tiling $\omega'$ of the child statements is like $\omega$, but with a new $\textit{op}: \mathcal{O}$ appended to $\omega'_k.\textit{ops}$, with
\begin{itemize}
  \item $\textit{op}.\textit{iter} = y$
  \item $\textit{op}.\textit{offset} = 0$
  \item $\textit{op}.\textit{box} = \delta.B_k$
  \item $\textit{op}.\textit{tileCount} = c_\text{hi}$
\end{itemize}
so that the output collective type of $\omega'$ is the same as that of $\omega$, except that $\omega'.B_k = \delta.B_k$.
If all other box coordinates of $\omega'$ already match those of $\delta$, then the output collective type of $\omega'$ is $\delta$.
This is commonly the case, but we designed this to allow for mismatches on dimensions other than $k$ to make it easier to place CTA-in-cluster loops inside thread-in-CTA loops or \lighttt{with CudaWarps} blocks.

The generated collective tiling is more precisely defined by (def~\ref{sec:gDeriveCollTiling})
\begin{equation*}
    \mathsf{deriveCollTiling}(\omega_\text{raw}, y, \delta_\text{raw}, 0, c_\text{hi}, c_\text{hi})
\end{equation*}


\magicSubsection{CudaWarps}{sec:CollTilingCudaWarps}

% Pants-on-fire simplified explanation for the end user.
We describe this somewhat informally in terms of the \lighttt{cuda\_threads} loop behavior.
A \lighttt{with CudaWarps(lo, hi, name=...)} statement has the following defaults:
\begin{itemize}
  \item \lighttt{lo = 0} if not given.
  \item \lighttt{hi}, if not given, is the number of warps of the warp variable named.
  \item \lighttt{name} is \lighttt{""} if this is a top-level case (see below), or the same as the parent \lighttt{with CudaWarps} statement if this is a nested case.
\end{itemize}

\lighttt{with CudaWarps} statements that appear in CUDA device functions with at least two warp variables and with no other \lighttt{with CudaWarps} as a direct or indirect parent are a top-level case.
A top-level case must appear in \lighttt{cuda\_agnostic\_intact\_cta}-scope (def~\ref{sec:gCollUnit}).
The \lighttt{true\_lo} and \lighttt{true\_hi} of the statement are \lighttt{lo + p} and \lighttt{hi + p}, \lighttt{p} being the prefix of the warp variable (def~\ref{sec:gWarpVariable}).
All other cases are nested cases, which have \lighttt{true\_lo} and \lighttt{true\_hi} being the same as \lighttt{lo} and \lighttt{hi}.

The \lighttt{with CudaWarps} statement defines the collective tiling $\omega'$ of its child statements in a similar manner as \lighttt{for \_ in cuda\_threads(0, true\_hi, unit=cuda\_warp)}, except that the threads that would have executed iterations \lighttt{true\_lo} through \lighttt{true\_hi - 1} instead cooperate to execute the statement body.
The \lighttt{with CudaWarps} statement defines a dummy iterator variable $y$ and adds a new collective dimension operator to $\omega'$ for $y$ (except in case of a trivial tiling).

The generated collective tiling is more precisely defined by (def~\ref{sec:gDeriveCollTiling})
\begin{equation*}
    \mathsf{deriveCollTiling}(\omega_\text{raw}, y, \delta_\text{warp}, \texttt{true\_lo}, \texttt{true\_hi}, 1)
\end{equation*}
where $\delta_\text{warp}$ is unpacked from \lighttt{cuda\_warp} with alignment and 1-padding (def~\ref{sec:gCollUnit}).


\magicSubsection{Collective Tiling Figure}{sec:CollTilingFigure}

We will illustrate the collective tiling that annotates the inner-most statement of the following example proc.
The illustration is on a separate page.

\filbreak
\input{b_samples/for_CollTiling_figure.0.tex}

where we note that
\begin{itemize}
  \item The \texttt{\violetBox{n\_cta}} loop has a collective unit (def~\ref{sec:gCollUnit}) of \lighttt{4 * cuda\_cta\_in\_cluster\_strided(2)}, indicating that each iteration is executed cooperatively by a thread collective comprising 4 CTAs, with \lighttt{cluster\_ctarank} of the CTAs in the thread collective increasing by 2's.
  \item All \lighttt{unit} parameters are documented (def~\ref{sec:gCollUnit}).
  \item The \lighttt{CudaWarps} statement has the effect of deactivating the 0th warpgroup (def~\ref{sec:gWarpgroup}) of a CTA.
    The Exo-GPU compiler generates a hidden \lighttt{CudaWarps\_consumer\_None\_None} variable associated with this statement.
\end{itemize}

\filbreak
{
\sffamily
\begin{tikzpicture}[node distance=0mm]
\node(t_0_1_0) [CollTilingExampleStyle, fill=lightgray, anchor=north west, xshift=0.000000mm, yshift=0.000000mm] at(0, 0) {384\\$c_0$=0\\$c_1$=1\\$c_2$=0};
\node(t_0_1_1) [CollTilingExampleStyle, fill=lightgray, anchor=north west, xshift=14.000000mm, yshift=0.000000mm] at(t_0_1_0.north west) {385\\$c_0$=0\\$c_1$=1\\$c_2$=1};
\node(t_0_1_127) [CollTilingExampleStyle, fill=lightgray, anchor=north west, xshift=17.600000mm, yshift=0.000000mm] at(t_0_1_1.north west) {511\\$c_0$=0\\$c_1$=1\\$c_2$=127};
\draw[dotted, thick] (t_0_1_1.east) -- (t_0_1_127.west);
\node(t_0_1_128) [CollTilingExampleStyle, fill=lightgray, anchor=north west, xshift=14.000000mm, yshift=0.000000mm] at(t_0_1_127.north west) {512\\$c_0$=0\\$c_1$=1\\$c_2$=128};
\node(t_0_1_129) [CollTilingExampleStyle, fill=lightgray, anchor=north west, xshift=14.000000mm, yshift=0.000000mm] at(t_0_1_128.north west) {513\\$c_0$=0\\$c_1$=1\\$c_2$=129};
\node(t_0_1_255) [CollTilingExampleStyle, fill=lightgray, anchor=north west, xshift=17.600000mm, yshift=0.000000mm] at(t_0_1_129.north west) {639\\$c_0$=0\\$c_1$=1\\$c_2$=255};
\draw[dotted, thick] (t_0_1_129.east) -- (t_0_1_255.west);
\node(t_0_1_256) [CollTilingExampleStyle, fill=lightgray, anchor=north west, xshift=14.000000mm, yshift=0.000000mm] at(t_0_1_255.north west) {640\\$c_0$=0\\$c_1$=1\\$c_2$=256};
\node(t_0_1_257) [CollTilingExampleStyle, fill=lightgray, anchor=north west, xshift=14.000000mm, yshift=0.000000mm] at(t_0_1_256.north west) {641\\$c_0$=0\\$c_1$=1\\$c_2$=257};
\node(t_0_1_383) [CollTilingExampleStyle, fill=lightgray, anchor=north west, xshift=17.600000mm, yshift=0.000000mm] at(t_0_1_257.north west) {767\\$c_0$=0\\$c_1$=1\\$c_2$=383};
\draw[dotted, thick] (t_0_1_257.east) -- (t_0_1_383.west);
\node(cta1) [yellowstyle, anchor=center] at(t_0_1_0.north west) {1$\rightarrow$};
\node(t_1_1_0) [CollTilingExampleStyle, fill=lightgray, anchor=north west, xshift=0.000000mm, yshift=-24.000000mm] at(t_0_1_0.north west) {1152\\$c_0$=1\\$c_1$=1\\$c_2$=0};
\node(t_1_1_1) [CollTilingExampleStyle, fill=lightgray, anchor=north west, xshift=14.000000mm, yshift=0.000000mm] at(t_1_1_0.north west) {1153\\$c_0$=1\\$c_1$=1\\$c_2$=1};
\node(t_1_1_127) [CollTilingExampleStyle, fill=lightgray, anchor=north west, xshift=17.600000mm, yshift=0.000000mm] at(t_1_1_1.north west) {1279\\$c_0$=1\\$c_1$=1\\$c_2$=127};
\draw[dotted, thick] (t_1_1_1.east) -- (t_1_1_127.west);
\node(t_1_1_128) [CollTilingExampleStyle, fill=lightgray, anchor=north west, xshift=14.000000mm, yshift=0.000000mm] at(t_1_1_127.north west) {1280\\$c_0$=1\\$c_1$=1\\$c_2$=128};
\node(t_1_1_129) [CollTilingExampleStyle, fill=lightgray, anchor=north west, xshift=14.000000mm, yshift=0.000000mm] at(t_1_1_128.north west) {1281\\$c_0$=1\\$c_1$=1\\$c_2$=129};
\node(t_1_1_255) [CollTilingExampleStyle, fill=lightgray, anchor=north west, xshift=17.600000mm, yshift=0.000000mm] at(t_1_1_129.north west) {1407\\$c_0$=1\\$c_1$=1\\$c_2$=255};
\draw[dotted, thick] (t_1_1_129.east) -- (t_1_1_255.west);
\node(t_1_1_256) [CollTilingExampleStyle, fill=lightgray, anchor=north west, xshift=14.000000mm, yshift=0.000000mm] at(t_1_1_255.north west) {1408\\$c_0$=1\\$c_1$=1\\$c_2$=256};
\node(t_1_1_257) [CollTilingExampleStyle, fill=lightgray, anchor=north west, xshift=14.000000mm, yshift=0.000000mm] at(t_1_1_256.north west) {1409\\$c_0$=1\\$c_1$=1\\$c_2$=257};
\node(t_1_1_383) [CollTilingExampleStyle, fill=lightgray, anchor=north west, xshift=17.600000mm, yshift=0.000000mm] at(t_1_1_257.north west) {1535\\$c_0$=1\\$c_1$=1\\$c_2$=383};
\draw[dotted, thick] (t_1_1_257.east) -- (t_1_1_383.west);
\node(cta3) [yellowstyle, anchor=center] at(t_1_1_0.north west) {3$\rightarrow$};
\node(t_2_1_0) [CollTilingExampleStyle, fill=lightgray, anchor=north west, xshift=0.000000mm, yshift=-24.000000mm] at(t_1_1_0.north west) {1920\\$c_0$=2\\$c_1$=1\\$c_2$=0};
\node(t_2_1_1) [CollTilingExampleStyle, fill=lightgray, anchor=north west, xshift=14.000000mm, yshift=0.000000mm] at(t_2_1_0.north west) {1921\\$c_0$=2\\$c_1$=1\\$c_2$=1};
\node(t_2_1_127) [CollTilingExampleStyle, fill=lightgray, anchor=north west, xshift=17.600000mm, yshift=0.000000mm] at(t_2_1_1.north west) {2047\\$c_0$=2\\$c_1$=1\\$c_2$=127};
\draw[dotted, thick] (t_2_1_1.east) -- (t_2_1_127.west);
\node(t_2_1_128) [CollTilingExampleStyle, fill=lightgray, anchor=north west, xshift=14.000000mm, yshift=0.000000mm] at(t_2_1_127.north west) {2048\\$c_0$=2\\$c_1$=1\\$c_2$=128};
\node(t_2_1_129) [CollTilingExampleStyle, fill=lightgray, anchor=north west, xshift=14.000000mm, yshift=0.000000mm] at(t_2_1_128.north west) {2049\\$c_0$=2\\$c_1$=1\\$c_2$=129};
\node(t_2_1_255) [CollTilingExampleStyle, fill=lightgray, anchor=north west, xshift=17.600000mm, yshift=0.000000mm] at(t_2_1_129.north west) {2175\\$c_0$=2\\$c_1$=1\\$c_2$=255};
\draw[dotted, thick] (t_2_1_129.east) -- (t_2_1_255.west);
\node(t_2_1_256) [CollTilingExampleStyle, fill=lightgray, anchor=north west, xshift=14.000000mm, yshift=0.000000mm] at(t_2_1_255.north west) {2176\\$c_0$=2\\$c_1$=1\\$c_2$=256};
\node(t_2_1_257) [CollTilingExampleStyle, fill=lightgray, anchor=north west, xshift=14.000000mm, yshift=0.000000mm] at(t_2_1_256.north west) {2177\\$c_0$=2\\$c_1$=1\\$c_2$=257};
\node(t_2_1_383) [CollTilingExampleStyle, fill=lightgray, anchor=north west, xshift=17.600000mm, yshift=0.000000mm] at(t_2_1_257.north west) {2303\\$c_0$=2\\$c_1$=1\\$c_2$=383};
\draw[dotted, thick] (t_2_1_257.east) -- (t_2_1_383.west);
\node(cta5) [yellowstyle, anchor=center] at(t_2_1_0.north west) {5$\rightarrow$};
\node(t_3_1_0) [CollTilingExampleStyle, fill=lightgray, anchor=north west, xshift=0.000000mm, yshift=-24.000000mm] at(t_2_1_0.north west) {2688\\$c_0$=3\\$c_1$=1\\$c_2$=0};
\node(t_3_1_1) [CollTilingExampleStyle, fill=lightgray, anchor=north west, xshift=14.000000mm, yshift=0.000000mm] at(t_3_1_0.north west) {2689\\$c_0$=3\\$c_1$=1\\$c_2$=1};
\node(t_3_1_127) [CollTilingExampleStyle, fill=lightgray, anchor=north west, xshift=17.600000mm, yshift=0.000000mm] at(t_3_1_1.north west) {2815\\$c_0$=3\\$c_1$=1\\$c_2$=127};
\draw[dotted, thick] (t_3_1_1.east) -- (t_3_1_127.west);
\node(t_3_1_128) [CollTilingExampleStyle, fill=lightgray, anchor=north west, xshift=14.000000mm, yshift=0.000000mm] at(t_3_1_127.north west) {2816\\$c_0$=3\\$c_1$=1\\$c_2$=128};
\node(t_3_1_129) [CollTilingExampleStyle, fill=lightgray, anchor=north west, xshift=14.000000mm, yshift=0.000000mm] at(t_3_1_128.north west) {2817\\$c_0$=3\\$c_1$=1\\$c_2$=129};
\node(t_3_1_255) [CollTilingExampleStyle, fill=lightgray, anchor=north west, xshift=17.600000mm, yshift=0.000000mm] at(t_3_1_129.north west) {2943\\$c_0$=3\\$c_1$=1\\$c_2$=255};
\draw[dotted, thick] (t_3_1_129.east) -- (t_3_1_255.west);
\node(t_3_1_256) [CollTilingExampleStyle, fill=lightgray, anchor=north west, xshift=14.000000mm, yshift=0.000000mm] at(t_3_1_255.north west) {2944\\$c_0$=3\\$c_1$=1\\$c_2$=256};
\node(t_3_1_257) [CollTilingExampleStyle, fill=lightgray, anchor=north west, xshift=14.000000mm, yshift=0.000000mm] at(t_3_1_256.north west) {2945\\$c_0$=3\\$c_1$=1\\$c_2$=257};
\node(t_3_1_383) [CollTilingExampleStyle, fill=lightgray, anchor=north west, xshift=17.600000mm, yshift=0.000000mm] at(t_3_1_257.north west) {3071\\$c_0$=3\\$c_1$=1\\$c_2$=383};
\draw[dotted, thick] (t_3_1_257.east) -- (t_3_1_383.west);
\node(cta7) [yellowstyle, anchor=center] at(t_3_1_0.north west) {7$\rightarrow$};
\node(t_0_0_0) [CollTilingExampleStyle, fill=white, anchor=north west, xshift=12.000000mm, yshift=42.000000mm] at(t_0_1_0.north west) {0\\$c_0$=0\\$c_1$=0\\$c_2$=0};
\node(t_0_0_1) [CollTilingExampleStyle, fill=white, anchor=north west, xshift=14.000000mm, yshift=0.000000mm] at(t_0_0_0.north west) {1\\$c_0$=0\\$c_1$=0\\$c_2$=1};
\node(t_0_0_127) [CollTilingExampleStyle, fill=white, anchor=north west, xshift=17.600000mm, yshift=0.000000mm] at(t_0_0_1.north west) {127\\$c_0$=0\\$c_1$=0\\$c_2$=127};
\draw[dotted, thick] (t_0_0_1.east) -- (t_0_0_127.west);
\node(t_0_0_128) [CollTilingExampleStyle, fill=white, anchor=north west, xshift=14.000000mm, yshift=0.000000mm] at(t_0_0_127.north west) {128\\$c_0$=0\\$c_1$=0\\$c_2$=128};
\node(t_0_0_129) [CollTilingExampleStyle, fill=white, anchor=north west, xshift=14.000000mm, yshift=0.000000mm] at(t_0_0_128.north west) {129\\$c_0$=0\\$c_1$=0\\$c_2$=129};
\node(t_0_0_255) [CollTilingExampleStyle, fill=white, anchor=north west, xshift=17.600000mm, yshift=0.000000mm] at(t_0_0_129.north west) {255\\$c_0$=0\\$c_1$=0\\$c_2$=255};
\draw[dotted, thick] (t_0_0_129.east) -- (t_0_0_255.west);
\node(t_0_0_256) [CollTilingExampleStyle, fill=white, anchor=north west, xshift=14.000000mm, yshift=0.000000mm] at(t_0_0_255.north west) {256\\$c_0$=0\\$c_1$=0\\$c_2$=256};
\node(t_0_0_257) [CollTilingExampleStyle, fill=white, anchor=north west, xshift=14.000000mm, yshift=0.000000mm] at(t_0_0_256.north west) {257\\$c_0$=0\\$c_1$=0\\$c_2$=257};
\node(t_0_0_383) [CollTilingExampleStyle, fill=white, anchor=north west, xshift=17.600000mm, yshift=0.000000mm] at(t_0_0_257.north west) {383\\$c_0$=0\\$c_1$=0\\$c_2$=383};
\draw[dotted, thick] (t_0_0_257.east) -- (t_0_0_383.west);
\node(cta0) [yellowstyle, anchor=center] at(t_0_0_0.north west) {0$\rightarrow$};
\node(t_1_0_0) [CollTilingExampleStyle, fill=white, anchor=north west, xshift=0.000000mm, yshift=-24.000000mm] at(t_0_0_0.north west) {768\\$c_0$=1\\$c_1$=0\\$c_2$=0};
\node(t_1_0_1) [CollTilingExampleStyle, fill=white, anchor=north west, xshift=14.000000mm, yshift=0.000000mm] at(t_1_0_0.north west) {769\\$c_0$=1\\$c_1$=0\\$c_2$=1};
\node(t_1_0_127) [CollTilingExampleStyle, fill=white, anchor=north west, xshift=17.600000mm, yshift=0.000000mm] at(t_1_0_1.north west) {895\\$c_0$=1\\$c_1$=0\\$c_2$=127};
\draw[dotted, thick] (t_1_0_1.east) -- (t_1_0_127.west);
\node(t_1_0_128) [CollTilingExampleStyle, fill=white, anchor=north west, xshift=14.000000mm, yshift=0.000000mm] at(t_1_0_127.north west) {896\\$c_0$=1\\$c_1$=0\\$c_2$=128};
\node(t_1_0_129) [CollTilingExampleStyle, fill=white, anchor=north west, xshift=14.000000mm, yshift=0.000000mm] at(t_1_0_128.north west) {897\\$c_0$=1\\$c_1$=0\\$c_2$=129};
\node(t_1_0_255) [CollTilingExampleStyle, fill=white, anchor=north west, xshift=17.600000mm, yshift=0.000000mm] at(t_1_0_129.north west) {1023\\$c_0$=1\\$c_1$=0\\$c_2$=255};
\draw[dotted, thick] (t_1_0_129.east) -- (t_1_0_255.west);
\node(t_1_0_256) [CollTilingExampleStyle, fill=white, anchor=north west, xshift=14.000000mm, yshift=0.000000mm] at(t_1_0_255.north west) {1024\\$c_0$=1\\$c_1$=0\\$c_2$=256};
\node(t_1_0_257) [CollTilingExampleStyle, fill=white, anchor=north west, xshift=14.000000mm, yshift=0.000000mm] at(t_1_0_256.north west) {1025\\$c_0$=1\\$c_1$=0\\$c_2$=257};
\node(t_1_0_383) [CollTilingExampleStyle, fill=white, anchor=north west, xshift=17.600000mm, yshift=0.000000mm] at(t_1_0_257.north west) {1151\\$c_0$=1\\$c_1$=0\\$c_2$=383};
\draw[dotted, thick] (t_1_0_257.east) -- (t_1_0_383.west);
\node(cta2) [yellowstyle, anchor=center] at(t_1_0_0.north west) {2$\rightarrow$};
\node(t_2_0_0) [CollTilingExampleStyle, fill=white, anchor=north west, xshift=0.000000mm, yshift=-24.000000mm] at(t_1_0_0.north west) {1536\\$c_0$=2\\$c_1$=0\\$c_2$=0};
\node(t_2_0_1) [CollTilingExampleStyle, fill=white, anchor=north west, xshift=14.000000mm, yshift=0.000000mm] at(t_2_0_0.north west) {1537\\$c_0$=2\\$c_1$=0\\$c_2$=1};
\node(t_2_0_127) [CollTilingExampleStyle, fill=white, anchor=north west, xshift=17.600000mm, yshift=0.000000mm] at(t_2_0_1.north west) {1663\\$c_0$=2\\$c_1$=0\\$c_2$=127};
\draw[dotted, thick] (t_2_0_1.east) -- (t_2_0_127.west);
\node(t_2_0_128) [CollTilingExampleStyle, fill=white, anchor=north west, xshift=14.000000mm, yshift=0.000000mm] at(t_2_0_127.north west) {1664\\$c_0$=2\\$c_1$=0\\$c_2$=128};
\node(t_2_0_129) [CollTilingExampleStyle, fill=white, anchor=north west, xshift=14.000000mm, yshift=0.000000mm] at(t_2_0_128.north west) {1665\\$c_0$=2\\$c_1$=0\\$c_2$=129};
\node(t_2_0_255) [CollTilingExampleStyle, fill=white, anchor=north west, xshift=17.600000mm, yshift=0.000000mm] at(t_2_0_129.north west) {1791\\$c_0$=2\\$c_1$=0\\$c_2$=255};
\draw[dotted, thick] (t_2_0_129.east) -- (t_2_0_255.west);
\node(t_2_0_256) [CollTilingExampleStyle, fill=white, anchor=north west, xshift=14.000000mm, yshift=0.000000mm] at(t_2_0_255.north west) {1792\\$c_0$=2\\$c_1$=0\\$c_2$=256};
\node(t_2_0_257) [CollTilingExampleStyle, fill=white, anchor=north west, xshift=14.000000mm, yshift=0.000000mm] at(t_2_0_256.north west) {1793\\$c_0$=2\\$c_1$=0\\$c_2$=257};
\node(t_2_0_383) [CollTilingExampleStyle, fill=white, anchor=north west, xshift=17.600000mm, yshift=0.000000mm] at(t_2_0_257.north west) {1919\\$c_0$=2\\$c_1$=0\\$c_2$=383};
\draw[dotted, thick] (t_2_0_257.east) -- (t_2_0_383.west);
\node(cta4) [yellowstyle, anchor=center] at(t_2_0_0.north west) {4$\rightarrow$};
\node(t_3_0_0) [CollTilingExampleStyle, fill=white, anchor=north west, xshift=0.000000mm, yshift=-24.000000mm] at(t_2_0_0.north west) {2304\\$c_0$=3\\$c_1$=0\\$c_2$=0};
\node(t_3_0_1) [CollTilingExampleStyle, fill=white, anchor=north west, xshift=14.000000mm, yshift=0.000000mm] at(t_3_0_0.north west) {2305\\$c_0$=3\\$c_1$=0\\$c_2$=1};
\node(t_3_0_127) [CollTilingExampleStyle, fill=white, anchor=north west, xshift=17.600000mm, yshift=0.000000mm] at(t_3_0_1.north west) {2431\\$c_0$=3\\$c_1$=0\\$c_2$=127};
\draw[dotted, thick] (t_3_0_1.east) -- (t_3_0_127.west);
\node(t_3_0_128) [CollTilingExampleStyle, fill=white, anchor=north west, xshift=14.000000mm, yshift=0.000000mm] at(t_3_0_127.north west) {2432\\$c_0$=3\\$c_1$=0\\$c_2$=128};
\node(t_3_0_129) [CollTilingExampleStyle, fill=white, anchor=north west, xshift=14.000000mm, yshift=0.000000mm] at(t_3_0_128.north west) {2433\\$c_0$=3\\$c_1$=0\\$c_2$=129};
\node(t_3_0_255) [CollTilingExampleStyle, fill=white, anchor=north west, xshift=17.600000mm, yshift=0.000000mm] at(t_3_0_129.north west) {2559\\$c_0$=3\\$c_1$=0\\$c_2$=255};
\draw[dotted, thick] (t_3_0_129.east) -- (t_3_0_255.west);
\node(t_3_0_256) [CollTilingExampleStyle, fill=white, anchor=north west, xshift=14.000000mm, yshift=0.000000mm] at(t_3_0_255.north west) {2560\\$c_0$=3\\$c_1$=0\\$c_2$=256};
\node(t_3_0_257) [CollTilingExampleStyle, fill=white, anchor=north west, xshift=14.000000mm, yshift=0.000000mm] at(t_3_0_256.north west) {2561\\$c_0$=3\\$c_1$=0\\$c_2$=257};
\node(t_3_0_383) [CollTilingExampleStyle, fill=white, anchor=north west, xshift=17.600000mm, yshift=0.000000mm] at(t_3_0_257.north west) {2687\\$c_0$=3\\$c_1$=0\\$c_2$=383};
\draw[dotted, thick] (t_3_0_257.east) -- (t_3_0_383.west);
\node(cta6) [yellowstyle, anchor=center] at(t_3_0_0.north west) {6$\rightarrow$};
\draw[thick, greenstyle] ($(t_0_1_0.north west) + (0.000000mm, -7.800000mm)$) -- ($(t_0_1_0.north west) + (-7.500000mm, -7.800000mm)$);
\draw[thick, greenstyle] ($(t_3_1_0.north west) + (0.000000mm, -7.800000mm)$) -- ($(t_3_1_0.north west) + (-7.500000mm, -7.800000mm)$);
\draw[thick, <->, greenstyle] ($(t_0_1_0.north west) + (-3.750000mm, -7.800000mm)$) -- ($(t_3_1_0.north west) + (-3.750000mm, -7.800000mm)$);
\node(D0) [anchor=west] at($(t_0_1_0.north west)!0.5!(t_3_1_0.north west) + (-7.500000mm, -7.800000mm)$) {\greenBox{$\omega.D_0 = 4$}};
\draw[thick, violetstyle] ($(t_0_1_0.north west) + (0.000000mm, -13.000000mm)$) -- ($(t_0_1_0.north west) + (-7.500000mm, -13.000000mm)$);
\draw[thick, violetstyle] ($(t_0_0_0.north west) + (0.000000mm, -13.000000mm)$) -- ($(t_0_0_0.north west) + (-7.500000mm, -13.000000mm)$);
\draw[thick, <->, violetstyle] ($(t_0_1_0.north west) + (-3.750000mm, -13.000000mm)$) -- ($(t_0_0_0.north west) + (-3.750000mm, -13.000000mm)$);
\node(D1) [anchor=center] at($(t_0_1_0.north west)!0.5!(t_0_0_0.north west) + (-3.750000mm, -13.000000mm)$) {\violetBox{$\omega.D_1 = 2$}};
\draw[thick, bluestyle] ($(t_3_1_0.south) + (0mm, 0mm)$) -- ($(t_3_1_0.south) + (0mm, -7.5000000mm)$);
\draw[thick, bluestyle] ($(t_3_1_383.south) + (0mm, 0mm)$) -- ($(t_3_1_383.south) + (0mm, -7.5000000mm)$);
\draw[thick, <->, bluestyle] ($(t_3_1_0.south) + (0mm, -3.750000mm)$) -- ($(t_3_1_383.south) + (0mm, -3.750000mm)$);
\node(D2) [anchor=center] at($(t_3_1_0.south)!0.5!(t_3_1_383.south) + (0mm, -3.750000mm)$) {\blueBox{$\omega.D_2 = 384$}};
\draw[thick, violetstyle] ($(t_3_0_383.south east)!0.950000!(t_3_1_383.south east)$) -- ($(t_3_0_383.south east)!0.950000!(t_3_1_383.south east) + (7.500000mm, 0mm)$);
\draw[thick, violetstyle] ($(t_3_0_383.south east)!0.550000!(t_3_1_383.south east)$) -- ($(t_3_0_383.south east)!0.550000!(t_3_1_383.south east) + (7.500000mm, 0mm)$);
\draw[thick, violetstyle] ($(t_3_0_383.south east)!0.950000!(t_3_1_383.south east) + (7.500000mm, 0mm)$) -- ($(t_3_0_383.south east)!0.550000!(t_3_1_383.south east) + (7.500000mm, 0mm)$);
\node[anchor=center] at($(t_3_0_383.south east)!0.750000!(t_3_1_383.south east) + (7.500000mm, 0mm)$) {\violetBox{\texttt{n\_cta=1}}};
\draw[thick, violetstyle] ($(t_3_0_383.south east)!0.450000!(t_3_1_383.south east)$) -- ($(t_3_0_383.south east)!0.450000!(t_3_1_383.south east) + (7.500000mm, 0mm)$);
\draw[thick, violetstyle] ($(t_3_0_383.south east)!0.050000!(t_3_1_383.south east)$) -- ($(t_3_0_383.south east)!0.050000!(t_3_1_383.south east) + (7.500000mm, 0mm)$);
\draw[thick, violetstyle] ($(t_3_0_383.south east)!0.450000!(t_3_1_383.south east) + (7.500000mm, 0mm)$) -- ($(t_3_0_383.south east)!0.050000!(t_3_1_383.south east) + (7.500000mm, 0mm)$);
\node[anchor=center] at($(t_3_0_383.south east)!0.250000!(t_3_1_383.south east) + (7.500000mm, 0mm)$) {\violetBox{\texttt{n\_cta=0}}};
\draw[thick, greenstyle] ($(t_0_0_383.north east) + (1.875000mm, 0mm)$) -- ($(t_0_0_383.north east) + (9.375000mm, 0mm)$);
\draw[thick, greenstyle] ($(t_0_0_383.south east) + (1.875000mm, 0mm)$) -- ($(t_0_0_383.south east) + (9.375000mm, 0mm)$);
\draw[thick, greenstyle] ($(t_0_0_383.south east) + (9.375000mm, 0mm)$) -- ($(t_0_0_383.north east) + (9.375000mm, 0mm)$);
\node[anchor=center] at($(t_0_0_383.north east)!0.5!(t_0_0_383.south east) + (9.375000mm, 0mm)$) {\greenBox{\texttt{m\_cta=0}}};
\draw[thick, greenstyle] ($(t_1_0_383.north east) + (1.875000mm, 0mm)$) -- ($(t_1_0_383.north east) + (9.375000mm, 0mm)$);
\draw[thick, greenstyle] ($(t_1_0_383.south east) + (1.875000mm, 0mm)$) -- ($(t_1_0_383.south east) + (9.375000mm, 0mm)$);
\draw[thick, greenstyle] ($(t_1_0_383.south east) + (9.375000mm, 0mm)$) -- ($(t_1_0_383.north east) + (9.375000mm, 0mm)$);
\node[anchor=center] at($(t_1_0_383.north east)!0.5!(t_1_0_383.south east) + (9.375000mm, 0mm)$) {\greenBox{\texttt{m\_cta=1}}};
\draw[thick, greenstyle] ($(t_2_0_383.north east) + (1.875000mm, 0mm)$) -- ($(t_2_0_383.north east) + (9.375000mm, 0mm)$);
\draw[thick, greenstyle] ($(t_2_0_383.south east) + (1.875000mm, 0mm)$) -- ($(t_2_0_383.south east) + (9.375000mm, 0mm)$);
\draw[thick, greenstyle] ($(t_2_0_383.south east) + (9.375000mm, 0mm)$) -- ($(t_2_0_383.north east) + (9.375000mm, 0mm)$);
\node[anchor=center] at($(t_2_0_383.north east)!0.5!(t_2_0_383.south east) + (9.375000mm, 0mm)$) {\greenBox{\texttt{m\_cta=2}}};
\draw[thick, greenstyle] ($(t_3_0_383.north east) + (1.875000mm, 0mm)$) -- ($(t_3_0_383.north east) + (9.375000mm, 0mm)$);
\draw[thick, greenstyle] ($(t_3_0_383.south east) + (1.875000mm, 0mm)$) -- ($(t_3_0_383.south east) + (9.375000mm, 0mm)$);
\draw[thick, greenstyle] ($(t_3_0_383.south east) + (9.375000mm, 0mm)$) -- ($(t_3_0_383.north east) + (9.375000mm, 0mm)$);
\node[anchor=center] at($(t_3_0_383.north east)!0.5!(t_3_0_383.south east) + (9.375000mm, 0mm)$) {\greenBox{\texttt{m\_cta=3}}};
\draw[thick, bluestyle] ($(t_0_0_128.north west) + (1.750000mm, 20.625000mm)$) -- ($(t_0_0_128.north west) + (1.750000mm, 28.125000mm)$);
\draw[thick, bluestyle] ($(t_0_0_383.north east) + (-1.750000mm, 20.625000mm)$) -- ($(t_0_0_383.north east) + (-1.750000mm, 28.125000mm)$);
\draw[thick, bluestyle] ($(t_0_0_128.north west) + (1.750000mm, 28.125000mm)$) -- ($(t_0_0_383.north east) + (-1.750000mm, 28.125000mm)$);
\node[anchor=center] at($(t_0_0_128.north west)!0.5!(t_0_0_383.north east) + (0.000000mm, 28.125000mm)$) {\blueBox{\texttt{CudaWarps\_consumer\_None\_None=0}}};
\draw[thick, bluestyle] ($(t_0_0_128.north west) + (1.750000mm, 11.250000mm)$) -- ($(t_0_0_128.north west) + (1.750000mm, 18.750000mm)$);
\draw[thick, bluestyle] ($(t_0_0_255.north east) + (-1.750000mm, 11.250000mm)$) -- ($(t_0_0_255.north east) + (-1.750000mm, 18.750000mm)$);
\draw[thick, bluestyle] ($(t_0_0_128.north west) + (1.750000mm, 18.750000mm)$) -- ($(t_0_0_255.north east) + (-1.750000mm, 18.750000mm)$);
\node[anchor=center] at($(t_0_0_128.north west)!0.5!(t_0_0_255.north east) + (0.000000mm, 18.750000mm)$) {\blueBox{\texttt{wg=0}}};
\draw[thick, bluestyle] ($(t_0_0_128.north west) + (1.750000mm, 1.875000mm)$) -- ($(t_0_0_128.north west) + (1.750000mm, 9.375000mm)$);
\draw[thick, bluestyle] ($(t_0_0_128.north east) + (-1.750000mm, 1.875000mm)$) -- ($(t_0_0_128.north east) + (-1.750000mm, 9.375000mm)$);
\draw[thick, bluestyle] ($(t_0_0_128.north west) + (1.750000mm, 9.375000mm)$) -- ($(t_0_0_128.north east) + (-1.750000mm, 9.375000mm)$);
\node[anchor=center] at($(t_0_0_128.north west)!0.5!(t_0_0_128.north east) + (0.000000mm, 9.375000mm)$) {\blueBox{\texttt{t=0}}};
\draw[thick, bluestyle] ($(t_0_0_129.north west) + (1.750000mm, 1.875000mm)$) -- ($(t_0_0_129.north west) + (1.750000mm, 9.375000mm)$);
\draw[thick, bluestyle] ($(t_0_0_129.north east) + (-1.750000mm, 1.875000mm)$) -- ($(t_0_0_129.north east) + (-1.750000mm, 9.375000mm)$);
\draw[thick, bluestyle] ($(t_0_0_129.north west) + (1.750000mm, 9.375000mm)$) -- ($(t_0_0_129.north east) + (-1.750000mm, 9.375000mm)$);
\node[anchor=center] at($(t_0_0_129.north west)!0.5!(t_0_0_129.north east) + (0.000000mm, 9.375000mm)$) {\blueBox{\texttt{t=1}}};
\draw[thick, bluestyle] ($(t_0_0_255.north west) + (1.750000mm, 1.875000mm)$) -- ($(t_0_0_255.north west) + (1.750000mm, 9.375000mm)$);
\draw[thick, bluestyle] ($(t_0_0_255.north east) + (-1.750000mm, 1.875000mm)$) -- ($(t_0_0_255.north east) + (-1.750000mm, 9.375000mm)$);
\draw[thick, bluestyle] ($(t_0_0_255.north west) + (1.750000mm, 9.375000mm)$) -- ($(t_0_0_255.north east) + (-1.750000mm, 9.375000mm)$);
\node[anchor=center] at($(t_0_0_255.north west)!0.5!(t_0_0_255.north east) + (0.000000mm, 9.375000mm)$) {\blueBox{\texttt{t=127}}};
\draw[thick, bluestyle] ($(t_0_0_256.north west) + (1.750000mm, 11.250000mm)$) -- ($(t_0_0_256.north west) + (1.750000mm, 18.750000mm)$);
\draw[thick, bluestyle] ($(t_0_0_383.north east) + (-1.750000mm, 11.250000mm)$) -- ($(t_0_0_383.north east) + (-1.750000mm, 18.750000mm)$);
\draw[thick, bluestyle] ($(t_0_0_256.north west) + (1.750000mm, 18.750000mm)$) -- ($(t_0_0_383.north east) + (-1.750000mm, 18.750000mm)$);
\node[anchor=center] at($(t_0_0_256.north west)!0.5!(t_0_0_383.north east) + (0.000000mm, 18.750000mm)$) {\blueBox{\texttt{wg=1}}};
\draw[thick, bluestyle] ($(t_0_0_256.north west) + (1.750000mm, 1.875000mm)$) -- ($(t_0_0_256.north west) + (1.750000mm, 9.375000mm)$);
\draw[thick, bluestyle] ($(t_0_0_256.north east) + (-1.750000mm, 1.875000mm)$) -- ($(t_0_0_256.north east) + (-1.750000mm, 9.375000mm)$);
\draw[thick, bluestyle] ($(t_0_0_256.north west) + (1.750000mm, 9.375000mm)$) -- ($(t_0_0_256.north east) + (-1.750000mm, 9.375000mm)$);
\node[anchor=center] at($(t_0_0_256.north west)!0.5!(t_0_0_256.north east) + (0.000000mm, 9.375000mm)$) {\blueBox{\texttt{t=0}}};
\draw[thick, bluestyle] ($(t_0_0_257.north west) + (1.750000mm, 1.875000mm)$) -- ($(t_0_0_257.north west) + (1.750000mm, 9.375000mm)$);
\draw[thick, bluestyle] ($(t_0_0_257.north east) + (-1.750000mm, 1.875000mm)$) -- ($(t_0_0_257.north east) + (-1.750000mm, 9.375000mm)$);
\draw[thick, bluestyle] ($(t_0_0_257.north west) + (1.750000mm, 9.375000mm)$) -- ($(t_0_0_257.north east) + (-1.750000mm, 9.375000mm)$);
\node[anchor=center] at($(t_0_0_257.north west)!0.5!(t_0_0_257.north east) + (0.000000mm, 9.375000mm)$) {\blueBox{\texttt{t=1}}};
\draw[thick, bluestyle] ($(t_0_0_383.north west) + (1.750000mm, 1.875000mm)$) -- ($(t_0_0_383.north west) + (1.750000mm, 9.375000mm)$);
\draw[thick, bluestyle] ($(t_0_0_383.north east) + (-1.750000mm, 1.875000mm)$) -- ($(t_0_0_383.north east) + (-1.750000mm, 9.375000mm)$);
\draw[thick, bluestyle] ($(t_0_0_383.north west) + (1.750000mm, 9.375000mm)$) -- ($(t_0_0_383.north east) + (-1.750000mm, 9.375000mm)$);
\node[anchor=center] at($(t_0_0_383.north west)!0.5!(t_0_0_383.north east) + (0.000000mm, 9.375000mm)$) {\blueBox{\texttt{t=127}}};


\node(keyText) [anchor=south west, yshift=60mm, xshift=-5mm] at(cta1.north west) {\textbf{KEY:}};
\node(keyThread) [CollTilingExampleStyle, anchor=west] at(keyText.east) {tid\\$c_0$\\$c_1$\\$c_2$};
\node(keyTid) [anchor=west, text width=40mm] at(keyThread.east) {where ``tid'' is the local thread index given by toLocal($\omega.D, (c_0, c_1, c_2)$) (def~\ref{sec:gLocalThreadIndex},~def~\ref{sec:gToLocal}).};
\node(keyCta) [anchor=north west, yellowstyle, yshift=-2mm] at(keyThread.south west) {\texttt{cluster\_ctarank} (def~\ref{sec:gCluster})$\rightarrow$};

\node(ops_2_2_value) [anchor=south east, bluestyle, text width=80mm, yshift=+35mm] at(t_0_0_383.north east)  {\rmfamily $\textit{offset}=0$, $\textit{box}=1$, $\textit{tileCount}=128, \textit{iter}=\texttt{t}$};
\node(ops_2_1_value) [anchor=south east, bluestyle, text width=80mm, yshift=+1mm] at(ops_2_2_value.north east)  {\rmfamily $\textit{offset}=0$, $\textit{box}=128$, $\textit{tileCount}=2, \textit{iter}=\texttt{wg}$};
\node(ops_2_0_value) [anchor=south east, bluestyle, text width=80mm, yshift=+1mm] at(ops_2_1_value.north east)  {\rmfamily $\textit{offset}=128$, $\textit{box}=256$, $\textit{tileCount}=1$,\\$\textit{iter}=\texttt{CudaWarps\_consumer\_None\_None}$};
\node(ops_1_0_value) [anchor=south east, violetstyle, text width=80mm, yshift=+1mm] at(ops_2_0_value.north east) {\rmfamily $\textit{offset}=0$, $\textit{box}=1$, $\textit{tileCount}=2$, $\textit{iter}=\texttt{n\_cta}$};
\node(ops_0_0_value) [anchor=south east, greenstyle, text width=80mm, yshift=+1mm] at(ops_1_0_value.north east) {\rmfamily $\textit{offset}=0$, $\textit{box}=1$, $\textit{tileCount}=4$, $\textit{iter}=\texttt{m\_cta}$};

\node(ops_2_2_label) [anchor=east] at(ops_2_2_value.west) {\rmfamily $\omega_2.\textit{ops}_2$};
\node(ops_2_1_label) [anchor=east] at(ops_2_1_value.west) {\rmfamily $\omega_2.\textit{ops}_1$};
\node(ops_2_0_label) [anchor=east] at(ops_2_0_value.west) {\rmfamily $\omega_2.\textit{ops}_0$};
\node(ops_1_0_label) [anchor=east] at(ops_1_0_value.west) {\rmfamily $\omega_1.\textit{ops}_0$};
\node(ops_0_0_label) [anchor=east] at(ops_0_0_value.west) {\rmfamily $\omega_0.\textit{ops}_0$};

\node(dim2) [anchor=north west, xshift=-50mm] at(ops_2_0_value.north west) {\textbf{Dim 2:} $\omega_2.n = 384$};
\node(dim1) [anchor=north west, xshift=-50mm] at(ops_1_0_value.north west) {\textbf{Dim 1:} $\omega_1.n = 2$};
\node(dim0) [anchor=north west, xshift=-50mm] at(ops_0_0_value.north west) {\textbf{Dim 0:} $\omega_0.n = 4$};
\node(CollTiling) [draw=black, anchor=east, text width=20mm] at(dim1.west) {CollTiling\\$\omega: \Omega$ state};

\node(domain) [anchor=north east, yshift=-2mm, text width=50mm] at(dim2.south east) {The domain $\omega.D = (4, 2, 384)$ is given by ($\omega_0.n$, $\omega_1.n$, $\omega_2.n$) (def~\ref{sec:gDomain}).};

\end{tikzpicture}
}

\FloatBarrier
\newpage

