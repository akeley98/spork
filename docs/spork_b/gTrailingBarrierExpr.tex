\magicSubsection{Trailing Barrier Expression}{sec:gTrailingBarrierExpr}

Syntax ``\texttt{>}\texttt{> }$e$'' added after the closing \texttt{)} of an instruction call, with $e$ being a single barrier expression (def~\ref{sec:gBarrierExpr}).
This allows direct usage of \lighttt{Await} (Section~\ref{sec:ArriveAwaitPairing}) to wait for the instruction to retire, which is treated as a special case in sync-check (Section~\ref{sec:VisRecordCreation}, Section~\ref{sec:AwaitSemantics}).

The trailing barrier expression for an instruction is separate from its formal parameters.
The \lighttt{replace} function does \textbf{not} require a correct trailing barrier expression.
Use the scheduling function \lighttt{set\_trailing\_barrier\_expr} after the \lighttt{replace}.

Note, the home barrier expression (def~\ref{sec:gHomeBarrier}) concept does \textbf{not} apply to the trailing barrier expression.

% >U

