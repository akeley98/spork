\magicSection{Distributed Memory}{sec:DistributedMemory}

Both barrier and data allocations in CUDA scope are subject to distributed memory analysis, which deduces a certain number of distributed dimensions (def~\ref{sec:gDistributedDimension}), and properties of those dimensions.
All other allocations have 0 distributed dimensions.

The rules are somewhat different for the two cases.
For barrier allocations, all dimensions are distributed.
For data allocations, the memory type specifies a collective unit (the \myKeyA{native unit}, def~\ref{sec:gNativeUnit}) from which a collective type $\delta_0$ is unpacked with alignment and 1-padding (def~\ref{sec:gCollUnit}); this must not be agnostic (def~\ref{sec:gAgnostic}).
Exo-GPU then deduces a certain number of distributed dimensions (say, $R$) so that each shard (def~\ref{sec:gShard}) $x[c_0,...,c_{R-1},:,...,:]$ is allocated on a different thread collective, all of which are instances of $\delta_0$ (Section~\ref{sec:CollTypeThreadCollective}).
The distributed dimensions are always to the left of non-distributed dimensions.

During codegen (Section~\ref{sec:InstrCodegen}), Exo-GPU erases indicies and array extents corresponding to distributed dimensions.
The codegen functions for memory and instructions only see indices and extents corresponding to non-distributed dimensions.

Distributed memory analysis runs independently for each data variable and barrier variable, with the following rules.

\magicSubsection{Collective Indexing Pairs}{sec:CollIndexingPairs}

Distributed memory analysis for a variable is based on the collective tiling $\omega_0^\text{alloc}$ (Section~\ref{sec:CollTiling}) of the variable allocation statement, and the set of \myKeyA{collective indexing pairs} $\Omega \times \mathsf{Expr}^*$ collected from all statements and expressions that index the variable:
\begin{itemize}
  \item For read expressions ``$x[e^*]$'' not in the context of an instruction call, the collective indexing pair is $(\omega, e^*)$, where $\omega$ is the collective tiling of the statement containing the read expression.
  \item For write statements ``$x[e^*] \texttt{= \_}$'' and reduce statements ``$x[e^*] \texttt{+= \_}$'', the collective indexing pair is $(\omega, e^*)$, where $\omega$ is the collective tiling of the statement.
  \item For sync statements, the collective indexing pair is $(\omega, e^*)$, where $\omega$ is the collective tiling of the statement and $e^*$ is extracted from the home barrier expression $z[e^*]$ (def~\ref{sec:gHomeBarrier}).
  \item The collective indexing expressions collected from instruction calls are described in Section~\ref{sec:InstrDistributedMemory}.
\end{itemize}


\magicSubsection{Thread Pitch Requirement}{sec:DistributedMemoryThreadPitch}

For each collective indexing pair $(\omega_0, e^*)$, we deduce a thread pitch tuple.
The deduced number of distributed dimensions ($R$) is the length of this tuple.
All collective indexing pairs must lead to the same deduced tuple.

We will soon define what a \myKeyA{required iterator} is separately for data and barriers.
A \myKeyA{required index expression} is a plain read of a required iterator.
A \myKeyA{permitted index expression} is a plain read of a required iterator, or of a \lighttt{cuda\_threads} iterator with 0 thread pitch, as defined by $\omega_0$ (Section~\ref{sec:CollTilingDerivedState}).

\mainKey{Data:} A collective index pair $(\omega_0, e^*)$ is first updated with domain completion (def~\ref{sec:gDomainCompletion}) so that the memory type's collective type ($\delta_0$) has the same domain as the collective tiling.
Let $(\omega, \delta) = \mathsf{domainCompletion}(\omega_0, \delta_0)$.

Dimension $i$ is a \myKeyA{subdivided dimension} if $\delta.B_i = 1$.
An iterator is a \myKeyA{required iterator} if all these conditions apply:
\begin{itemize}
  \item appears in $\omega$
  \item does not appear in $\omega_0^\text{alloc}$
  \item its tiled dimension index is that of a subdivided dimension (as defined by $\omega$, Section~\ref{sec:CollTilingDerivedState})
  \item its thread pitch is not 0 (as defined by $\omega$, Section~\ref{sec:CollTilingDerivedState})
\end{itemize}
The number of distributed dimensions $R$ is the lowest possible value $R$ such that $e_0, ..., e_{R-1}$ consists only of permitted index expressions, and all required index expressions appear exactly once (fail if this is not possible).
The thread pitch tuple is $(g(e_0), ..., g(e_{R-1}))$ where $g(e)$ gives the thread pitch (as defined by $\omega$) of the iterator indexed by the permitted index expression $e$.

\mainKey{Barrier:} A collective index pair $(\omega_0, e^*)$ is processed without domain completion.
Let $\omega = \omega_0$.
An iterator is a \myKeyA{required iterator} if all these conditions apply:
\begin{itemize}
  \item appears in $\omega$
  \item does not appear in $\omega_0^\text{alloc}$
  \item its thread pitch is not 0 (as defined by $\omega$, Section~\ref{sec:CollTilingDerivedState})
\end{itemize}
The number of distributed dimensions $R$ is the number of index expressions $e^*$, which must consist only of permitted index expressions, and all required index expressions must appear exactly once (fail if this is not the case).
The thread pitch tuple is $(g(e_0), ..., g(e_{R-1}))$ where $g(e)$ gives the thread pitch (as defined by $\omega$) of the iterator indexed by the permitted index expression $e$.


\magicSubsection{Range Requirement}{sec:DistributedMemoryRange}

The iterator (control variable) used to index a distributed dimension must have range $[0, x_i - 1]_\mathbb{N}$ where $x_i$ is the extent of the array on that dimension.


\magicSubsection{Base Threads Requirement}{sec:DistributedMemoryBaseThreads}

The thread pitch tuple requirement will diagnose many cases of inconsistent sharding, but will not detect ``offset-only'' mismatches, such as $x[0], x[1], x[2]...$ being accessed by threads 0, 1, 2, ... in one usage and threads 64, 65, 66, ... in another usage (as could occur with warp specialization).
The base threads requirement addresses this issue.
We will define the \myKeyA{base offset} and \myKeyA{base box} of a collective tiling separately for data and barrier variables.

\mainKey{Barrier:} Given an $M$-dimensional collective tiling $\omega$, the \myKeyA{base offset} is defined by the tuple $(O_0, ..., O_{M-1})$ where $O_i = \sum_{j} \omega_i.\textit{ops}_j.\textit{offset}$ (Section~\ref{sec:CollDimDescriptor}).
The \myKeyA{base box} is $\omega.B$.

\mainKey{Data:} Given a collective tiling $\omega_0$ and the collective type $\delta_0$ unpacked from the native unit (def~\ref{sec:gNativeUnit}), let $(\omega, \delta) = \mathsf{domainCompletion}(\omega_0, \delta_0)$.
The \myKeyA{base offset} $O$ and the \myKeyA{base box} $\textit{BB}$ are both $M$ tuples, $M$ being the dimensionality of $\omega$.
The $i^{th}$ dimension is a \myKeyA{intact dimension} if $\delta.B_i = \delta.D_i$.
For intact dimensions\footnote{The motivation of this is to fill in intact dimensions with ``default'' values that will never fail the equality test.
For example, a collective type for a CTA may have domain (4, 384) and box (1, 384), so dimension 0 is subdivided and dimension 1 is intact.
The defaulting on dimension 1 ensures the base threads requirement allows different thread collectives within the same CTA to access an SMEM shard, while still ensuring incompatibilities on dimension 0 (different CTAs in the cluster) are diagnosed.}
$i$, $O_i = 0$ and $\textit{BB}_i = \delta.D_i$.
For other dimensions, $O_i = \sum_{j} \omega_i.\textit{ops}_j.\textit{offset}$ (Section~\ref{sec:CollDimDescriptor}) and $\textit{BB}_i = \omega.B_i$.

\mainKey{Base Offset Equivalence:} Compute the linear offset of $\omega$ as $\sum_{i} O_i (\omega.P_i)$, where $O$ denotes the base offset derived from $\omega$, and $\omega.P_i$ denotes the dimension thread pitch (def~\ref{sec:gThreadPitch}).
Base offset equivalence means that the linear offset computed for two collective tilings are the same.

\mainKey{Base Box Equivalence:} Given collective tilings $\omega_1$ and $\omega_2$ with base boxes $\textit{BB}_1$ and $\textit{BB}_2$ respectively, the two collective tilings have equivalent base boxes when $\delta_1 = \delta_2$ given $(\delta_1, \delta_2) = \mathsf{domainCompletionTypeOnly}((\omega_1.D, \textit{BB}_1), (\omega_2.D, \textit{BB}_2))$ (def~\ref{sec:gDomainCompletionTypeOnly}).

\mainKey{Barrier Requirements:} The base threads requirement is met for two collective indexing pairs when their two collective tilings satisfy base offset equivalence and base box equivalence.
The specific pairs that must satisfy this requirement varies depending on the barrier mechanism (def~\ref{sec:gBarrierMechanism}).

\mainKey{Data Requirements:} Any two collective indexing pairs for a given data variable must have their two collective tilings satisfy base offset equivalence and base box equivalence.



\magicSubsection{Box Size Requirements}{sec:DistributedMemoryBoxSize}

These additional requirements, which apply to data allocations only, ensure the storage for each shard is truly allocated by a thread collective that is an instance of $\delta_0$ (Section~\ref{sec:CollTypeThreadCollective}).

\mainKey{Alloc Box Requirement:} Let $(\omega^\text{alloc}, \delta) = \mathsf{domainCompletion}(\omega_0^\text{alloc}, \delta_0)$.
The $i^{th}$ dimension is an \myKeyA{intact dimension} if $\delta.B_i = \delta.D_i$.
For each intact dimension $i$, we must have $\omega^\text{alloc}.B_i = \delta.B_i$.

\mainKey{Usage Box Requirement:} For each collective indexing pair $(\omega_0, e^*)$, let $(\omega, \delta) = \mathsf{domainCompletion}(\omega_0, \delta_0)$.
The $i^{th}$ dimension is a \myKeyA{subdivided dimension} if $\delta.B_i = 1$.
Given $\delta_0$ is in aligned form (def~\ref{sec:gAlignedForm}) and not agnostic (def~\ref{sec:gAgnostic}), this is mutually exclusive with being an intact dimension.
For each subdivided dimension $i$, we must have $\omega.B_i = 1$.

\FloatBarrier
\newpage

