% exocc f16_gemm_samples.py && python3 code_to_tex.py f16_gemm_samples.py f16_gemm && xelatex f16_gemm_todo.tex </dev/null
\documentclass[11pt]{article}

\usepackage[letterpaper, portrait, margin=2.54cm]{geometry}
\usepackage{enumitem}
\usepackage{amsmath}
\usepackage{amssymb}
\usepackage{amsfonts}
\usepackage{placeins}
\usepackage{graphicx}
\usepackage{listings}
\usepackage{caption}
\usepackage{colortbl}
\usepackage[parfill]{parskip}
\usepackage[mathscr]{euscript}
\usepackage[usenames,dvipsnames,svgnames,table,hyperref]{xcolor}
\usepackage[hidelinks]{hyperref}
\usepackage{fontspec}
\usepackage{mdframed}
\usepackage{tikz}
\usetikzlibrary{shapes.geometric, arrows, positioning, calc}

\setsansfont{FreeSans}
\setmonofont{Ubuntu Mono}

\hyphenation{WebGL}

%\definecolor{webColor}{RGB}{0, 124, 204}
\definecolor{webColor}{RGB}{0, 108, 174}
\newcommand{\web}[1]{{\color{webColor} \small \url{#1}}}
\newcommand{\webText}[2]{{\color{webColor} \href{#2}{#1}}}
\newcommand{\email}[2]{{\small \color{webColor} \textsf{\href{mailto:#1@#2}{#1[at]#2}}}}
\definecolor{titleColor}{RGB}{0, 0, 0}
\ifdefined\RESUME
\newcommand{\myTitle}[1]{{\large \color{titleColor} \hspace{-12mm} \textbf{\textsf{#1}}}}
\else
\newcommand{\myTitle}[1]{{ \vspace{2mm} \large \color{titleColor} \hspace{-12mm} \textbf{\textsf{#1}} \vspace{2mm}}}
\fi
\definecolor{subColor}{RGB}{170, 149, 0}
\newcommand{\mySub}[1]{{\color{subColor}\hspace{-9mm} \textsf{#1}}}
\definecolor{keyColor}{RGB}{118, 185, 0}
\newcommand{\myKey}[1]{{\color{keyColor}\textbf{#1}}}

\definecolor{keyColorA}{RGB}{170, 149, 0}
\definecolor{keyColorB}{RGB}{170, 210, 0}
\newcommand{\myKeyA}[1]{\textbf{\color{keyColorA}#1}}
\newcommand{\myKeyB}[1]{\textbf{\color{keyColorB}#1}}

\definecolor{mainColor}{RGB}{170, 0, 170}
\definecolor{minorColor}{RGB}{0, 149, 232}
\newcommand{\mainKey}[1]{\textbf{\color{mainColor}#1}}
\newcommand{\minorKey}[1]{\textbf{\color{minorColor}#1}}
\newcommand{\mainSub}[1]{{\color{mainColor}\hspace{-9mm} \textsf{#1}}}
\newcommand{\minorSub}[1]{{\color{minorColor}\hspace{-9mm} \textsf{#1}}}

\definecolor{hookColor}{RGB}{204, 0, 255}
\newcommand{\hook}[1]{{\color{hookColor}\textbf{#1}}}
\definecolor{flaggedBoxColor}{RGB}{255, 204, 204}
\newcommand{\flagged}[1]{{\color{black}\colorbox{flaggedBoxColor}{#1}}}

\definecolor{lightttColor}{RGB}{69, 69, 80}
\newcommand{\lighttt}[1]{{\color{lightttColor}\texttt{#1}}}
\newcommand{\graytt}[1]{{\color{gray}\texttt{#1}}}
\newcommand{\blacktt}[1]{{\color{black}\texttt{#1}}}

% Red and blue boxes have bright text color.
% Yellow, green, and violet have intentionally muted text colors.
% They used to be black, but it looks slightly better with a tiny bit of color.
\definecolor{redBoxFg}{RGB}{224, 0, 0}
\definecolor{redBoxBg}{RGB}{255, 211, 225}
\newcommand{\redBox}[1]{{\color{redBoxFg}\colorbox{redBoxBg}{#1}}}
\definecolor{yellowBoxFg}{RGB}{89, 89, 0}
\definecolor{yellowBoxBg}{RGB}{255, 232, 0}
\newcommand{\yellowBox}[1]{{\color{yellowBoxFg}\colorbox{yellowBoxBg}{#1}}}
\definecolor{greenBoxFg}{RGB}{0, 89, 63}
\definecolor{greenBoxBg}{RGB}{179, 232, 160}
\newcommand{\greenBox}[1]{{\color{greenBoxFg}\colorbox{greenBoxBg}{#1}}}
\definecolor{blueBoxFg}{RGB}{0, 97, 232}
\definecolor{blueBoxBg}{RGB}{224, 232, 255}
\newcommand{\blueBox}[1]{{\color{blueBoxFg}\colorbox{blueBoxBg}{#1}}}
\definecolor{violetBoxFg}{RGB}{108, 63, 124}
\definecolor{violetBoxBg}{RGB}{218, 204, 255}
\newcommand{\violetBox}[1]{{\color{violetBoxFg}\colorbox{violetBoxBg}{#1}}}

\mdfdefinestyle{MyFrame}{%
    linecolor=black,
    outerlinewidth=0pt,
    linewidth=0pt,
    innertopmargin=2.7pt,
    innerbottommargin=0pt,
    innerrightmargin=0pt,
    innerleftmargin=0pt,
        leftmargin = 0pt,
        rightmargin = 0pt}


\renewcommand*{\theenumi}{\alph{enumi}}
\renewcommand*\labelenumi{(\theenumi)}
\renewcommand*{\theenumii}{\roman{enumii}}
\renewcommand*\labelenumii{\theenumii.}

\sloppy
\newcommand{\abs}[1]{\left \vert #1 \right\vert}

% Some of these links are now dead...
\newcommand{\mbarrier}{\webText{mbarrier}{https://docs.nvidia.com/cuda/parallel-thread-execution/\#parallel-synchronization-and-communication-instructions-mbarrier}}
\newcommand{\cpAsync}{\webText{cp.async}{https://docs.nvidia.com/cuda/parallel-thread-execution/\#data-movement-and-conversion-instructions-cp-async}}
\newcommand{\cpAsyncBulk}{\webText{cp.async.bulk}{https://docs.nvidia.com/cuda/parallel-thread-execution/\#data-movement-and-conversion-instructions-cp-async-bulk}}
\newcommand{\fenceProxyAsync}{\webText{fence.proxy.async}{https://docs.nvidia.com/cuda/parallel-thread-execution/\#asynchronous-warpgroup-level-matrix-async-proxy}}
\newcommand{\wgmma}{\webText{wgmma}{https://docs.nvidia.com/cuda/parallel-thread-execution/\#asynchronous-warpgroup-level-matrix-instructions}}
\newcommand{\hopperBlog}{\webText{NVIDIA Hopper Architecture In-Depth}{https://developer.nvidia.com/blog/nvidia-hopper-architecture-in-depth/}}
\newcommand{\expectTxOperation}{\webText{expect-tx operation}{https://docs.nvidia.com/cuda/parallel-thread-execution/\#parallel-synchronization-and-communication-instructions-mbarrier-expect-tx-operation}}
\newcommand{\completeTxOperation}{\webText{complete-tx operation}{https://docs.nvidia.com/cuda/parallel-thread-execution/\#parallel-synchronization-and-communication-instructions-mbarrier-complete-tx-operation}}
\newcommand{\wgmmaFence}{\webText{wgmma.fence}{https://docs.nvidia.com/cuda/parallel-thread-execution/\#asynchronous-warpgroup-level-matrix-instructions-wgmma-fence}}


\begin{document}
\raggedright

For the f16 gemm testbed, you need to clone the following:

\texttt{git@github.com:akeley98/sporkbench.git} \codecomment{\# or pull latest main}

\texttt{git@github.com:NVIDIA/cutlass.git}

Check out the Exo repo and pull and branch off the \lighttt{akeley98/spork26} branch.

Currently I have a really crappy float16 gemm implemented in sporkbench at \lighttt{examples/f16\_experiment/exocc\_Sm80\_naive\_gemm.py}.\\
Benchmark it against cublas and cutlass on Ubuchan:

\input{f16_gemm/run_sporkbench.0.tex}

Your task is to implement the \lighttt{cp.async}, \lighttt{ldmatrix}, and \lighttt{mma} (Ampere tensor core) instructions, upgrade the simple gemm using these new instructions, and finally write a new gemm with split barriers that performs similarly to the cutlass gemm.

\section{Cutlass Reference Code}

Sporkbench saved the intermediate PTX files in the bin directory.
Grep is your best friend for cutting this file down to the useful parts:

\input{f16_gemm/grep.0.tex}

\section{Packed Registers}

As I mentioned, the tensor cores expect 2 f16 values packed inside each 32-bit register.
The new \lighttt{exo.cuda.CudaRmemPacked32} memory type models this in Exo.
For your use case, you need to allocate tensors with \lighttt{f16[..., 2]} type, with the last ``2'' modeling the two scalars packed per register.

\section{cp.async}

This should be the easiest instruction to implement.
Adapt \lighttt{exo.platforms.Sm80.Sm80\_cp\_async\_f32} to be for \lighttt{f16} instead.
You should recycle the \lighttt{cp\_async\_impl} base class already defined.

This provides a simple example of how instruction templates (classes) are defined in Exo-GPU.
This instruction uses \lighttt{instr\_tl = Sm80\_cp\_async\_instr}.
All your other instructions (for Ampere) will use \lighttt{instr\_tl = cuda\_in\_order\_instr}.

You can test it by substituting it in-place of the code in the naïve gemm that fills the shared memory \lighttt{s\_A} and \lighttt{s\_B}.
Update the first fence to \lighttt{Fence(Sm80\_cp\_async, cuda\_in\_order)}.
Don't change the second fence (I changed the synchronization rules on you).

\section{ldmatrix}

\web{docs.nvidia.com/cuda/parallel-thread-execution/\#warp-level-matrix-instructions-ldmatrix}

The linked documentation for ldmatrix is real shitty.
It implicitly assumes that matrices are row major (even though the MMA and wgmma instrs that consume this data \emph{require} column major in many cases).
Also it just uses the term ``a matrix'' without a clear definition.
For clarity, I'll use my own terms:
\begin{itemize}
  \item L-row: this is 16 aligned bytes of data.
    If the source matrix is row-major, this is 8 float16 values from a row.
    Otherwise, this is 8 float16 values from a column.
    Note, currently Exo has no way to reason about alignment requirements, so you'll have to take it on faith that the SMEM source pointer is 16-byte aligned.
  \item L-matrix: this is 8 L-rows.
    If the 8 L-rows are loaded from 8 consecutive rows/columns in memory (for row/column major respectively), then an L-matrix stores an $8 \times 8$ tile of float16 values.
\end{itemize}
Each L-row is stored in 1 register each of 4 adjacent threads.
Each L-matrix is stored in 1 register each of the 32 threads (lanes) in a warp, with lanes $4n, 4n+1, 4n+2, 4n+3$ in the warp holding the $n^{th}$ L-row (0-indexed).

In the \lighttt{codegen} of the function, you have to use \lighttt{InlinePtxGen} to wrap the \lighttt{ldmatrix.sync.aligned.x4.m8n8.shared.b16} instruction.
This loads 4 L-matrices, each stored with its own register ID.
The input register of threads $8n, 8n + 1, ..., 8n + 7$ stores the 8 L-row pointers of the $n^{th}$ L-matrix.
Look at the Cutlass PTX listing from before to see correct usage.

The behavior (I think???) of the instruction is

\input{f16_gemm/ldmatrix_behavior.0.tex}

Note that the dimensions of the \lighttt{rmem} parameter were re-ordered so the distributed dimensions are to the left (Exo-GPU requires this).
You will need to set the \lighttt{exo.core.instr\_info.AccessInfo.distributed\_coll\_units} and \lighttt{exo.core.instr\_info.InstrInfo.coll\_unit} attributes to configure this distributed memory.

For now we'll implement GEMM using the MMA instruction's native \lighttt{row.col} mode (row-major A, column-major B).
The Exo params for the gemm are \lighttt{A[M, K], B[N, K]}.
So you can use \lighttt{ldmatrix} without the transpose (\lighttt{.trans}).
For the B matrix, when \lighttt{ldmatrix} loads an L-row, it's really a column of 16 bytes (8 float16 values).

\section{MMA}

\web{docs.nvidia.com/cuda/parallel-thread-execution/\#warp-level-matrix-fragment-mma-16816-float}

Like ldmatrix, you will have to again write a warp-convergent instruction.
This time you will wrap \lighttt{mma.sync.aligned.m16n8k16.row.col.f16.f16.f16.f16}.
The float16 usage requires \lighttt{f16x2}-typed PTX registers; note the following when writing \lighttt{codegen} for the instruction:
\begin{itemize}
  \item When generating the PTX, use \lighttt{constraint="r"}.
  \item Pass \lighttt{ptx\_data=True} when calling \lighttt{args.$x$.index()}.
    This generates a C++ array indexing expression that conveys the int that packed the 2 floats together.
    See \lighttt{exo.platforms.cuda.cuda\_packed\_load\_base} to see what I mean.
  \item You may want to read the instruction codegen section of \lighttt{spork\_b.pdf}.
\end{itemize}
In the linked PTX documentation, you will see that the input A/B (multiplicand) consists of L-matrices.
On the other hand, if you scroll down to the output C (accumulator) fragment format, you'll see that it's completely different.
Once you substitute the mma instruction into the gemm, to write out the accumulator (C), you will have to use a mixture of \lighttt{seq} and \lighttt{cuda\_threads} loops to unpack the registers and store to GMEM.
For Ampere there's no \lighttt{stmatrix} instruction, just do it yourself.

\section{Split Barriers}

Once you get the custom instructions working, you can move on to optimizing the algorithm itself.
To match the Cutlass GEMM, you will have to use commit group synchronization.
The state of this commit group is per-thread, i.e., each thread can only use this to wait for its own \lighttt{cp.async} instructions.
So in Exo-GPU, this will be modelled as distributed memory:

\input{f16_gemm/commit_group.0.tex}

Typically you will follow with a \lighttt{Fence(cuda\_in\_order, cuda\_in\_order)} at CTA-scope (``syncthreads'').
See \lighttt{exo.tests.cuda.test\_cuda\_sync.mkproc\_cp\_async\_commit\_group} for more example code.

The pseudocode for the cutlass algorithm seems to be

\input{f16_gemm/cutlass_pseudocode.0.tex}

It's up to you to further analyze the cutlass algorithm and fill in the details (in fact I could be mistaken in this outline).
Another thing needed to match cutlass is to write the C register outputs to SMEM first (``epilogue''), then copy the SMEM to GMEM using vectorized store instructions (\lighttt{st.global.v4.u32}).
I'm not so sure how important this really is; I leave it up to you whether you want to work on this.

\end{document}
