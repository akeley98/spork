%python3 code_to_tex.py guide_correct.py guide && python3 code_to_tex.py guide_wrong.py guide && xelatex spork_guide.tex < /dev/null
\documentclass[11pt]{article}

\usepackage[letterpaper, portrait, margin=2.54cm]{geometry}
\usepackage{enumitem}
\usepackage{amsmath}
\usepackage{amssymb}
\usepackage{amsfonts}
\usepackage{placeins}
\usepackage{graphicx}
\usepackage{listings}
\usepackage{caption}
\usepackage{colortbl}
\usepackage[parfill]{parskip}
\usepackage[mathscr]{euscript}
\usepackage[usenames,dvipsnames,svgnames,table,hyperref]{xcolor}
\usepackage[hidelinks]{hyperref}
\usepackage{fontspec}
\usepackage{mdframed}
\usepackage{tikz}
\usetikzlibrary{shapes.geometric, arrows, positioning, calc}

\setsansfont{FreeSans}
\setmonofont{Ubuntu Mono}

\hyphenation{WebGL}

%\definecolor{webColor}{RGB}{0, 124, 204}
\definecolor{webColor}{RGB}{0, 108, 174}
\newcommand{\web}[1]{{\color{webColor} \small \url{#1}}}
\newcommand{\webText}[2]{{\color{webColor} \href{#2}{#1}}}
\newcommand{\email}[2]{{\small \color{webColor} \textsf{\href{mailto:#1@#2}{#1[at]#2}}}}
\definecolor{titleColor}{RGB}{0, 0, 0}
\ifdefined\RESUME
\newcommand{\myTitle}[1]{{\large \color{titleColor} \hspace{-12mm} \textbf{\textsf{#1}}}}
\else
\newcommand{\myTitle}[1]{{ \vspace{2mm} \large \color{titleColor} \hspace{-12mm} \textbf{\textsf{#1}} \vspace{2mm}}}
\fi
\definecolor{subColor}{RGB}{170, 149, 0}
\newcommand{\mySub}[1]{{\color{subColor}\hspace{-9mm} \textsf{#1}}}
\definecolor{keyColor}{RGB}{118, 185, 0}
\newcommand{\myKey}[1]{{\color{keyColor}\textbf{#1}}}

\definecolor{keyColorA}{RGB}{170, 149, 0}
\definecolor{keyColorB}{RGB}{170, 210, 0}
\newcommand{\myKeyA}[1]{\textbf{\color{keyColorA}#1}}
\newcommand{\myKeyB}[1]{\textbf{\color{keyColorB}#1}}

\definecolor{mainColor}{RGB}{170, 0, 170}
\definecolor{minorColor}{RGB}{0, 149, 232}
\newcommand{\mainKey}[1]{\textbf{\color{mainColor}#1}}
\newcommand{\minorKey}[1]{\textbf{\color{minorColor}#1}}
\newcommand{\mainSub}[1]{{\color{mainColor}\hspace{-9mm} \textsf{#1}}}
\newcommand{\minorSub}[1]{{\color{minorColor}\hspace{-9mm} \textsf{#1}}}

\definecolor{hookColor}{RGB}{204, 0, 255}
\newcommand{\hook}[1]{{\color{hookColor}\textbf{#1}}}
\definecolor{flaggedBoxColor}{RGB}{255, 204, 204}
\newcommand{\flagged}[1]{{\color{black}\colorbox{flaggedBoxColor}{#1}}}

\definecolor{lightttColor}{RGB}{69, 69, 80}
\definecolor{grayttColor}{RGB}{144, 144, 160}
\newcommand{\lighttt}[1]{{\color{lightttColor}\texttt{#1}}}
\newcommand{\graytt}[1]{{\color{grayttColor}\texttt{#1}}}
\newcommand{\blacktt}[1]{{\color{black}\texttt{#1}}}

% Red and blue boxes have bright text color.
% Yellow, green, and violet have intentionally muted text colors.
% They used to be black, but it looks slightly better with a tiny bit of color.
\definecolor{redBoxFg}{RGB}{224, 0, 0}
\definecolor{redBoxBg}{RGB}{255, 211, 225}
\newcommand{\redBox}[1]{{\color{redBoxFg}\colorbox{redBoxBg}{#1}}}
\definecolor{yellowBoxFg}{RGB}{89, 89, 0}
\definecolor{yellowBoxBg}{RGB}{255, 232, 0}
\newcommand{\yellowBox}[1]{{\color{yellowBoxFg}\colorbox{yellowBoxBg}{#1}}}
\definecolor{greenBoxFg}{RGB}{0, 89, 63}
\definecolor{greenBoxBg}{RGB}{179, 232, 160}
\newcommand{\greenBox}[1]{{\color{greenBoxFg}\colorbox{greenBoxBg}{#1}}}
\definecolor{blueBoxFg}{RGB}{0, 97, 232}
\definecolor{blueBoxBg}{RGB}{224, 232, 255}
\newcommand{\blueBox}[1]{{\color{blueBoxFg}\colorbox{blueBoxBg}{#1}}}
\definecolor{violetBoxFg}{RGB}{108, 63, 124}
\definecolor{violetBoxBg}{RGB}{218, 204, 255}
\newcommand{\violetBox}[1]{{\color{violetBoxFg}\colorbox{violetBoxBg}{#1}}}

\definecolor{codecommentColor}{RGB}{124, 170, 0}
\newcommand{\codecomment}[1]{{\color{codecommentColor}\textsf{#1}}}

\mdfdefinestyle{MyFrame}{%
    linecolor=black,
    outerlinewidth=0pt,
    linewidth=0pt,
    innertopmargin=2.7pt,
    innerbottommargin=0pt,
    innerrightmargin=0pt,
    innerleftmargin=0pt,
        leftmargin = 0pt,
        rightmargin = 0pt}


\renewcommand*{\theenumi}{\alph{enumi}}
\renewcommand*\labelenumi{(\theenumi)}
\renewcommand*{\theenumii}{\roman{enumii}}
\renewcommand*\labelenumii{\theenumii.}

\newcommand{\fixminipage}{\raggedright \setlength{\parskip}{0.3\baselineskip}}
\newcommand{\codeminipage}{\raggedright \setlength{\parskip}{0\baselineskip}}

\sloppy
\newcommand{\abs}[1]{\left \vert #1 \right\vert}

% Some of these links are now dead...
\newcommand{\mbarrier}{\webText{mbarrier}{https://docs.nvidia.com/cuda/parallel-thread-execution/\#parallel-synchronization-and-communication-instructions-mbarrier}}
\newcommand{\cpAsync}{\webText{cp.async}{https://docs.nvidia.com/cuda/parallel-thread-execution/\#data-movement-and-conversion-instructions-cp-async}}
\newcommand{\cpAsyncBulk}{\webText{cp.async.bulk}{https://docs.nvidia.com/cuda/parallel-thread-execution/\#data-movement-and-conversion-instructions-cp-async-bulk}}
\newcommand{\fenceProxyAsync}{\webText{fence.proxy.async}{https://docs.nvidia.com/cuda/parallel-thread-execution/\#asynchronous-warpgroup-level-matrix-async-proxy}}
\newcommand{\wgmma}{\webText{wgmma}{https://docs.nvidia.com/cuda/parallel-thread-execution/\#asynchronous-warpgroup-level-matrix-instructions}}
\newcommand{\hopperBlog}{\webText{NVIDIA Hopper Architecture In-Depth}{https://developer.nvidia.com/blog/nvidia-hopper-architecture-in-depth/}}
\newcommand{\expectTxOperation}{\webText{expect-tx operation}{https://docs.nvidia.com/cuda/parallel-thread-execution/\#parallel-synchronization-and-communication-instructions-mbarrier-expect-tx-operation}}
\newcommand{\completeTxOperation}{\webText{complete-tx operation}{https://docs.nvidia.com/cuda/parallel-thread-execution/\#parallel-synchronization-and-communication-instructions-mbarrier-complete-tx-operation}}
\newcommand{\wgmmaFence}{\webText{wgmma.fence}{https://docs.nvidia.com/cuda/parallel-thread-execution/\#asynchronous-warpgroup-level-matrix-instructions-wgmma-fence}}


\tikzstyle{smallnode} = [rectangle, minimum width=1.25cm, minimum height=1cm, text centered, text width=1.25cm, draw=black, fill=white]
\tikzstyle{memnode} = [rectangle, minimum width=2.25cm, minimum height=1cm, text centered, text width=2.25cm, draw=greenBoxFg, fill=greenBoxBg, text=greenBoxFg]
\tikzstyle{smallishnode} = [rectangle, minimum width=2cm, minimum height=1cm, text centered, text width=2cm, draw=black, fill=white]
\tikzstyle{normalnode} = [rectangle, minimum width=3cm, minimum height=1cm, text centered, text width=3cm, draw=black, fill=white]
\tikzstyle{widenode} = [rectangle, minimum width=62mm, minimum height=8mm, text centered, text width=62mm, draw=black, fill=white]
\tikzstyle{bignode} = [rectangle, minimum width=3.5cm, minimum height=2cm, text centered, text width=3cm, draw=black, fill=white]
\tikzstyle{smemnode} = [rectangle, minimum width=3cm, minimum height=1cm, text centered, text width=3cm, draw=keyColorB, fill=white]
\tikzstyle{gmemnode} = [rectangle, minimum width=3cm, minimum height=1cm, text centered, text width=3cm, draw=keyColorA, fill=white]
\tikzstyle{smallishsmemnode} = [rectangle, minimum width=2cm, minimum height=1cm, text centered, text width=2cm, draw=keyColorB, fill=white]
\tikzstyle{arrow} = [thick,->,>=stealth]
\tikzstyle{line} = [thick]

\begin{document}
\myTitle{Exo GPU -- Spork User Guide}

We're extending Exo with constructs for generating CUDA device code, with the goals of supporting
\begin{itemize}
  \item explicit instruction selection, including support for asynchronous copies and tensor cores (wgmma)
  \filbreak
  \item explicit selection of memory types for variables
  \filbreak
  \item explicit assignment of work to threads (no implied parallelism; Exo's imperative C-style programming is preserved) and explicit synchronization (choice of barriers, mbarriers, commit group)
  \filbreak
  \item codegen to CUDA C++ headers and CPU C code, which will allow us to support mixed CPU/GPU code, and mixed Exo-generated and handwritten CUDA
  \filbreak
  \item ``sequential logic'' for scheduling; that is, Exo's existing analysis continues to use sequential semantics;
    all parallelism features are expressed using existing concepts with ``annotations'' (e.g. parallel-for loops) that can be ignored to analyze the program as a sequential program
  \filbreak
\item S/M equivalence (single-threaded / multi-threaded equivalence); we check that the scheduled program, interpreted sequentially, will give the same results as the scheduled program interpreted as a parallel CUDA program.
    This two phase checking (Exo scheduling with sequential semantics, S/M equivalence checking) proves that the unscheduled sequential program gives the same results as the final parallelized program.
\end{itemize}
\filbreak
In the initial product, we will support checking only for concrete problem sizes, using a per-memory-access simulator.
\filbreak
\begin{tikzpicture}[node distance=8mm]
\sffamily
\node(proc0) [normalnode, text width=2cm, minimum width=2cm] {Original proc};
\node(procNS) [normalnode, right=of proc0, text width=4cm, minimum width=4cm] {Scheduled proc, interpreted \myKeyA{sequentially}};
\node(procNM) [normalnode, right=of procNS, text width=4cm, minimum width=4cm] {Scheduled proc, \myKeyB{parallel} interpretation};
\node(cuda) [normalnode, right=of procNM] {CUDA C++ Header};

\draw [arrow] (proc0) to node(exo)[]{} (procNS);
\draw [arrow] (procNS) to node(SM)[]{} (procNM);
\draw [arrow] (procNM) to node(toCuda)[]{} (cuda);

\node(exo) [normalnode, text width=4cm, minimum width=4cm, below=of exo, draw=blueBoxFg, fill=blueBoxBg, text=blueBoxFg] {Existing Exo Rewrites};
\node(sync) [normalnode, text width=5cm, below=of SM, draw=blueBoxFg, fill=blueBoxBg, text=blueBoxFg] {S/M equivalence check: ``above the waterline''};
\node(spork) [normalnode, text width=6cm, xshift=1cm, below=of toCuda, draw=blueBoxFg, fill=blueBoxBg, text=blueBoxFg] {Spork Backend Compiler: ``below the waterline'' (not proven correct)};
\node(note) [above=of SM, draw=redBoxFg, fill=redBoxBg, text=redBoxFg, xshift=15mm] {Physically same proc, \textbf{different interpretation}};
\node(caption) [left=of note] {``\textbf{chain of equivalence}''};
\node(c) [smallnode, right=of note] {C code};

\draw [line, draw=blueBoxFg] (exo) -- (proc0);
\draw [line, draw=blueBoxFg] (exo) -- (procNS);
\draw [line, draw=blueBoxFg] (sync) -- (procNS);
\draw [line, draw=blueBoxFg] (sync) -- (procNM);
\draw [line, draw=redBoxFg] (note) -- (procNS);
\draw [line, draw=redBoxFg] (note) -- (procNM);
\draw [line, draw=blueBoxFg] (spork) -- (procNM);
\draw [line, draw=blueBoxFg] (spork) -- (cuda);
\draw [arrow] (procNM.east) to (c.west);
\end{tikzpicture}

\filbreak
\myTitle{Fundamental Language Concepts}

\filbreak
\myTitle{Object Code Features}

\filbreak
\myTitle{Abstract Machine (Synchronization Model)}

\end{document}
