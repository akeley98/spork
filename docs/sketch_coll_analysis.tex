% xelatex </dev/null sketch_coll_analysis.tex

\documentclass[11pt]{article}

\usepackage[letterpaper, portrait, margin=2.54cm]{geometry}
\usepackage{enumitem}
\usepackage{amsmath}
\usepackage{amssymb}
\usepackage{amsfonts}
\usepackage{placeins}
\usepackage{graphicx}
\usepackage{listings}
\usepackage{caption}
\usepackage{colortbl}
\usepackage[parfill]{parskip}
\usepackage[mathscr]{euscript}
\usepackage[usenames,dvipsnames,svgnames,table,hyperref]{xcolor}
\usepackage[hidelinks]{hyperref}
\usepackage{fontspec}
\usepackage{mdframed}
\usepackage{tikz}
\usetikzlibrary{shapes.geometric, arrows, positioning, calc}

\setsansfont{FreeSans}
\setmonofont{Ubuntu Mono}

\hyphenation{WebGL}

%\definecolor{webColor}{RGB}{0, 124, 204}
\definecolor{webColor}{RGB}{0, 108, 174}
\newcommand{\web}[1]{{\color{webColor} \small \url{#1}}}
\newcommand{\webText}[2]{{\color{webColor} \href{#2}{#1}}}
\newcommand{\email}[2]{{\small \color{webColor} \textsf{\href{mailto:#1@#2}{#1[at]#2}}}}
\definecolor{titleColor}{RGB}{0, 0, 0}
\ifdefined\RESUME
\newcommand{\myTitle}[1]{{\large \color{titleColor} \hspace{-12mm} \textbf{\textsf{#1}}}}
\else
\newcommand{\myTitle}[1]{{ \vspace{2mm} \large \color{titleColor} \hspace{-12mm} \textbf{\textsf{#1}} \vspace{2mm}}}
\fi
\definecolor{subColor}{RGB}{170, 149, 0}
\newcommand{\mySub}[1]{{\color{subColor}\hspace{-9mm} \textsf{#1}}}
\definecolor{keyColor}{RGB}{118, 185, 0}
\newcommand{\myKey}[1]{{\color{keyColor}\textbf{#1}}}

\definecolor{keyColorA}{RGB}{170, 149, 0}
\definecolor{keyColorB}{RGB}{170, 210, 0}
\newcommand{\myKeyA}[1]{\textbf{\color{keyColorA}#1}}
\newcommand{\myKeyB}[1]{\textbf{\color{keyColorB}#1}}

\definecolor{mainColor}{RGB}{170, 0, 170}
\definecolor{minorColor}{RGB}{0, 149, 232}
\newcommand{\mainKey}[1]{\textbf{\color{mainColor}#1}}
\newcommand{\minorKey}[1]{\textbf{\color{minorColor}#1}}
\newcommand{\mainSub}[1]{{\color{mainColor}\hspace{-9mm} \textsf{#1}}}
\newcommand{\minorSub}[1]{{\color{minorColor}\hspace{-9mm} \textsf{#1}}}

\definecolor{hookColor}{RGB}{204, 0, 255}
\newcommand{\hook}[1]{{\color{hookColor}\textbf{#1}}}
\definecolor{flaggedBoxColor}{RGB}{255, 204, 204}
\newcommand{\flagged}[1]{{\color{black}\colorbox{flaggedBoxColor}{#1}}}

\definecolor{lightttColor}{RGB}{69, 69, 80}
\definecolor{grayttColor}{RGB}{144, 144, 160}
\newcommand{\lighttt}[1]{{\color{lightttColor}\texttt{#1}}}
\newcommand{\graytt}[1]{{\color{grayttColor}\texttt{#1}}}
\newcommand{\blacktt}[1]{{\color{black}\texttt{#1}}}

% Red and blue boxes have bright text color.
% Yellow, green, and violet have intentionally muted text colors.
% They used to be black, but it looks slightly better with a tiny bit of color.
\definecolor{redBoxFg}{RGB}{224, 0, 0}
\definecolor{redBoxBg}{RGB}{255, 211, 225}
\newcommand{\redBox}[1]{{\color{redBoxFg}\colorbox{redBoxBg}{#1}}}
\definecolor{yellowBoxFg}{RGB}{89, 89, 0}
\definecolor{yellowBoxBg}{RGB}{255, 232, 0}
\newcommand{\yellowBox}[1]{{\color{yellowBoxFg}\colorbox{yellowBoxBg}{#1}}}
\definecolor{greenBoxFg}{RGB}{0, 89, 63}
\definecolor{greenBoxBg}{RGB}{179, 232, 160}
\newcommand{\greenBox}[1]{{\color{greenBoxFg}\colorbox{greenBoxBg}{#1}}}
\definecolor{blueBoxFg}{RGB}{0, 97, 232}
\definecolor{blueBoxBg}{RGB}{224, 232, 255}
\newcommand{\blueBox}[1]{{\color{blueBoxFg}\colorbox{blueBoxBg}{#1}}}
\definecolor{violetBoxFg}{RGB}{108, 63, 124}
\definecolor{violetBoxBg}{RGB}{218, 204, 255}
\newcommand{\violetBox}[1]{{\color{violetBoxFg}\colorbox{violetBoxBg}{#1}}}

\definecolor{codecommentColor}{RGB}{124, 170, 0}
\newcommand{\codecomment}[1]{{\color{codecommentColor}\textsf{#1}}}

\mdfdefinestyle{MyFrame}{%
    linecolor=black,
    outerlinewidth=0pt,
    linewidth=0pt,
    innertopmargin=2.7pt,
    innerbottommargin=0pt,
    innerrightmargin=0pt,
    innerleftmargin=0pt,
        leftmargin = 0pt,
        rightmargin = 0pt}


\renewcommand*{\theenumi}{\alph{enumi}}
\renewcommand*\labelenumi{(\theenumi)}
\renewcommand*{\theenumii}{\roman{enumii}}
\renewcommand*\labelenumii{\theenumii.}

\newcommand{\fixminipage}{\raggedright \setlength{\parskip}{0.3\baselineskip}}
\newcommand{\codeminipage}{\raggedright \setlength{\parskip}{0\baselineskip}}

\sloppy
\newcommand{\abs}[1]{\left \vert #1 \right\vert}

% Some of these links are now dead...
\newcommand{\mbarrier}{\webText{mbarrier}{https://docs.nvidia.com/cuda/parallel-thread-execution/\#parallel-synchronization-and-communication-instructions-mbarrier}}
\newcommand{\cpAsync}{\webText{cp.async}{https://docs.nvidia.com/cuda/parallel-thread-execution/\#data-movement-and-conversion-instructions-cp-async}}
\newcommand{\cpAsyncBulk}{\webText{cp.async.bulk}{https://docs.nvidia.com/cuda/parallel-thread-execution/\#data-movement-and-conversion-instructions-cp-async-bulk}}
\newcommand{\fenceProxyAsync}{\webText{fence.proxy.async}{https://docs.nvidia.com/cuda/parallel-thread-execution/\#asynchronous-warpgroup-level-matrix-async-proxy}}
\newcommand{\wgmma}{\webText{wgmma}{https://docs.nvidia.com/cuda/parallel-thread-execution/\#asynchronous-warpgroup-level-matrix-instructions}}
\newcommand{\hopperBlog}{\webText{NVIDIA Hopper Architecture In-Depth}{https://developer.nvidia.com/blog/nvidia-hopper-architecture-in-depth/}}
\newcommand{\expectTxOperation}{\webText{expect-tx operation}{https://docs.nvidia.com/cuda/parallel-thread-execution/\#parallel-synchronization-and-communication-instructions-mbarrier-expect-tx-operation}}
\newcommand{\completeTxOperation}{\webText{complete-tx operation}{https://docs.nvidia.com/cuda/parallel-thread-execution/\#parallel-synchronization-and-communication-instructions-mbarrier-complete-tx-operation}}
\newcommand{\wgmmaFence}{\webText{wgmma.fence}{https://docs.nvidia.com/cuda/parallel-thread-execution/\#asynchronous-warpgroup-level-matrix-instructions-wgmma-fence}}


\begin{document}
\myTitle{Exo-GPU Collective Analysis (CollAnalysis)}

We begin by summarizing the types involved.
Many objects are defined here as functions, but they contain internal state that may be inspected (i.e. they are not just black-box functions).

\filbreak
\mainKey{Statement:} A statement within a program, akin to ``static instruction'' in computer architecture.

\filbreak
\mainKey{Statement Instance:} A given execution/interpretation of a statement, akin to ``dynamic instruction'' in computer architecture.

\filbreak
\mainKey{Device Scope:} CPU or CUDA.

\filbreak
\mainKey{Natural Thread Index:} Integer uniquely identifying a thread within a CUDA cluster, given by \lighttt{cluster\_ctarank * blockDim.x + threadIdx.x}.
Exo only uses the $x$ block/thread dimension, and only has static analysis of thread assignment within clusters -- the mapping between Exo tasks and CUDA clusters is not statically analyzed.

\filbreak
\mainKey{Thread Collective:} $(\mu \in \mathbb{T})$ A set of threads within a cluster assigned to execute a specific statement instance.
Threads are identified by natural thread index.
Runtime concept.
The top-level-collective $\mu^\text{top}$ consists of all threads in the cluster.

For a given \lighttt{cuda\_threads} loop with at least two iterations, let $\mu_0$ and $\mu_1$ be thread collectives assigned to two consecutive iterations.
The \myKeyA{thread pitch} is $\min(\mu_1) - \min(\mu_0)$
(c.f. ``seat pitch''; we are measuring the distance between adjacent things).
The thread pitch is 0 for $\le 1$ iteration loops.

\filbreak
\mainKey{Domain:} A tuple of natural numbers describing a multidimensional organization of threads in the cluster.
Let $M$ be the dimension, and $D: \mathbb{N}^M$ be the domain.
Then the product of the domain coordinates $(D_1) ...(D_M)$ equals \lighttt{blockDim * clusterDim}.

Let $\mathbb{N}^M_\delta$ = $[0, D_1 - 1]_\mathbb{N} \times ... \times [0, D_M - 1]_\mathbb{N}$.
This is an $M$-dimensional grid of coordinates each uniquely identifying a thread in a cluster.

The domain defines a ``linearize'' function $\ell_D: \mathbb{N}^M_\delta \to \mu^\text{top}$ mapping multidimensional coordinates to natural thread indices in lexicographical order.

\filbreak
\mainKey{Dimension Thread Pitch Set:} The dimension thread pitch set of a domain $D: \mathbb{N}^M$ is $\{ \prod_{i=m+1}^M D_i \mid i = 1 ... M \}$.

\filbreak
\mainKey{Collective Type:} $(\delta \in \Delta)$ A ``unit of measure'' describing a certain number and arrangement of threads (e.g. ``warp'' or ``CTA'') within one CUDA cluster.
An $M$-dimensional $\delta$ is a 2-tuple of domain $D: \mathbb{N}^M$ and box $B: \mathbb{N}_\top^M$; the product of the domain coordinates equals the number of threads in the cluster.
$\mathbb{N}_\top = \mathbb{N} \cup \{\top\}$.

$\mathbb{T}_\delta$ denotes the set of all thread collectives that are described by (match) the collective type $\delta$.

\filbreak
\mainKey{Collective Unit:} $(\tau_u \in \mathsf{CollUnit})$
Frontend concept, used to parameterize \lighttt{cuda\_threads} loops and hardware instructions.
This contains a collective type $\delta$ (in the language, these contain \lighttt{clusterDim} and \lighttt{blockDim} which are immediately substituted for concrete values).

\filbreak
\mainKey{Thread Mapping:} $\mathcal{F}: ((\mathbb{Y} \to \mathbb{Z}) \to \mathbb{T})$
Description of the static assignment of parallel loop iterations to thread collectives.
So basically the control environment determines which parallel loop iteration is being executed, and this gets translated to a thread collective for execution.
The output of collective analysis is to annotate each statement at CUDA scope with a collective assignment function.

\filbreak
\mainKey{Collective Tiling:} $\omega \in \Omega$.
This is a tuple of $\mathcal{D}$ (dimension descriptor; to be defined later); with $M$ being the length of the tuple, this is a description of a division of the cluster's threads into an $M$-dimensional space and which parallel loop iterators iterate upon each dimension.
Underlying data for a thread mapping.
This $\omega$ can be converted to a thread mapping, denoted $f_\omega$, which is the \myKeyA{resulting function} of a collective tiling.


\filbreak
\myTitle{Scoped State}

Each scope either has (or is a) CPU or CUDA device scope.

\filbreak
A CUDA scope has additional attributes:
\begin{itemize}
  \item \lighttt{clusterDim} \& \lighttt{blockDim} collective parameter values (uniform throughout the device function)
  \item collective tiling: $\omega: \Omega$, convertible to $(\mathbb{Y} \to \mathbb{Z}) \to \mathbb{T}$ (for the paper we may view the thread mapping itself as the attribute intsead)
\end{itemize}

\filbreak
Note, the collective tiling is a \textit{static} property of a statement; evaluating its resulting function (passing in $\sigma_c$) yields a thread collective, which is a \textit{dynamic} property of a statement instance.

\filbreak
For a given $\tau_u : \mathsf{CollUnit}$, we say that a scope is a $\tau_u$-scope when its collective tiling always has output that matches $\delta$ where $\delta$ is the collective type held by $\tau_u$.
In other words, this is the case when we can accurately type the collective tiling's resulting function as $(\mathbb{Y} \to \mathbb{Z}) \to \mathbb{T}_\delta$.

\filbreak
\myTitle{Collective Type Interpretation}

An $M$-dimensional collective type $\delta = (D: \mathbb{N}^M, B: \mathbb{N}^M)$ is in \myKeyA{aligned form} if $\forall B_m, B_m \in \{1, D_m, \top\}$.
A thread collective $\mu$ matches $\delta$ if there exist subsets $C_m \subseteq [0, D_m - 1]_\mathbb{N}$ such that
\begin{itemize}
  \item $B_m \ne \top \implies \abs{C_m} = B_m$
  \item $\mu = \{ \ell_D(v) \mid v \in C_1 \times ... \times C_M \}$
\end{itemize}

An example of an aligned collective type is $D = (\lighttt{clutserDim},\;\lighttt{blockDim} / 32,\;32)$, $B = (\lighttt{clusterDim},\;1,\;32)$ which represents ``one warp selected from each CTA in the cluster''.
A specific thread collective that matches this collective type is ``warp 2 of each CTA in the cluster'', which satisfies the above criteria with
\begin{itemize}
  \item $C_1 = \{ 0, 1, ..., \lighttt{clusterDim}-1 \}$
  \item $C_2 = \{ 2 \}$
  \item $C_3 = \{ 0, 1, ..., 31 \}$
\end{itemize}

\filbreak
\myTitle{Collective Tiling Interpretation}

tl;dr we reorganize the cluster into an $M$-dimensional space, and hierarchically decompose each dimension.
\filbreak
We build collective tilings out of \myKeyA{dimension operators} $\mathcal{O}: \mathbb{Y} \times \mathbb{N}^5$, containing:
\begin{itemize}
  \item $\mathsf{iter}: \mathbb{Y}$
  \filbreak
  \item $\mathsf{linearBox}: \mathbb{N}$
  \filbreak
  \item $\mathsf{linearOffset}: \mathbb{N}$
  \filbreak
  \item $\mathsf{threadPitch}: \mathbb{N}$
  \filbreak
  \item $\mathsf{tileCount}: \mathbb{N}$
  \filbreak
  \item $\mathsf{treeDepth}: \mathbb{N}$
\end{itemize}
\filbreak
and usable in the function $\mathsf{evalCollDimOp}: \mathcal{O} \to (\mathbb{Y} \to \mathbb{Z}) \to [\mathbb{N}, \mathbb{N}] \to [\mathbb{N}, \mathbb{N}]$
\begin{gather*}
    (\mathsf{iter}, \mathsf{linearBox}, \mathsf{linearOffset}, \mathsf{threadPitch}, \mathsf{tileCount}, \mathsf{treeDepth}) \to
    \sigma_c \to
    [a, b]_\mathbb{N} \to
    [x, x + \mathsf{linearBox} - 1]_\mathbb{N} \\
    \text{where } x = a + \mathsf{linearOffset} + \sigma_c[\mathsf{iter}] \mathsf{linearBox}
\end{gather*}
$b$ is not explicitly used, but the compiler aims to statically guarantee that the output interval is a subset of the input interval $[a, b]$.

\filbreak
If the collective tiling corresponds to an $M$-dimensional domain, then the collective tiling consists of an $M$-tuple of \myKeyA{dimension descriptors} $\mathcal{D}: \mathbb{N} \times \mathbb{N} \times \mathcal{O}^*$.

\filbreak
Each $\mathcal{D}_m$ in $\omega = (\mathcal{D}_1, ..., \mathcal{D}_M)$ contains attributes
\begin{itemize}
  \item $D_m: \mathbb{N}$ (domain coordinate)
  \filbreak
  \item $P_m: \mathbb{N}$ (dimension thread pitch)
  \filbreak
  \item $\mathcal{O}_m$: ordered list of dimension operators $\mathcal{O}$
\end{itemize}
where the \myKeyA{dimension thread pitch} is defined by $P_m = \prod_{i = m+1}^M D_i$

\filbreak
Define $\mathsf{evalCollDim}: \mathcal{D} \to (\mathbb{Y} \to \mathbb{Z}) \to [\mathbb{N}, \mathbb{N}]$
as the mapping $\mathcal{D}_m \to \sigma_c \to [a', b']_\mathbb{N}$ defined by the pseudocode
\begin{itemize}
  \item $[a, b]_\mathbb{N} = [0, D_mP_m-1]_\mathbb{N}$
  \item For each $\mathsf{op}$ in order in $\mathcal{O}_m$, $[a, b]_\mathbb{N} = \mathsf{evalCollDimOp}(\mathsf{op}, \sigma_c, [a, b]_\mathbb{N})$
  \item Set $[a', b']_\mathbb{N} = [a / P_m, b / P_m]_\mathbb{N}$
\end{itemize}

\filbreak
The \myKeyA{domain} $D$ of the collective tiling is $(D_1, ..., D_M)$

\filbreak
The \myKeyA{box} $B$ of the collective tiling is $(B_1, ..., B_M)$ where
\begin{itemize}
  \item $B_m = D_m$ if $\mathcal{O}_m$ is empty.
  \item $B_m$ is $\mathsf{linearBox} / P_m$ of the last $\mathsf{op}$ in $\mathcal{O}_m$, otherwise.
\end{itemize}

\filbreak
The \myKeyA{resulting function} $f_\omega: (\mathbb{Y} \to \mathbb{Z}) \to \mathbb{T}$ of the collective tiling $\omega$ is defined by
\begin{gather*}
  \sigma_c \to \{ \ell_D(v) \mid v \in \mathcal{D}_1(\sigma_c) \times ... \times \mathcal{D}_M(\sigma_c) \}
\end{gather*}
\filbreak
The output thread collectives of this function match $\delta = (D, B)$ with domain and box as defined above; however, this is not necessarily the only collective type that the thread collectives match.
Corollary, a collective tiling with no operators $\mathcal{O}$ gives $\mu^\text{top}$.

\filbreak
\myTitle{Domain Completion}

\mainSub{Collective Tiling Reshape}

Since each dimension of a collective tiling supports only hierarchical decomposition, we need to provide the user with a way to reshape the dimensions of a collective tiling to implement non-hierarchical decompositions
(for example, assigning work to every other CTA in a cluster).

\filbreak
We split a dimension by a factor $f$ when we replace a single dimension descriptor $\mathcal{D}_m$ with a pair $\mathcal{D}_m', \mathcal{D}_{m+1}'$ defined by
\begin{itemize}
  \item $D_m' = D_m / f$
  \filbreak
  \item $D_{m+1}' = f$
  \filbreak
  \item $P_m' = P_m f$
  \filbreak
  \item $P_{m+1}' = P_m$
  \filbreak
  \item $\mathcal{O}_m' = [\mathsf{op} \in \mathcal{O}_m \mid \mathsf{op.linearBox} / P_m \ge f]$
  \filbreak
  \item $\mathcal{O}_{m+1}' = [\mathsf{op} \in \mathcal{O}_m \mid \mathsf{op.linearBox} / P_m < f]$
\end{itemize}
\filbreak
For the split to be well-formed, all divisions must produce natural numbers, and the $\mathsf{linearBox}$ and $\mathsf{linearOffset}$ of each dimension operator $\mathsf{op} \in \mathcal{O}_{m'}$ must be divisible by the $\mathsf{threadPitch}$ of the dimension descriptor $\mathcal{D}_{m'}$ that contains it.

\filbreak
A collective tiling is completed for a given \myKeyA{dimension thread pitch set} $S$ by splitting dimensions until the set of dimension thread pitches $\{P_1, ..., P_M\}$ is a non-strict superset of $S$.

\filbreak
Define $\mathsf{completeCollTiling}: \mathbb{N}^* \to \Omega \to \Omega$ as the mapping $D \to \omega \to \omega'$ where the output tiling is the input tiling completed for the dimension thread pitch set of the domain $D$.

\filbreak
\mainSub{Collective Type Reshape}

Define \myKeyA{unitary collective type completion} $\mathsf{completeCollType}_1: \mathbb{N}^* \to \Delta \to \Delta$ as the mapping $D^\text{env} \to (D, B) \to (D', B')$ where $(D', B')$ is defined as $(D, B)$ with the following modifications applied:

\filbreak
(1/3) For all dimension indices $m$, check that $1 \le B_m \le D_m$; the collective type is not well-formed otherwise.
Then, remove all dimensions where $D_m = 1$.

\filbreak
(2/3) Calculate the product $\prod_m D_m$ of the domain coordinates.
If not equal to $\prod_m D^\text{env}_m$ (which is expected to be \lighttt{clusterDim} $\times$ \lighttt{blockDim} for CUDA), prepend
\filbreak
\begin{itemize}
  \item domain coordinate $D'_1 = \prod_m D_m^\text{env} / \prod_m D_m$
  \filbreak
  \item box coordinate $B'_1 = 1$
\end{itemize}
\filbreak
(3/3) Then, split dimensions until the dimension thread pitch set of the collective type's domain is a non-strict superset of the dimension thread pitch set of $D^\text{env}$.
We split a dimension by a factor $f$ by replacing a domain coordinate $D_m$ and a box coordinate $B_m$ with a pair of coordinates $(D_m', D_{m+1}')$ and $(B_m', B_{m+1}')$:
\begin{itemize}
  \item $(D_m', D_{m+1}') = (D_m / f, f)$
  \filbreak
  \item $(B_m', B_{m+1}') = (\top, \top)$ if $B_m$ is $\top$
  \filbreak
  \item $(B_m', B_{m+1}') = (1, B_m)$ if $B_m < f$
  \filbreak
  \item $(B_m', B_{m+1}') = (B_m / f, f)$ if $B_m \ge f$
\end{itemize}
\filbreak
Define \myKeyA{agnostic collective type completion} $\mathsf{completeCollType}_\top: \mathbb{N}^* \to \Delta \to \Delta$
the same as above, except with $B'_1 = \top$ in the prepend step.

\filbreak
Define $\mathsf{align}: \Delta \to \Delta$; this splits dimensions until the collective type is an aligned collective type.

\filbreak
All of these functions can fail if a division fails to yield a natural number, or if no sequence of splitting dimensions yields the desired result.c

\filbreak
\mainSub{Domain Completion}

Domain completion operates on a pair of collective tiling and collective type.
This yields a new pair with the same domain and dimensionality, making them ``compatible'' with each other.
\filbreak
$\mathsf{domainCompletion_\top} : \Omega \times \Delta \to \Omega \times \Delta$
maps $(\omega, \delta) \to (\omega', \delta')$ where
\begin{itemize}
  \item Let $\delta' = \mathsf{completeCollType}_\top(D^\text{env}, \delta)$
    where $D^\text{env}$ is the domain of $\omega$.
  \filbreak
  \item Let $\omega' = \mathsf{completeCollTiling}(D, \omega)$
    where $D$ is the domain of $\delta$.
\end{itemize}
\filbreak
$\mathsf{domainCompletion_{align}} : \Omega \times \Delta \to \Omega \times \Delta$
maps $(\omega, \delta) \to (\omega', \delta')$ where
\begin{itemize}
  \item Let $\delta' = \mathsf{completeCollType}_1(D^\text{env}, \mathsf{align}(\delta))$
    where $D^\text{env}$ is the domain of $\omega$.
  \filbreak
  \item Let $\omega' = \mathsf{completeCollTiling}(D, \omega)$
    where $D$ is the domain of $\mathsf{align}(\delta)$.
\end{itemize}

\filbreak
\myTitle{Derived Collective Tilings}

We create a new collective tiling $\omega' \in \Omega$ from another by
\begin{itemize}
  \item Applying domain completion
  \filbreak
  \item Conditionally adding a new dimension operator ($\mathcal{O}$)
\end{itemize}
\filbreak
Define $\mathsf{derivedCollTiling}: \Omega \to \mathbb{Y} \to \mathsf{CollUnit} \to \mathbb{N} \to \mathbb{N} \to \mathbb{N} \to \Omega$ as
\begin{gather*}
  \mathsf{\omega^{env}} \to y \to \tau_u \to \mathsf{lo} \to \mathsf{hi} \to \mathsf{tileCount} \to \omega'
\end{gather*}
where we create $\omega'$ by
\filbreak
\begin{itemize}
  \item Let $\omega, \delta = \mathsf{domainCompletion_\top}(\mathsf{\omega^{env}}, \delta^\text{env})$ where $\delta^\text{env}$ is unpacked from $\tau_u$.
  \filbreak
  \item Let $D^\text{tiling}$, $B^\text{tiling}$ be the domain and box of $\omega$ and let $D, B$ be the domain and box of $\delta$.
  \filbreak
  \item The $m^{th}$ dimension is a \myKeyA{tiled dimension} when $B_m \notin \{ \top, D_m^\text{tiling}, B_m^\text{tiling} \}$
  \filbreak
  \item If no dimension is tiled,
  \begin{itemize}
    \item Require $\mathsf{lo}=0, \mathsf{hi}=1, \mathsf{tileCount} = 1$, otherwise the program is not well-formed.
    \item $\omega' = \omega$.
  \end{itemize}
  \filbreak
  \item If multiple dimensions are tiled, the program is not well-formed.
  \filbreak
  \item If a unique dimension $m$ is tiled, first check that $B_m \mathsf{hi} \le B_m^\text{tiling}$ (otherwise the program is not well formed; there are not enough threads), and define $\omega'$ as $\omega$ with the following $\mathsf{op}: \mathcal{O}$ appended to $\mathcal{O}_m$:
  \begin{itemize}
    \item $\mathsf{op.iter} = y$
    \filbreak
    \item $\mathsf{op.linearBox} = P_m B_m (\mathsf{hi} - \mathsf{lo}) / \mathsf{tileCount}$
    \filbreak
    \item $\mathsf{op.linearOffset} = P_m B_m \mathsf{lo}$
    \filbreak
    \item $\mathsf{op.threadPitch} = 0$ if $\mathsf{tileCount} \le 1$, otherwise $B_m P_m$
    \filbreak
    \item $\mathsf{op.tileCount} = \mathsf{tileCount}$
    \filbreak
    \item Set $\mathsf{op.treeDepth}$ to be greater than the $\mathsf{treeDepth}$ of any $\mathsf{op} : \mathcal{O}$ on any dimension of $\omega$
  \end{itemize}
\end{itemize}
\filbreak
Most statements have the same collective tiling as their parents.
The exceptions get listed below:

\filbreak
\mainSub{CudaDeviceFunction}

If \lighttt{clusterDim} = 1, then the initial collective tiling is 1 dimensional, with $D_0 = \lighttt{blockDim}$ and $P_0 = 1$.
If \lighttt{clusterDim} > 1, then the initial collective tiling is 2 dimensional, with $D_0 = \lighttt{clusterDim}$, $D_1 = P_0 = \lighttt{blockDim}$, $P_1 = 1$.
In both cases, no dimension operators $\mathsf{op} \in \mathcal{O}_m$ exist for any dimension $m$.

\filbreak
\mainSub{cuda\_threads Loop}

Evaluate $\mathsf{derivedCollTiling}$ with $\mathsf{\omega^{env}}$ being that of the for loop statement's scope, $y$, $\tau_u$, $\mathsf{lo}$ and $\mathsf{hi}$ taken directly from the loop, and $\mathsf{tileCount} = \mathsf{hi}$.
Currently, $\mathsf{lo}$ must be 0 and $\mathsf{hi}$ must be a positive constant int.
The body statements of the loop use the resulting $\omega'$.

\filbreak
\mainSub{Warp Specialization $\texttt{CudaWarps}(\mathsf{lo\_arg}, \mathsf{hi\_arg}, r)$}

This requires a bit more set up.
In the \lighttt{CudaDeviceFunction} launch, there was a tuple of named warp variables $r \in \mathbb{W}$, each specifying a warp count, which we will denote as $\mathsf{warpCount}(r)$ (the register counts are not relevant here).
Further, let $\mathsf{warpPrefix}(r)$ be the exclusive prefix sum of warp counts up to $r$.

\filbreak
If no parent statement of the warp specialization statement is a warp specialization statement, set $\mathsf{prefix} = \mathsf{warpPrefix}(r)$; otherwise, set $\mathsf{prefix}=0$, and require that the same $r$ be used for all parent warp specialization statements.
Evaluate $\mathsf{derivedCollTiling}$ with
\begin{itemize}
  \item $\mathsf{\omega^{env}}$ being the collective tiling of the warp specialization statement's scope.
  \filbreak
  \item $y$ being a ``dummy'' iterator variable that's always 0.
  \filbreak
  \item $\tau_u$ = \lighttt{cuda\_warp}
  \filbreak
  \item $\mathsf{lo} = \mathsf{prefix} + \mathsf{lo\_arg}$
  \filbreak
  \item $\mathsf{hi} = \mathsf{prefix} + \mathsf{hi\_arg}$
  \filbreak
  \item $\mathsf{tileCount} = 1$
\end{itemize}
to get the $\omega'$ for the child statements.

\filbreak
In the current implementation, the two integer parameters are inferred to be $0$ and $\mathsf{warpCount}(r)$ if not given.
We additionally check that
\begin{itemize}
  \item $\mathsf{lo\_arg} \ge 0$
  \filbreak
  \item $\mathsf{hi\_arg} \le \mathsf{warpCount}(r)$
  \filbreak
  \item If no parent statement of the warp specialization statement is a warp specialization statement, and more than one warp variable was specified in the \lighttt{CudaDeviceFunction} launch, then require that the warp specialization statement appears at \lighttt{cuda\_agnostic\_intact\_cta} scope.
\end{itemize}

\filbreak
\myTitle{Unit Matching}

To check if a $\omega^\text{in}: \Omega$ matches a collective type $\delta^\text{in}$, use $\mathsf{domainCompletion_{align}}(\omega^\text{in}, \delta^\text{in})$ to get $\omega, \delta$; let $B^\text{tiling}$, $B$ be the boxes of $\omega$ and $\delta$ respectively.

The collective tiling matches $\delta$ when for all dimensions $m$, $B_m \ne \top \implies B_m = B_m^\text{tiling}$.

\end{document}
