% xelatex </dev/null sketch_coll_analysis.tex

\documentclass[11pt]{article}

\usepackage[letterpaper, portrait, margin=2.54cm]{geometry}
\usepackage{enumitem}
\usepackage{amsmath}
\usepackage{amssymb}
\usepackage{amsfonts}
\usepackage{placeins}
\usepackage{graphicx}
\usepackage{listings}
\usepackage{caption}
\usepackage{colortbl}
\usepackage[parfill]{parskip}
\usepackage[mathscr]{euscript}
\usepackage[usenames,dvipsnames,svgnames,table,hyperref]{xcolor}
\usepackage[hidelinks]{hyperref}
\usepackage{fontspec}
\usepackage{mdframed}
\usepackage{tikz}
\usetikzlibrary{shapes.geometric, arrows, positioning, calc}

\setsansfont{FreeSans}
\setmonofont{Ubuntu Mono}

\hyphenation{WebGL}

%\definecolor{webColor}{RGB}{0, 124, 204}
\definecolor{webColor}{RGB}{0, 108, 174}
\newcommand{\web}[1]{{\color{webColor} \small \url{#1}}}
\newcommand{\webText}[2]{{\color{webColor} \href{#2}{#1}}}
\newcommand{\email}[2]{{\small \color{webColor} \textsf{\href{mailto:#1@#2}{#1[at]#2}}}}
\definecolor{titleColor}{RGB}{0, 0, 0}
\ifdefined\RESUME
\newcommand{\myTitle}[1]{{\large \color{titleColor} \hspace{-12mm} \textbf{\textsf{#1}}}}
\else
\newcommand{\myTitle}[1]{{ \vspace{2mm} \large \color{titleColor} \hspace{-12mm} \textbf{\textsf{#1}} \vspace{2mm}}}
\fi
\definecolor{subColor}{RGB}{170, 149, 0}
\newcommand{\mySub}[1]{{\color{subColor}\hspace{-9mm} \textsf{#1}}}
\definecolor{keyColor}{RGB}{118, 185, 0}
\newcommand{\myKey}[1]{{\color{keyColor}\textbf{#1}}}

\definecolor{keyColorA}{RGB}{170, 149, 0}
\definecolor{keyColorB}{RGB}{170, 210, 0}
\newcommand{\myKeyA}[1]{\textbf{\color{keyColorA}#1}}
\newcommand{\myKeyB}[1]{\textbf{\color{keyColorB}#1}}

\definecolor{mainColor}{RGB}{170, 0, 170}
\definecolor{minorColor}{RGB}{0, 149, 232}
\newcommand{\mainKey}[1]{\textbf{\color{mainColor}#1}}
\newcommand{\minorKey}[1]{\textbf{\color{minorColor}#1}}
\newcommand{\mainSub}[1]{{\color{mainColor}\hspace{-9mm} \textsf{#1}}}
\newcommand{\minorSub}[1]{{\color{minorColor}\hspace{-9mm} \textsf{#1}}}

\definecolor{hookColor}{RGB}{204, 0, 255}
\newcommand{\hook}[1]{{\color{hookColor}\textbf{#1}}}
\definecolor{flaggedBoxColor}{RGB}{255, 204, 204}
\newcommand{\flagged}[1]{{\color{black}\colorbox{flaggedBoxColor}{#1}}}

\definecolor{lightttColor}{RGB}{69, 69, 80}
\newcommand{\lighttt}[1]{{\color{lightttColor}\texttt{#1}}}
\newcommand{\graytt}[1]{{\color{gray}\texttt{#1}}}
\newcommand{\blacktt}[1]{{\color{black}\texttt{#1}}}

% Red and blue boxes have bright text color.
% Yellow, green, and violet have intentionally muted text colors.
% They used to be black, but it looks slightly better with a tiny bit of color.
\definecolor{redBoxFg}{RGB}{224, 0, 0}
\definecolor{redBoxBg}{RGB}{255, 211, 225}
\newcommand{\redBox}[1]{{\color{redBoxFg}\colorbox{redBoxBg}{#1}}}
\definecolor{yellowBoxFg}{RGB}{89, 89, 0}
\definecolor{yellowBoxBg}{RGB}{255, 232, 0}
\newcommand{\yellowBox}[1]{{\color{yellowBoxFg}\colorbox{yellowBoxBg}{#1}}}
\definecolor{greenBoxFg}{RGB}{0, 89, 63}
\definecolor{greenBoxBg}{RGB}{179, 232, 160}
\newcommand{\greenBox}[1]{{\color{greenBoxFg}\colorbox{greenBoxBg}{#1}}}
\definecolor{blueBoxFg}{RGB}{0, 97, 232}
\definecolor{blueBoxBg}{RGB}{224, 232, 255}
\newcommand{\blueBox}[1]{{\color{blueBoxFg}\colorbox{blueBoxBg}{#1}}}
\definecolor{violetBoxFg}{RGB}{108, 63, 124}
\definecolor{violetBoxBg}{RGB}{218, 204, 255}
\newcommand{\violetBox}[1]{{\color{violetBoxFg}\colorbox{violetBoxBg}{#1}}}

\mdfdefinestyle{MyFrame}{%
    linecolor=black,
    outerlinewidth=0pt,
    linewidth=0pt,
    innertopmargin=2.7pt,
    innerbottommargin=0pt,
    innerrightmargin=0pt,
    innerleftmargin=0pt,
        leftmargin = 0pt,
        rightmargin = 0pt}


\renewcommand*{\theenumi}{\alph{enumi}}
\renewcommand*\labelenumi{(\theenumi)}
\renewcommand*{\theenumii}{\roman{enumii}}
\renewcommand*\labelenumii{\theenumii.}

\sloppy
\newcommand{\abs}[1]{\left \vert #1 \right\vert}

% Some of these links are now dead...
\newcommand{\mbarrier}{\webText{mbarrier}{https://docs.nvidia.com/cuda/parallel-thread-execution/\#parallel-synchronization-and-communication-instructions-mbarrier}}
\newcommand{\cpAsync}{\webText{cp.async}{https://docs.nvidia.com/cuda/parallel-thread-execution/\#data-movement-and-conversion-instructions-cp-async}}
\newcommand{\cpAsyncBulk}{\webText{cp.async.bulk}{https://docs.nvidia.com/cuda/parallel-thread-execution/\#data-movement-and-conversion-instructions-cp-async-bulk}}
\newcommand{\fenceProxyAsync}{\webText{fence.proxy.async}{https://docs.nvidia.com/cuda/parallel-thread-execution/\#asynchronous-warpgroup-level-matrix-async-proxy}}
\newcommand{\wgmma}{\webText{wgmma}{https://docs.nvidia.com/cuda/parallel-thread-execution/\#asynchronous-warpgroup-level-matrix-instructions}}
\newcommand{\hopperBlog}{\webText{NVIDIA Hopper Architecture In-Depth}{https://developer.nvidia.com/blog/nvidia-hopper-architecture-in-depth/}}
\newcommand{\expectTxOperation}{\webText{expect-tx operation}{https://docs.nvidia.com/cuda/parallel-thread-execution/\#parallel-synchronization-and-communication-instructions-mbarrier-expect-tx-operation}}
\newcommand{\completeTxOperation}{\webText{complete-tx operation}{https://docs.nvidia.com/cuda/parallel-thread-execution/\#parallel-synchronization-and-communication-instructions-mbarrier-complete-tx-operation}}
\newcommand{\wgmmaFence}{\webText{wgmma.fence}{https://docs.nvidia.com/cuda/parallel-thread-execution/\#asynchronous-warpgroup-level-matrix-instructions-wgmma-fence}}


\begin{document}
\myTitle{Exo-GPU Collective Analysis (CollAnalysis)}

We begin by summarizing the types involved.
Many objects are defined here as functions, but they contain internal state that may be inspected (i.e. they are not just black-box functions).

\filbreak
\mainKey{Statement Node:} A statement within a program, akin to ``static instruction'' in computer architecture.

\filbreak
\mainKey{Statement Instance:} A given execution/interpretation of a statement node, akin to ``dynamic instruction'' in computer architecture.

\filbreak
\mainKey{Device Scope:} CPU or CUDA.

\filbreak
\mainKey{Natural thread index:} Integer uniquely identifying a thread within a CUDA cluster, given by \lighttt{cluster\_ctarank * blockDim.x + threadIdx.x}.
Exo only uses the $x$ block/thread dimension, and only has static analysis of thread assignment within clusters -- the mapping between Exo tasks and CUDA clusters is not statically analyzed.

\filbreak
\mainKey{Thread Collective:} $(\mu \in \mathbb{T})$ A set of threads within a cluster assigned to execute a specific statement instance.
Threads are identified by natural thread index.
Runtime concept.
The top-level-collective $\mu^\text{top}$ consists of all threads in the cluster.

For a given \lighttt{cuda\_threads} loop with at least two iterations, let $\mu_0$ and $\mu_1$ be thread collectives assigned to two consecutive iterations.
The \myKeyA{thread pitch} is $\min(\mu_1) - \min(\mu_0)$
(c.f. ``seat pitch''; we are measuring the distance between adjacent things).
The thread pitch is 0 for $\le 1$ iteration loops.

\filbreak
\mainKey{Collective Type:} $(\delta \in \Delta)$ A ``unit of measure'' describing a certain number and arrangement of threads (e.g. ``warp'' or ``CTA'') within one CUDA cluster.
An $M$-dimensional $\delta$ is a 2-tuple of domain $D: \mathbb{N}^M$ and box $B: \mathbb{N}_\top^M$; the product of the domain coordinates equals the number of threads in the cluster.
$\mathbb{N}_\top = \mathbb{N} \cup \{\top\}$.

Let $\mathbb{N}^M_\delta$ = $[0, D_1 - 1]_\mathbb{N} \times ... \times [0, D_M - 1]_\mathbb{N}$ .

The domain defines a ``linearize'' function $\ell_D: \mathbb{N}^M_\delta \to \mu^\text{top}$ mapping multidimensional coordinates to natural thread indices in lexicographical order.

$\mathbb{T}_\delta$ denotes the set of all thread collectives that are described by (match) the collective type $\delta$.

\filbreak
\mainKey{Collective Unit:} $(\tau_u \in \mathsf{CollUnit})$
Frontend concept, used to parameterize \lighttt{cuda\_threads} loops and hardware instructions.
An $M$-dimensional $\tau_u$ is a 2-tuple of domain $D: \mathsf{CollSizeExpr}^M$ and box $B: \mathsf{CollSizeExpr}^M_\top$, where a $\mathsf{CollSizeExpr}$ is a product of a rational number and zero or more of the collective parameters \lighttt{blockDim} and \lighttt{clusterDim}.
The ``unit typing function'' translates $\tau_u$ to $\delta$ through \myKeyA{collective parameter substitution} and \myKeyA{unit domain completion}.

\filbreak
\mainKey{Collective Tiling:} $(\mathbb{Y} \to \mathbb{Z}) \to \mathbb{T}$.
Static assignment of parallel loop iterations to thread collectives.
So basically the control environment determines which parallel loop iteration is being executed, and this gets translated to a thread collective for execution.

\filbreak
\myTitle{Scoped State}

Each scope either has (or is a) CPU or CUDA device scope.

\filbreak
A CUDA scope has additional attributes:
\begin{itemize}
  \item \lighttt{clusterDim} \& \lighttt{blockDim} collective parameter values (uniform throughout the device function)
  \item collective tiling: $(\mathbb{Y} \to \mathbb{Z}) \to \mathbb{T}$
  \item unit typing function: $\mathsf{unitTyping}: \mathsf{CollUnit} \to \Delta$ [derived from other attributes]
\end{itemize}

\filbreak
Note, the collective tiling is a \textit{static} property of a statement node; evaluating it (by passing in $\sigma_c$) yields a thread collective, which is a \textit{dynamic} property of a statement instance.

\filbreak
For a given $\tau_u : \mathsf{CollUnit}$, we say that a scope is a $\tau_u$-scope when its collective tiling always has output that matches $\delta$ where $\delta = \mathsf{unitTyping}(\tau_u)$.
In other words, this is the case when we can accurately type the collective tiling as $(\mathbb{Y} \to \mathbb{Z}) \to \mathbb{T}_\delta$.

\filbreak
\myTitle{Collective Type Interpretation}

An $M$-dimensional collective type $\delta = (D, B)$ is in \myKeyA{aligned form} if $\forall B_m, B_m \in \{1, D_m, \top\}$.
A thread collective $\mu$ matches $\delta$ if there exist subsets $C_m \subseteq [0, D_m - 1]_\mathbb{N}$ such that
\begin{itemize}
  \item $B_m = 1 \implies \abs{C_m} = 1$
  \item $B_m = D_m \implies C_m = [0, D_m-1]_\mathbb{N}$
  \item $\mu = \{ \ell_D(v) \mid v \in C_1 \times ... \times C_m \}$
\end{itemize}

\filbreak
\myTitle{Collective Tiling Interpretation}

tl;dr we reorganize the cluster into an $M$-dimensional space, and hierarchically decompose each dimension.

We build collective tilings out of $\mathsf{CollDimOp}: (\mathbb{Y} \to \mathbb{Z}) \times [\mathbb{N}, \mathbb{N}] \to [\mathbb{N}, \mathbb{N}]$, which contains attributes
\begin{itemize}
  \item $\mathsf{iter}: \mathbb{Y} \cup \{0\}$
  \filbreak
  \item $\mathsf{linearBox}: \mathbb{N}$
  \filbreak
  \item $\mathsf{linearOffset}: \mathbb{N}$
  \filbreak
  \item $\mathsf{threadPitch}: \mathbb{N}$
  \filbreak
  \item $\mathsf{tileCount}: \mathbb{N}$
  \filbreak
  \item $\mathsf{treeDepth}: \mathbb{N}$
\end{itemize}
and maps $(\sigma_c, [a, b]_\mathbb{N})$ to $[x, x + \mathsf{linearBox} - 1]_\mathbb{N}$ where $x = (a + \mathsf{linearOffset} + \sigma_c[\mathsf{iter}] \mathsf{linearBox})$.

\filbreak
$b$ is not explicitly used, but the compiler aims to statically guarantee that the output interval is a subset of the input interval $[a, b]$.

\filbreak
If the collective tiling corresponds to an $M$-dimensional domain, then the collective tiling consists of an $M$-tuple of $\mathsf{CollDim}: (\mathbb{Y} \to \mathbb{Z}) \to [\mathbb{N}, \mathbb{N}]$.

\filbreak
$F_m: \mathsf{CollDim}$ contains attributes
\begin{itemize}
  \item $D_m: \mathbb{N}$ (dim extent)
  \filbreak
  \item $P_m: \mathbb{N}$ (thread pitch)
  \filbreak
  \item $\mathsf{ops}_m$: ordered list of $\mathsf{CollDimOp}$
\end{itemize}
where the \myKeyA{thread pitch} is defined by $P_m = \prod_{i = m+1}^M D_i$

\filbreak
Define $F_m(\sigma_c)$ with the pseudocode
\begin{itemize}
  \item $[a, b] = [0, D_mP_m-1]$
  \item For each $\mathsf{op}$ in order in $\mathsf{ops}_m$, $[a, b] = \mathsf{op}(\sigma_c, [a, b])$
  \item Return $[a / P_m, b / P_m]$.
\end{itemize}

\filbreak
The \myKeyA{domain} $D$ of the collective tiling is $(D_1, ..., D_M)$

The \myKeyA{function} $(\mathbb{Y} \to \mathbb{Z}) \to \mathbb{T}$ of the collective tiling is defined by
\begin{equation}
  \sigma_c \to \{ \ell_D(v) \mid v \in F_1(\sigma_c) \times ... \times F_M(\sigma_c) \}
\end{equation}

\filbreak
\myTitle{Domain Completion}

\filbreak
\myTitle{Unit Typing Function}

\filbreak
\myTitle{Derived Collective Tilings}

\end{document}
