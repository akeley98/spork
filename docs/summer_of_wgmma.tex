% python3 code_to_tex.py su_samples.py su_samples && xelatex summer_of_wgmma.tex </dev/null
\documentclass[11pt]{article}

\usepackage[letterpaper, portrait, margin=2.54cm]{geometry}
\usepackage{enumitem}
\usepackage{amsmath}
\usepackage{amssymb}
\usepackage{amsfonts}
\usepackage{placeins}
\usepackage{graphicx}
\usepackage{listings}
\usepackage{caption}
\usepackage{colortbl}
\usepackage[parfill]{parskip}
\usepackage[mathscr]{euscript}
\usepackage[usenames,dvipsnames,svgnames,table,hyperref]{xcolor}
\usepackage[hidelinks]{hyperref}
\usepackage{fontspec}
\usepackage{mdframed}
\usepackage{tikz}
\usetikzlibrary{shapes.geometric, arrows, positioning, calc}

\setsansfont{FreeSans}
\setmonofont{Ubuntu Mono}

\hyphenation{WebGL}

%\definecolor{webColor}{RGB}{0, 124, 204}
\definecolor{webColor}{RGB}{0, 108, 174}
\newcommand{\web}[1]{{\color{webColor} \small \url{#1}}}
\newcommand{\webText}[2]{{\color{webColor} \href{#2}{#1}}}
\newcommand{\email}[2]{{\small \color{webColor} \textsf{\href{mailto:#1@#2}{#1[at]#2}}}}
\definecolor{titleColor}{RGB}{0, 0, 0}
\ifdefined\RESUME
\newcommand{\myTitle}[1]{{\large \color{titleColor} \hspace{-12mm} \textbf{\textsf{#1}}}}
\else
\newcommand{\myTitle}[1]{{ \vspace{2mm} \large \color{titleColor} \hspace{-12mm} \textbf{\textsf{#1}} \vspace{2mm}}}
\fi
\definecolor{subColor}{RGB}{170, 149, 0}
\newcommand{\mySub}[1]{{\color{subColor}\hspace{-9mm} \textsf{#1}}}
\definecolor{keyColor}{RGB}{118, 185, 0}
\newcommand{\myKey}[1]{{\color{keyColor}\textbf{#1}}}

\definecolor{keyColorA}{RGB}{170, 149, 0}
\definecolor{keyColorB}{RGB}{170, 210, 0}
\newcommand{\myKeyA}[1]{\textbf{\color{keyColorA}#1}}
\newcommand{\myKeyB}[1]{\textbf{\color{keyColorB}#1}}

\definecolor{mainColor}{RGB}{170, 0, 170}
\definecolor{minorColor}{RGB}{0, 149, 232}
\newcommand{\mainKey}[1]{\textbf{\color{mainColor}#1}}
\newcommand{\minorKey}[1]{\textbf{\color{minorColor}#1}}
\newcommand{\mainSub}[1]{{\color{mainColor}\hspace{-9mm} \textsf{#1}}}
\newcommand{\minorSub}[1]{{\color{minorColor}\hspace{-9mm} \textsf{#1}}}

\definecolor{hookColor}{RGB}{204, 0, 255}
\newcommand{\hook}[1]{{\color{hookColor}\textbf{#1}}}
\definecolor{flaggedBoxColor}{RGB}{255, 204, 204}
\newcommand{\flagged}[1]{{\color{black}\colorbox{flaggedBoxColor}{#1}}}

\definecolor{lightttColor}{RGB}{69, 69, 80}
\newcommand{\lighttt}[1]{{\color{lightttColor}\texttt{#1}}}
\newcommand{\graytt}[1]{{\color{gray}\texttt{#1}}}
\newcommand{\blacktt}[1]{{\color{black}\texttt{#1}}}

% Red and blue boxes have bright text color.
% Yellow, green, and violet have intentionally muted text colors.
% They used to be black, but it looks slightly better with a tiny bit of color.
\definecolor{redBoxFg}{RGB}{224, 0, 0}
\definecolor{redBoxBg}{RGB}{255, 211, 225}
\newcommand{\redBox}[1]{{\color{redBoxFg}\colorbox{redBoxBg}{#1}}}
\definecolor{yellowBoxFg}{RGB}{89, 89, 0}
\definecolor{yellowBoxBg}{RGB}{255, 232, 0}
\newcommand{\yellowBox}[1]{{\color{yellowBoxFg}\colorbox{yellowBoxBg}{#1}}}
\definecolor{greenBoxFg}{RGB}{0, 89, 63}
\definecolor{greenBoxBg}{RGB}{179, 232, 160}
\newcommand{\greenBox}[1]{{\color{greenBoxFg}\colorbox{greenBoxBg}{#1}}}
\definecolor{blueBoxFg}{RGB}{0, 97, 232}
\definecolor{blueBoxBg}{RGB}{224, 232, 255}
\newcommand{\blueBox}[1]{{\color{blueBoxFg}\colorbox{blueBoxBg}{#1}}}
\definecolor{violetBoxFg}{RGB}{108, 63, 124}
\definecolor{violetBoxBg}{RGB}{218, 204, 255}
\newcommand{\violetBox}[1]{{\color{violetBoxFg}\colorbox{violetBoxBg}{#1}}}

\mdfdefinestyle{MyFrame}{%
    linecolor=black,
    outerlinewidth=0pt,
    linewidth=0pt,
    innertopmargin=2.7pt,
    innerbottommargin=0pt,
    innerrightmargin=0pt,
    innerleftmargin=0pt,
        leftmargin = 0pt,
        rightmargin = 0pt}


\renewcommand*{\theenumi}{\alph{enumi}}
\renewcommand*\labelenumi{(\theenumi)}
\renewcommand*{\theenumii}{\roman{enumii}}
\renewcommand*\labelenumii{\theenumii.}

\sloppy
\newcommand{\abs}[1]{\left \vert #1 \right\vert}

% Some of these links are now dead...
\newcommand{\mbarrier}{\webText{mbarrier}{https://docs.nvidia.com/cuda/parallel-thread-execution/\#parallel-synchronization-and-communication-instructions-mbarrier}}
\newcommand{\cpAsync}{\webText{cp.async}{https://docs.nvidia.com/cuda/parallel-thread-execution/\#data-movement-and-conversion-instructions-cp-async}}
\newcommand{\cpAsyncBulk}{\webText{cp.async.bulk}{https://docs.nvidia.com/cuda/parallel-thread-execution/\#data-movement-and-conversion-instructions-cp-async-bulk}}
\newcommand{\fenceProxyAsync}{\webText{fence.proxy.async}{https://docs.nvidia.com/cuda/parallel-thread-execution/\#asynchronous-warpgroup-level-matrix-async-proxy}}
\newcommand{\wgmma}{\webText{wgmma}{https://docs.nvidia.com/cuda/parallel-thread-execution/\#asynchronous-warpgroup-level-matrix-instructions}}
\newcommand{\hopperBlog}{\webText{NVIDIA Hopper Architecture In-Depth}{https://developer.nvidia.com/blog/nvidia-hopper-architecture-in-depth/}}
\newcommand{\expectTxOperation}{\webText{expect-tx operation}{https://docs.nvidia.com/cuda/parallel-thread-execution/\#parallel-synchronization-and-communication-instructions-mbarrier-expect-tx-operation}}
\newcommand{\completeTxOperation}{\webText{complete-tx operation}{https://docs.nvidia.com/cuda/parallel-thread-execution/\#parallel-synchronization-and-communication-instructions-mbarrier-complete-tx-operation}}
\newcommand{\wgmmaFence}{\webText{wgmma.fence}{https://docs.nvidia.com/cuda/parallel-thread-execution/\#asynchronous-warpgroup-level-matrix-instructions-wgmma-fence}}


\tikzstyle{smallnode} = [rectangle, minimum width=1.25cm, minimum height=1cm, text centered, text width=1.25cm, draw=black, fill=white]
\tikzstyle{smallishnode} = [rectangle, minimum width=2cm, minimum height=1cm, text centered, text width=2cm, draw=black, fill=white]
\tikzstyle{normalnode} = [rectangle, minimum width=3cm, minimum height=1cm, text centered, text width=3cm, draw=black, fill=white]
\tikzstyle{widenode} = [rectangle, minimum width=62mm, minimum height=8mm, text centered, text width=62mm, draw=black, fill=white]
\tikzstyle{bignode} = [rectangle, minimum width=3.5cm, minimum height=2cm, text centered, text width=3cm, draw=black, fill=white]
\tikzstyle{smemnode} = [rectangle, minimum width=3cm, minimum height=1cm, text centered, text width=3cm, draw=keyColorB, fill=white]
\tikzstyle{gmemnode} = [rectangle, minimum width=3cm, minimum height=1cm, text centered, text width=3cm, draw=keyColorA, fill=white]
\tikzstyle{smallishsmemnode} = [rectangle, minimum width=2cm, minimum height=1cm, text centered, text width=2cm, draw=keyColorB, fill=white]
\tikzstyle{arrow} = [thick,->,>=stealth]
\tikzstyle{line} = [thick]

\begin{document}
\myTitle{Summer Plans for Spork}

For now I'm just focusing on proposed changes to the Exo object language.
I also have to figure out how to update the synchronization checker to handle new features, and glue it to Exo.
After that, implementing algorithms in Exo-GPU and performance results (sounds like part of this work may be outsourced to Jason).

New/changed features:

\begin{itemize}
\item Instructions are currently conveyed as Python functions; as an alternative, we should also support ``instruction templates'' conveyed as Python classes.
These instruction templates can be instantiated to create a family of similar instrs, and give us more room syntactically to fit all the instruction metadata needed for Exo-GPU.
\item Move away from string interpolation and towards more ``structured'' interfaces for generating windows and C/C++ instruction syntax.
\item Changes to distributed memory to support multicasting and warp-shuffle instructions, i.e., cases where it's no longer true that each distributed shard is accessed solely by its owner.
\item Possibly, replace the actor kind idea with timelines: \textit{instruction timeline} (\lighttt{instr\_tl}) and \textit{usage timeline} (\lighttt{usage\_tl}), which are per-instruction and per-instruction-parameter attributes, respectively.
\item Improvements to Arrive/Await statements to support TMA and multicast: we need to allow arriving on multiple barriers, and associating a barrier with a specific instruction, and no prior instructions, in an instruction stream.
\end{itemize}

\newpage
\myTitle{Instr Class}

``New-style'' instruction templates are created as \lighttt{@instr}-decorated Python classes, rather than functions.
The class must have two functions
\begin{itemize}
  \item \lighttt{behavior}: Exo code specifying the behavior of the function.
  \item \lighttt{instance}: Python code executed by the compiler and codegen.
\end{itemize}

\input{su_samples/instr_class.0.tex}

\filbreak
The decorated class is not itself an instr, but an instr template.
All parameters in common between the \lighttt{behavior} and \lighttt{instance} functions (in this case, \lighttt{K}) must be substituted with concrete values.
These are ``template parameters''.
The other \lighttt{behavior} parameters are ``runtime parameters''.

\input{su_samples/instr_class.1.tex}

\filbreak
Besides allowing code-reuse through the template mechanism and class inheritance, conveying the \lighttt{instr} as a class gives us more room syntactically (in the \lighttt{instance} function) to convey the large amount of metadata for Exo-GPU instructions:
\begin{itemize}
  \item collective unit
  \filbreak
  \item include files and ``utils'' for C and CUDA C++
  \filbreak
  \item instruction timeline (n\'ee actor kind)
  \filbreak
  \item per-parameter information: usage timeline (``actor signature''), sync/async access
\end{itemize}

\filbreak
\mainSub{Exo Syntax}

Pass template parameters as keyword arguments, and runtime parameters as positional:

\input{su_samples/instr_class.2.tex}

\filbreak
\mainSub{LoopIR\_Unification Changes}

For convenience, usually the user shouldn't substitute template parameters manually.
Instead, we should program \lighttt{replace()} to be able to take an instr template, and deduce and substitute template parameters as part of unification.
It's an error if the substituted value is not a constant.

\filbreak
\mainSub{Include Files \& Utility Code}

We will have a lot of similar instructions that may share snippets of useful ``global'' source code.
As a replacement for \lighttt{c\_global}, the \lighttt{instance} function of an instr class configures these attributes defining source code needed for compiling the instrs:
\begin{itemize}
\item \lighttt{c\_includes: List[str]}: names of include files needed in C code (the \lighttt{.c} file)
\item \lighttt{c\_utils: List[str]}: C code snippets
\item \lighttt{cu\_includes: List[str]}: names of include files needed in CUDA C++ code (the \lighttt{.cuh} file)
\item \lighttt{cu\_utils: List[str]}: CUDA C++ code snippets
\end{itemize}
Duplicate include files/utils are discarded.

\filbreak
Because the header file (\lighttt{.cuh}) for the generated CUDA code is not really private, all the CUDA utilities are placed into a ``unique'' namespace (uniquify based on Python source file name).
This is aliased as \lighttt{namespace exo\_CudaUtil} in generated CUDA functions.

\filbreak
\mainSub{Instr Format Function}

The current system of generating C (or CUDA) code for instrs using \lighttt{str.format} is really unweildly.
As an alternative, let's support codegen with a Python function that's given information on all runtime parameters and returns a C/C++ string.
This ties into my proposed changes to the window system that allow deeper introspection of windows.

\newpage
\myTitle{Make Windows Better}

Windows serve two purposes in Exo
\begin{itemize}
 \item In the common usage, we use windows to pass slices of tensors to instrs.
 Typically, the instr's code format does not actually create a window, and just formats using \lighttt{\textit{arg\_name}\_data}.
 I call this a \myKeyA{non-materialized} window.
 \item Much less commonly, we actually \myKeyA{materialize} a window struct, such as when we cross a true function boundary, or we compile a \lighttt{WindowStmt} (can of worms I won't get into in this document).
\end{itemize}
\filbreak
The uncommon materialized window case puts artificial constraints on our thinking.
I propose we make it much easier to unpack and inspect attributes of a window within the Exo compiler, with limited (or no) support for materializing the window, depending on memory type.
Currently, this is only possible for the limited case of \lighttt{\textit{arg\_name}\_data} (inspecting the offseted pointer for a window) and the weird \lighttt{\textit{arg\_name}\_int} thing.

\filbreak
Furthermore, we should have language-level support for expressing the different ``hardware modes'' for tensor dimensions (distributed memory, array, packed/vectorized), and enforcing commonplace restrictions (e.g. that an AVX vector isn't strided).
I think this is better than trying to repeatedly implement hacky assertions with \lighttt{stride}, etc., in \lighttt{Memory}/\lighttt{SpecialWindow} implementations.

\filbreak
\mainSub{Dimension Hardware Modes}

From left-to-right, a tensor's dimensions consist of \redBox{distributed}, \yellowBox{array}, and \blueBox{packed} (``vectorized'') dimensions.
The distributed dimensions are deduced by Exo-GPU, and mostly hidden from the window implementation; the Exo-GPU compiler rewrites the subtree to eliminate distributed dimensions prior to C/C++ codegen.
The packed dimensions are defined by the \lighttt{Memory} type (e.g. 8, for \lighttt{\_\_m256}).
The remaining dimensions (in the middle) are array dimensions.

\myKeyA{AVX Example:}

\input{su_samples/avx_example.0.tex}

\filbreak

\myKeyA{CUDA Example:}

\input{su_samples/warp_example.0.tex}

\filbreak
For window parameters, we can explicitly tag the hardware modes, from left-to-right:
\begin{itemize}
  \item ``\lighttt{distributed:}'' rarely used; needed for multicast and warp shuffle
  \filbreak
  \item ``\lighttt{strided:}'' array dimension, with arbitrary stride
  \filbreak
  \item ``\lighttt{dense:}'' array dimension, densely packed
  \filbreak
  \item ``\lighttt{packed:}'' packed dimension
\end{itemize}

\filbreak

\filbreak
\myKeyA{Backwards Compatibility:} If not tagged, query the number of packed dimensions $P$ from the \lighttt{Memory} type, and assume the right-most up-to $P$ dimensions are \lighttt{packed}, and the rest \lighttt{strided}.

\filbreak
\myKeyA{AVX Example:}

\input{su_samples/avx_window.0.tex}

\filbreak
\myKeyA{TMA GMEM Example:}

\input{su_samples/tma_window.0.tex}

\filbreak
\myKeyA{TMA SMEM Example:}

\input{su_samples/tma_window.1.tex}

\filbreak
\mainSub{Expected Patterns}

For Exo-GPU, there's several usage patterns for windows:

\filbreak
\myKeyA{Pointer + strides:} Only thing that is supported today.
Allocated in C/C++ as a flattened array.
We can freely load/store scalars, and reduce the dimensionality of a window.
``Mutable'' pointer: we do pointer arithmetic to implement offsets.
These windows may be materialized as structs.

\input{su_samples/pointer_strides.0.tex}

\filbreak
\myKeyA{Register-like:} Allocated in C/C++ as a multidimensional array.
Cannot be materialized as a window struct, or passed to non-instr procs.

\input{su_samples/register_alloc.0.tex}

\filbreak
\myKeyA{Pointer + offsets:}
These windows can be materialized as structs, but with an ``immutable'' pointer: the offsets of the window are encoded explicitly.
The dimensionality of the window can't be changed (i.e. we support only intervals, not points)

\input{su_samples/pointer_offsets.0.tex}
\end{document}
